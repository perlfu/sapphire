\documentclass[10pt]{article}

\usepackage{listings}
\newcommand{\code}[1]{\lstinline!#1!}

\title{Guide to the MMTk refactoring}
\author{Robin Garner\\Daniel Frampton}

\begin{document}

\lstset{language=Java, basicstyle=\ttfamily\small}

\maketitle

\section{Introduction}

This is a guide to the recent MMTk refactoring, intended to help people
who have existing MMTk plans to refactor these plans to fit the new
structure. 

\section{The new structure}

The most obvious change is that an MMTk plan has now been split into 3
or more classes.  The main aim of this refactoring is to enable the
use of instance methods for methods that used to be static, and hence
make better use of inheritance to structure a plan.

Previously, system-global data was kept in static fields of a plan,
and thread-local data in instance fields of a plan.  Now, the
separation has become stronger by separating thread-local data into a
separate class hierarchy.  

\subsection{Collection}

The mechanism used for performing garbage collection has become much
more general.  Each (global and local) plan class now provides a
method called \code{collectionPhase}, which takes an integer parameter
\code{phaseId}.  The \code{StopTheWorld} class defines the phases for
a stop-the-world collection, and processes them sequentially at each
garbage collection, calling the \code{collectionPhase} method of the
plan for each phase.  This method follows a common pattern, and for
some phases, the plan will perform actions required by this layer, and
then delegate to the super-class, while for the remaining phases it
will simply delegate to the superclass. This is considerably more
flexible than the \code{threadLocalPrepare} etc methods provided by
the old system.

Collection phases are represented by a \code{Phase} object that
specifies a timer name and whether the phase has local and/or global
components, and in which order they are executed.  

\subsection{Tracing}

The core of most garbage collections is the operation of tracing the
heap.  This has been separated into a separate class hierarchy,
allowing collectors such as the generational collectors to create
separate nursery and full-heap trace classes rather than have complex
methods that behave differently in the different cases.

As a consequence, policy and VM interface classes that participate in
tracing now also take \code{TraceLocal} parameters.

\subsection{Interfacing with the VM}

The most visible change in the interface is that the \code{Plan} class
has been replaced with three classes: \code{ActivePlan},
\code{ActivePlanLocal} and \code{ActivePlanConstraints}.  These classes
extend the corresponding selected plan class in much the same
way as the old \code{Plan} class.

Selecting the plan at build time is now done by substituting a value
rather than by a long series of boolean options.  For example,
\code{config/build/gc/SemiSpace} now has,
\begin{lstlisting}
   export RVM_WITH_JMTK_PLAN="org.mmtk.plan.semispace.SS"
\end{lstlisting} 
and the various \code{ActivePlan} classes construct their superclass
names according to the naming scheme outlined below.

Code that needs to access the global or local plan instances can call
methods of ActivePlan to locate the correct object.

\section{Refactoring an existing plan}

This section describes in outline how to fit an existing plan into the
new structure.  It goes though the process of changing the existing
MarkSweep collector to work in the new structure.

\begin{enumerate}
\item Create a package for the plan.  For example, the marksweep collector now
  lives in the package \code{org.mmtk.plan.marksweep}.

\item Create 4 classes,
  \begin{itemize}
  \item The global class, in this case \code{MS}, which will extend
  \code{StopTheWorld} in the case of a full-heap collector, or
  \code{Gen} for a classic generational collector.

  \item The local class, eg \code{MSLocal}, which will extend
  \code{StopTheWorldLocal} or \code{GenLocal} respectively.  The name
  of the class must be constructed by appending 'Local' to the global
  class name for the build process to work.

  \item A constraints class, eg \code{MSConstraints}, extending
  the appropriate Constraints superclass, (again, the naming scheme is
  mandatory) and

  \item One or more trace classes, eg \code{MSTraceLocal}, extending
  \code{TraceLocal} or any other trace classes that may
  be necessary for the collector.  For example, generational
  collectors define a \code{MatureTraceLocal} class, and inherit a
  \code{NurseryTraceLocal} class from the \code{generational} package.
  The naming scheme for trace classes is not mandatory.
  \end{itemize}

\end{enumerate}

Now start moving fields and methods from the original source file to
the appropriate new class.

\subsection{The Global Class}
\begin{enumerate}
\item Move constant definitions (eg \code{ALLOC_MS}) to the global
  class, as well as space definitions and the corresponding space
  descriptors.  Generally every static field of the old class will
  become a static field of the new global class.

\item Create an instance field in the global class, like this
\begin{lstlisting}
  public final Trace msTrace;
\end{lstlisting}


This is the global trace object, where the shared queues used during
tracing etc are held.  The constructor for the global trace class
should create the instance.

\item Create a minimal global collectionPhase method, and include the
  contents of the existing globalPrepare() and globalRelease() methods.
\begin{lstlisting}
  public final void collectionPhase(int phaseId) 
      throws InlinePragma {
    if (phaseId == PREPARE) {
      super.collectionPhase(phaseId);
      // contents of the globalPrepare() method
      return;
    }
    if (phaseId == RELEASE) {
      // contents of the globalRelease() method
      super.collectionPhase(phaseId);
      return;
    }
    super.collectionPhase(phaseId);
  }
\end{lstlisting}
  Note that PREPARE should probably delegate to super before doing
  plan-specific things, and RELEASE afterwards.  The available
  collection phases are defined in \code{StopTheWorld}.

\item Copy the \code{poll} method across.  This will probably require
  no changes.

\item Add any necessary accounting methods.  At a minimum,
  getPagesUsed should add in the contribution of any spaces defined 
  in this level of the plan hierarchy, eg
\begin{lstlisting}
  public int getPagesUsed() {
    return (msSpace.reservedPages() + super.getPagesUsed());
  }
\end{lstlisting}
  Note that \code{getPagesReserved} (in \code{Plan}) is now 
  expressed in terms of \code{getPagesUsed} and
  \code{getCopyReserve}.  These methods are all instance methods.
\end{enumerate}

This is probably all that is required for the global plan class.

\subsection{The Local Class}
\begin{enumerate}
\item Define a convenience method in the Local class called global(),
  like this
\begin{lstlisting}
  private static final MS global() throws InlinePragma {
    return (MS)ActivePlan.global();
  }
\end{lstlisting}


\item Move allocator definitions to the new local class.

\item Define a \code{TraceLocal} object like this
\begin{lstlisting}
  private MSTraceLocal trace;
\end{lstlisting}
initialised in the constructor like
\begin{lstlisting}
    trace = new MSTraceLocal(global().msTrace);
\end{lstlisting}
and define a method
\begin{lstlisting}
  public final TraceLocal getCurrentTrace() {
    return trace;
  }
\end{lstlisting}
to return the current trace.  A plan with more than one trace method
will need this method to select the correct one---for example a
generational plan should return either the nursery or full-heap trace
object as required.

\item Copy the allocation methods \code{alloc}, \code{postAlloc} (and
  for copying collectors 
  \code{allocCopy} and \code{postCopy}) to the local class.  The body
  of the methods should be changed so that the spaces defined at this
  level of the plan hierarchy are handled, and then delegate to
  super, eg
\begin{lstlisting}
  public Address alloc(int bytes, int align, int offset, int allocator)
    throws InlinePragma {
    if (allocator == MS.ALLOC_DEFAULT) {
      return ms.alloc(bytes, align, offset, false);
    }
    return super.alloc(bytes, align, offset, allocator);
  }
\end{lstlisting}

  \code{allocCopy} and \code{postCopy} now also take an allocator
  as a parameter.  Non-copying plans no longer need to provide 
  \code{allocCopy} and \code{postCopy}.

\item Create a collectionPhase method - this takes 2 more parameters
  than the global equivalent.
\begin{lstlisting}
  public final void collectionPhase(int phaseId, boolean participating,
                                    boolean primary)
    throws InlinePragma {
    if (phaseId == MS.PREPARE) {
      super.collectionPhase(phaseId, participating, primary);
      // Contents of threadLocalPrepare()
      trace.prepare();
      return;
    }    
    if (phaseId == MS.START_CLOSURE) {
      trace.startTrace();
      return;
    }

    if (phaseId == MS.COMPLETE_CLOSURE) {
      trace.completeTrace();
      return;
    }

    if (phaseId == MS.RELEASE) {
      // Contents of threadLocalRelease()
      trace.release();
      super.collectionPhase(phaseId, participating, primary);
      return;
    }
  super.collectionPhase(phaseId, participating, primary);
  }
\end{lstlisting}
  The \code{primary} flag is used to distinguish one processor in case
  certain work needs to be done once only, but in a thread-local
  context.  Use this flag instead of adding additional rendezvous
  points and checking the arrival index.

\item Override \code{getSpaceFromAllocator} and
  \code{getAllocatorFromSpace}.  These should look something like
\begin{lstlisting}
  public Space getSpaceFromAllocator(Allocator a) {
    if (a == ms) return MS.msSpace;
    return super.getSpaceFromAllocator(a);
  }
\end{lstlisting}

  ie they handle spaces defined at this level, and then delegate.

\end{enumerate}

\subsection{The Trace Class}

\begin{enumerate}
\item The constructor should look like this
\begin{lstlisting}
  public MSTraceLocal(Trace trace) {
    super(trace);
  }
\end{lstlisting}

\item Copy the \code{isLive} method to the trace class, make it an
  instance method, and simplify it to reflect the standard pattern of
  handling local stuff and delegating to super.  

\item Copy the \code{traceObject} method across and make it an
  instance method.  Note that the \code{traceObject} method in a
  policy now takes a TraceLocal as its first parameter.

\item A copying collector requires a precopyObject method, and should
  override the willNotMove method.

\end{enumerate}

\subsection{The Constraints Class}

The \code{Constraints} class simply renames the existing
\code{Constants} class, and brings the method naming into line with
standards.  

This class exists so that the host VM can discover certain static
properties of the current MMTk plan without introducing additional
class initialization dependencies.

\subsection{Generational collectors}

A \code{Global} class for a generational collector must provide a
\code{activeMatureSpace()} method, and a \code{Local} class must
provide a \code{getFullHeapTrace()} method, but is structurally no
different to a full-heap collector.

 

\end{document}
