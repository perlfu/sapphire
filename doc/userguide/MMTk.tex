%      $Id$    

This section provides information on the implementation of JMTk, the
memory management component of Jikes\TMweb{} RVM's runtime system.

{\bf The following discussion is out of date; for instance, JMTk has been
made independent of Jikes RVM, and is now called MMTk.  Much more work
has been done on factoring out the VM-specific code into the
packages \texttt{org.mmtk.vm}
and \texttt{com.ibm.JikesRVM.memoryManagers.mmInterface}}. 

From Jikes RVM 2.2.0 onward, JMTk (Java Memory management
Toolkit) became the default memory management system for Jikes RVM.\@
JMTk is designed to be a portable toolkit, with Jikes
RVM-specific code factored out as far as possible.

\subsection{Directory Structure and Packages} \label{sssec:directories}

MMTk classes are contained in the
\texttt{\$RVM\_\-ROOT/org/mmtk} directory.

In keeping with the goal of modularity and portability, as far as
possible Jikes-RVM-specific code is factored out.  Such VM-specific
code resides in a separate package,
(\xlink{\texttt{\MMpackage\-.\mmInterface}}{\mmInterfacePackageURL}),
and can be found in the \texttt{\mmInterface} sub-directory.  All other
code is part of the
\xlink{\texttt{org.mmtk}}{\JMTkPackageURL}
package.

The \texttt{plan} sub-directory contains classes that define
\emph{memory management plans}.  A plan specifies a particular
configuration of JMTk components which together define the memory
management strategy for a particular build of Jikes RVM.\@  The
\texttt{policy} sub-directory contains classes implementing various
memory management policies (such as \xlink{mark-sweep
    collection}{\MarkSweepLocalURL}, \xlink{free-list
    allocation}{\SegregatedFreeListURL}, \xlink{bump-pointer
    allocation}{\BumpPointerURL}, etc.).  The \texttt{utility}
sub-directory contains classes implementing generic utilities (such as
\xlink{load-balancing parallel dequeues}{\SharedDequeURL},
\xlink{sequential store buffers}{\LocalSSBURL}, etc.).

\subsection{Choosing a Garbage Collector} \label{ssec:choosinggc}

Depending on your purposes, you may choose to build Jikes RVM with one
of the following plans:
\begin{itemize}
\item \texttt{SemiSpace} (copying),
\item \texttt{MarkSweep} (non-copying),
\item \texttt{CopyMS} (non-generational copy/mark-sweep hybrid)
\item \texttt{GenCopy} (classic copying generational),
\item \texttt{GenMS} (generational with mark-sweep mature space), or
\item \texttt{RefCount} (a reference counting collector with non-concurrent cycle collection)
\item \texttt{NoGC} (allocation only, no garbage collection)
\item \texttt{GCTrace} (copying collector used for heap trace generation)
\end{itemize}
The relative performance of these collectors is highly dependent on
the application. 

\texttt{NoGC} is provided for pedagogical and
experimental purposes only; it is not suitable for general use.
\texttt{GCTrace} is provided for generation of GC traces and is not 
useful for collector comparisons.
GC researchers will of course want to experiment with other GC choices.
The \texttt{RefCount} collector is currently slightly less stable than
the others and is not quite ready for heavy usage.

All of the memory managers (except \texttt{NoGC} and \texttt{RefCount}) 
support finalization and weak, soft, and phantom references.  They are all parallel 
and load-balancing.  A collection can proceed even when some threads are
executing in native code. When a collection starts, threads in native
code are blocked from returning to Java code for the duration of that
collection.

\subsection{Plans}%
        \label{sssec:plans}%
        \index{garbage collection}%
        \index{stop-the-world garbage collection}

\IndexttClass{BasePlan}%
All plans inherit from \xlink{\texttt{BasePlan}}{\BumpPointerURL},
and all plans in this release are ``stop-the-world'' collectors, so
they all inherit from
\IndexttClass{StopTheWorldGC}%
\xlink{\texttt{StopTheWorldGC}}{\StopTheWorldGCURL}, which implements
basic stop the world GC functionality.  The two generational
collectors both inherit from
\IndexttClass{Generational}%
\xlink{\texttt{Generational}}{\GenerationalURL}, which includes
common nursery and write barrier implementations.

All JMTk plans support parallel allocation and collection (with the
exception of \IndexTexttt{RefCount}, which does not perform parallel
collection).  To minimize synchronization overheads, unsynchronized
\emph{thread-local} actions are distinguished from \emph{global}
actions, which must only be performed by a single thread.  Global
state is held in each plan's class variables, while instance variables
reflect thread-local state, each \texttt{Plan} instance bound to a
\xlink{\IndexTexttt{VM\_Processor}}{\VMProcessorURL} instance.

The basic functions of each plan include:
\begin{itemize}
\item Identifying a virtual memory layout (using
  \xlink{\texttt{VMResource}}{\VMResourceURL} to, for example, bind a
  semi-space or the nursery to a particular address range).  It is
  important to note that only the \emph{virtual memory} layout is
  partitioned statically.  Actual \emph{memory usage} is dynamically
  spread among the various spaces (such as nursery, mature, metadata,
  etc.), the only constraint being that the total memory usage remain
  within the memory available.
\item Providing allocation by binding suitable allocators to different
  \texttt{VMResource}s.
\item Invoking collection when necessary through the use of a
  \emph{polling} mechanism.
\item Applying the appropriate collection policies to objects
  encountered during the collection process (objects may be subject to
  different collection regimens depending on where they reside in
  memory).
\item Implementing read and write barriers if necessary.
\end{itemize}

It is not difficult to add your own memory management plan (allocator
and collector) to JMTk, especially if it uses the same ``stop the
world'' parallel collection strategy used by all the collectors in
this release.  A good way to start is to compare some of the different
plans and understand the significance of the differences.

The basic steps are:

\cindex[jconfigure script]{\texttt{jconfigure} script}%
\IndexttClass{Plan}%
\begin{enumerate}
\item Create a new directory within the \texttt{plan} subdirectory, such as
  ``\texttt{NewGC}''.
\item Add a new configuration in \texttt{\$RVM\_\-ROOT/rvm/config/build}
  which includes your new directory in the build.  Name it
  appropriately, such as ``\texttt{BaseBaseNewGC}''.
\item Modify \texttt{\$RVM\_\-ROOT/bin/jconfigure} to handle the new
  collector sub-directory.
\item Copy the contents of some existing plan directory into your new
  directory, and modify the files (\texttt{Plan.java} and
  \texttt{Header.java}), choosing a starting point that has similar
  properties, such as copying or non-copying, generational or
  non-generational.
\end{enumerate}

\subsection{Policy} \label{sssec:policy}

The \texttt{policy} sub-directory contains implementations of key
memory management policy choices, such as \xlink{bump-pointer
    allocation}{\BumpPointerURL} and \xlink{free-list
    allocation}{\MarkSweepLocalURL}, \xlink{copying
    collection}{\CopyURL} and \xlink{mark-sweep
  collection}{\MarkSweepSpaceURL}.

\subsection{Utility} \label{sssec:utility}

The \texttt{utility} sub-directory contains basic utilities and
mechanisms, including:
\begin{itemize}
\item \xlink{Load-balancing shared parallel
    dequeues}{\SharedDequeURL}, which provide load balancing double
  headed queue management for \xlink{address
    dequeues}{\AddressDequeURL} (used by the GC work queue), and
  \xlink{sequential store buffers}{\LocalSSBURL} (used by some write
  barriers as a remembering mechanism).
\item \xlink{Memory resources}{\MemoryResourceURL}, which are the
  mechanism for space \emph{accounting}.  Memory resources are used
  for accounting for all space, including space used by meta data
  (such as queues, etc.).
\item \xlink{Virtual Memory Resources}{\VMResourceURL} (VM
  Resources), which are mechanism for virtual memory \emph{mapping}.
  VM resources are used to associate regions of virtual memory with
  particular policies or needs (such as the nursery of a generational
  collector, or a region of memory where meta data resides, etc), and
  allow multiple allocators to consume each such space. There are a
  number of different VM resources, for \xlink{monotonic
      allocation}{\MonotoneVMResourceURL} (used by bump pointer
  allocators), \xlink{free list allocation}{\FreeListVMResourceURL}
  (used by free list allocators), etc.
\item Low level tools for \xlink{\texttt{mmap}ping}{\LazyMmapperURL}
  memory on demand, and for \xlink{allocating raw
      memory}{\RawPageAllocatorURL} (for use by meta data, for example).
\end{itemize}

\subsection{\texttt{VMInterface}} \label{sssec:vminterface}

The \texttt{mmInterface} sub-directory provides the interface between
Jikes RVM and JMTk.  The primary interface is provided by
\xlink{\texttt{VM\-In\-ter\-face\-.java}}{\VMInterfaceURL}.  Key
VM-dependent mechanisms include:

\begin{itemize}
\item \xlink{Object}{\ScanObjectURL},
  \xlink{statics}{\ScanStaticsURL}, and
  \xlink{thread}{\ScanThreadURL} scanning,
\item \xlink{GC map iteration}{\VMGCMapIteratorURL}, and
\item GC \xlink{initiation}{\VMCollectorThreadURL} and \xlink{synchronization}{\VMHandshakeURL}.
\end{itemize}

\subsection{\texttt{Generating GC Traces}} \label{sssec:gctrace}
Builds using the \texttt{GCTrace} plan produce accurate heap traces of
the running program.  These builds output records for every object
creation (both objects in the boot image and heap allocations),
reference update, and the last time each object is reachable.
Generated traces are output as normal garbage collection information
and can be captured similarly.

To record the exact state of the heap, the trace format's object
creation record differentiates objects created in the boot image,
objects allocated into the immortal region of the heap, and the
remaining object allocations.  To enable accurate garbage collection
simulations, last reachable times are recorded for all these objects.

Last reachable times for each object are recorded at a granularity
measured in terms of bytes allocated.  The earliest an object can be
reclaimed is at the next ``perfectly accurate'' allocation following
this time.  To compute these times relatively quickly, JMTk uses the
Merlin lifetime analysis algorithm.  For more information about trace
granularity or the Merlin algorithm, see the SIGMETRICS 2002 paper by 
Hertz, Blackburn, McKinley, Moss, and Stefanovic.

Trace generation defaults to using the largest possible granularity,
but finer traces can be specified by specifying the \textbf{logarithm}
of the desired granularity with the \texttt{traceRate} command-line
option.  Because garbage collection cannot occur at every allocation
(i.e., before the Jikes RVM has finished booting), traces include
``weak'' allocations no matter the granularity specified.  At the
finest granularity, these ``weak'' allocations record only when
garbage collection is prohibited and objects cannot therefore be
reclaimed.

