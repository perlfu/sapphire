There are several types of source code in the Jikes RVM system.  They
include:


\subsection{Java} 
($\approx 1052$ source files as of this writing, November 21, 2003)  This is
the vast majority of the code.  The names of Java source files always end in
`{\tt .java}''.   We \link{have already discussed Java coding style}[ in
section~\Ref{} (page~\Pageref)]{section:javacodingstyle}. 

We have found it necessary to preprocess the Java code; \link{the
preprocessor is discussed in the \SectionName{Java Preprocessor} section}[ 
  (Section~\Ref), on page~\Pageref]{section:preprocessor}.


\subsection{C}

($\approx 27$ source files)  The file names always
end in ``{\tt .c}''.  We are committed to GNU C.\@  One of the core team
members uses the latest stable GNU C release (3.3.2 as of this writing), but many users
are still running GNU C 2.95 (which came out in July 29, 1999).  So if
you use constructs that are not accepted by GNU C 2.95, then please surround them with
appropriate preprocessor conditionals.  It is OK to use GCC 2.95;
that's the oldest version we'll support.

Our C style is straight Kernighan and Ritchie, with four-space
indenting.  

\subsection{C++} 

($\approx 15$ source files)  The file names always end in
``{\tt .C}''.   The material discussed for C applies here as well.
Some older C++ source files don't meet the coding standard yet.

\subsection{Bourne-Again Shell (Bash)}

We use the GNU Bourne-Again Shell (Bash) exclusively in our shell
scripts ($\approx 13$ source files).  Our build system is driven by
one massive Bash script, {\tt /bin/jconfigure}, which in turn
generates other scripts in the \varName{RVM\_BUILD} directory.

\subsubsection{Indenting}

We use four-space indenting for Bash.  One source file contains an
older indentation style.
We limit lines to 80 columns unless necessary for syntactic reasons to
do otherwise. 

\subsubsection{Declaring functions}

We use the {\tt function} keyword to declare functions:
\begin{verbatim}
function emitCopier () {
\end{verbatim}
rather than eliding it:
\begin{verbatim}
emitcopier () {
\end{verbatim}

\subsubsection{Exit in case of Build Error}

An important consideration for the build process is that if trouble
happens while buiding a part of the system, the build should abort
rather than continuing on.  We have encountered several problems with this,
where the build process continued on despite trouble building the GNU
Classpath library, and users then got an RVM that did not work.  

The Bash source code used in the build system is protected with 
{\tt set~-e}, which causes the shell to immediately exit in case of
trouble, and (on Bash versions that support it) with a {\tt trap}
against the {\tt ERR} pesudo-signal.  You can see an example of this
in the function {\tt emit\_\-enable\_\-exit\_\-on\_\-err\-or} in {\tt
rvm/\-bin/\-j\-con\-fi\-gure}.  

\paragraph{Subshells}

{\tt -e} and {\tt ERR} do not apply to commands executed within
subshells.  So, if you want to execute Make in the directory
{\it directory-name}, it's more robust to use:
\begin{example}
\tt{}make -C {\it directory-name} {\it{}make-target}
\end{example}
instead of:
\begin{example}
\tt{}( cd {\it directory-name} && make {\it{}make-target})
\end{example}
If you want to use a subshell in your build-system code, then try:
\begin{example}
\tt{}({\it subshell-commands }) || false
\end{example}
which will make the right thing happen, or:
\begin{example}
\tt{}if ! ({\it subshell-commands }); then 
    echo >&2 "\$ME: Something bad happened."
    exit 1
fi

\end{example}

\paragraph{{\bf \tt unset} fails on nonexistent variables}
The construct ``{\tt unset} \(\mbox{\textit{variable}}_1\)\ldots'' will
fail if any \(\mbox{\textit{variable}}_i\) is not already set.
If this could happen, you should guard the construct as in this
example:
\begin{example}
\texttt{unset \textit{variablename} || :}
\end{example} 
where the shell command ``\texttt{:}'' (colon) always exits with true
status.



\subsubsection{Miscellany}

\subsection{GNU Makefiles} 

We are committed to GNU Make.  All new makefiles should be named {\tt
GNUmakefile}, \link{just as discussed in the Sun\Rheadingweb{} style guide}[
  mentioned in section~\Ref, on
  page~\Pageref]{section:javacodingstyle}.  We are committed to
pure Bash, so Makefile writers can safely use Bash constructs.


