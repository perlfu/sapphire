The vast majority of Jikes\TMweb{} RVM's source code is in the Java\TMweb{}
programming language ($\approx 1052$ source files as of this writing,
November 21, 2003, including examples and tools.)  This section
describes a set of 
coding style guidelines that we recommend for all Java code added to
Jikes RVM.\@  We shall subsequently visit the other programming languages.

This section also includes coding conventions that we
use within Jikes RVM.\@  

\subsection{Character Set}

The Java source-to-byte-code compilers are invoked with the argument
``\texttt{-encoding iso-8859-1}''.  Unless you have a good reason
for doing so, please use the ISO Latin 1 (iso-8859-1) character set
when you create source files.  If you do have to write a source file
with another encoding, please be sure that the Java
source-to-byte-code compilers are invoked with the proper
\texttt{-encoding} flag.  They may not break on your platform with the
wrong -encoding flag, or with a missing -encoding flag, but they might
break on another platform or when run by someone with a different set of
locale settings.

\subsection{Exit Codes}

Inside the VM itself, and in any code that can be called by the VM,
please do not exit with low-numbered failure exit codes (1, 2,
etc.).  Those codes are often used by user programs.  

The standard exit codes for the Jikes RVM VM are defined in the
interface VM\_ExitStatus.java.  Please use them.  These rules also
apply to the C code that forms part of Jikes RVM.  Most of the exit
status codes (you can add the rest if you need them) are exported by
the program GenerateInterfaceDeclarations.java into the generated
include file \texttt{InterfaceDeclarations.h}.

This convention does not apply to tools, such as the boot image writer
or GenerateFromTemplate.  Those tools are independent programs
themselves, and do not run arbitrary Java code.  For them,
System.exit(1) is appropriate.  Also, the Java preprocessor has its
own exit convention, for which see the code.

Java code should generally exit by calling System.exit.  That is
because System.exit will run the user program's shutdown hooks, as
well as any shutdown hooks that might be installed by static
initializers in library classes (there are none such at the moment).
Internal errors, and impossible conditions, should usually call
VM.sysFail, which will print a stack trace.  Also, do not call
System.exit if your code might be invoked before the VM is fully
booted.

Please do not use the exit status \texttt{-1}.  \texttt{-1} is not
handled properly by the Cygwin libraries (it gets mapped to zero,
meaning success) and will bite us if we ever port to that platform.

\subsection{Coding Conventions}

\subsubsection{The {\tt VM\_} prefix}

By convention, any class which should be loaded into the boot image
starts its name with {\tt VM\_}.  This is used by the build configurator,
{\tt jconfigure}, which uses that information to create {\tt
\$RVM\-\_\-BUILD\-/\-R\-V\-M\-.pri\-mor\-di\-als}, the file containing the list of classes
that should be written into \jrvm{}'s  boot image.

\link{The MMTk classes}[ (see Section~\Ref, page~\Pageref)]{section:MMTk} don't follow this naming convention
because they're an independent subsystem, but the rest of \jrvm{} does.

\subsubsection{Assertions}
\label{assertions}

Partly for historical reasons, we use our own built-in assertion
facility rather than the one that appeared in Sun\Rweb{}'s JDK 1.4.   All
assertion checks have one of the two forms:
\begin{example}
\tt{}    if (VM.VerifyAssertions)  VM.\_assert({\it condition}
\tt{}    if (VM.VerifyAssertions)  VM.\_assert({\it condition}, {\it message})
\end{example}
{\tt VM.VerifyAssertions} is a {\tt 
public static final} field.  The \varName{RVM\_WITHOUT\_ASSERTIONS}
configuration variable determines {\tt VM.VerifyAssertions}' value.
If \varName{RVM\_WITHOUT\_ASSERTIONS} is enabled, \jrvm{} has no assertion
overhead. 

If you use the form without a {\it message}, then the default message
``{\tt vm internal error at:}''  will appear.  

If you use the form with a {\it message} the message {\em must} be a
single string literal.  Doing string appends in assertions can be a
source of horrible performance problems when assertions are enabled
(ie most development builds).  If you want to provide a
more detailed error message when the assertion fails, then you must
use the following coding pattern:
\begin{example}
\tt{}   if (VM.VerifyAssertions && {\it condition}) VM._assert(false, {\it message});
\end{example}

An assertion failure is always followed by a stack dump.


\subsection{Coding Style Introduction}

Regrettably, much code in the current system does not follow any
consistent coding style.  This is an unfortunate residuum of the
system's evolution.  It makes editing sometimes unpleasant, and
prevents Javadoc\TMweb{} from formatting comments in many files.  To alleviate
this problem, we present this style guide for new Java\TMweb{} code; it's just
a small tweak of Sun\Rweb{}'s style guide.

\subsection{File Headers}

Every file needs to have a header with a copyright notice, a Javadoc\TMweb{}
{\tt @author} tag, and an RCS/CVS {\tt \$Id\$} entry. There may be
multiple Javadoc {\tt @author} tags, and there may
additionally be a {\tt @modified} tag for someone who modified code but
doesn't want to claim co-authorship.  ({\tt @modified} is our own
extended tag.)  There is usually also a Javadoc {\tt @date} field for
the date of authorship.  There are skeleton files for Java, C/C++,
Bash (Bourne Shell), \texttt{make}, and \texttt{m4} in the source directory
``\texttt{etc/skel/}''; just cut-and-paste the header text from them.  

A Java example of the notices follows.

\begin{example}
/* -*-coding: iso-8859-1 -*-
 *
 * This file is part of Jikes RVM (http://jikesrvm.sourceforge.net).
 * The Jikes RVM project is distributed under the Common Public License (CPL).
 * A copy of the license is included in the distribution, and is also
 * available at http://www.opensource.org/licenses/cpl1.0.php
 *
 * (C) Copyright � IBM Corp 2003
 *
 * \$Id$
 */
package com.ibm.jikesrvm;       // FILL ME IN with the appropriate package.

import com.ibm.jikesrvm.classloader.VM_ClassLoader; // FILL ME IN 

/** TODO Substitute a brief description of what this program or library does.
 *
 * @author Your Name Here
 * @date 11 October 2003
 */
\end{example}

We add the Javadoc authorship tags even to non-Java\TMweb{} source
files, such as shell scripts and C programs.

\index{coding style}%
\index{Java source code style}%
\subsection{Coding style description}

The Jikes\TMweb{} RVM coding style guidelines are defined with
reference to \xlink{the Sun\Rweb{} 
Microsystems ``Code Conventions for the Java\TMweb{} Programming Language''}{\SunCodeConventionURL},
with a few exceptions listed below.\texonly{  The Sun coding
conventions can be found at 
\texttt{\SunCodeConventionURL} in HTML,
PostScript\Rweb{}, and PDF.\@}  Most of the style guide is intuitive; 
however, please read through the document (or at least look at its sample code).

We have adopted four modifications to the Sun code conventions:
\begin{enumerate}
\index{indenting}
\item {\bf Two-space indenting} The Sun coding convention suggests 4
space indenting; however with 80-column lines and four-space indenting,
there is very little room left for code.  Thus, we recommend using 2
space indenting.

\item {\bf 132 column lines in exceptional cases} The Sun coding convention is
that lines be no longer than 80 columns.  Several Jikes RVM
contributors have found this constraining.  Therefore, we allow 132
column lines for exceptional cases, such as to avoid bad line breaks.

\item {\bf \tt if (VM.VerifyAssertions)}
As a special case, the condition {\tt if (VM.VerifyAssertions)} is
usually immediately followed by the call to {\tt VM.\_assert()},
with a single space substituting for the normal
newline-and-indentation.  \link{There's an example \htmlonly{elsewhere
in this document}}[ in section~\Ref, on page~\Pageref]{assertions}.    

\item {\bf Capitalized fields} 
Under the Sun coding conventions, and as specified in 
{\em The Java Language Specification, Second Edition}, the names of
fields begin with a lowercase letter.  (The only exception they give
is for some {\tt final static} constants, which have names
ALL\-\_\-IN\-\_\-CA\-PI\-TAL\-\_\-LET\-TERS, with underscores separating them.)  That
convention reserves
IdentifiersBeginningWithACapitalLetterFollowedByMixedCase for the
names of classes and interfaces.  However, most of the {\tt final}
fields in the {\tt VM\_Configuration} class and the {\tt VM\_\-Pro\-per\-ties}
interface also are in that format.   

Since the {\tt VM} class inherits
fields from both {\tt VM\_Properties} and {\tt VM\_Configuration},
that's how we get {\tt VM.VerifyAssertions}, etc.
%
% If you run: find ${RVM_ROOT}/rvm -name \*.java -print0 | xargs -0 -e egrep -n -e '[^\.]\.[A-Z]+[_a-z0-9][_A-Za-z0-9]+\>[	 ]*=[^=]'  | grep -v sysWrite | egrep -v -e '^[[:space:]]*\*'
% you'll discover 58 lines in the source tree where we also have
% variables that don't meet the specification.  Bummer.

\end{enumerate}

\paragraph{Spacing and Tabs}

As of the \jrvm{} 2.3.1 release, the Java, C, and C++ source code will
use only spaces for indentation, not tabs.  This is because an IDE
that a potential contributor uses does not work well with code
containing tab characters.  

\subsection {Javadoc requirements}
\index{Javadoc}

All files should contain descriptive comments
\xlink{in Javadoc\TMweb{} form}[ (\texttt{\JavadocURL})]{\JavadocURL}
so
that documentation can be generated automatically.  Of course,
additional non-Javadoc source code comments should appear as
appropriate.
% For Javadoc, at a minimum,

\begin{enumerate}
\item All classes and methods should have a block comment describing
them
\item All methods contain a short description of their arguments
(using {\tt @param}), the return value (using {\tt @return}) and the
exceptions they may throw (using {\tt @throws}).
\item Each class should include {\tt @see} and {\tt @link} 
references as appropriate.
\end{enumerate}

\subsection{Useful Tools and Hints}\index{editing source code}

This section describes helpful hints for conforming with the style
guide.  Below are suggestions on how to setup the two most common
editors, Emacs and \texttt{vi}. 
%\remark{If we find a pretty-print code processor, we
%can describe it in this section.}

\subsubsection{Emacs}\index{Emacs}

\paragraph{jikes-rvm.el}

Load the contents of the
file \texttt{RVM\_ROOT/etc/jikes-rvm.el}.  This will set up your
Emacs to use Jikes RVM's coding conventions for Java, C, and C++ code.

The easiest way to enable Emacs to do this automatically is to copy
(cut-and-paste) the contents
of \texttt{RVM\_ROOT/etc/dot-emacs.el} into the \texttt{.emacs}
file in your home directory.

\texttt{dot-emacs.el} also contains sample customizations for telling
Emacs to: 

\begin{itemize} 
\item \textbf{use fonts and colors} to highlight your
code syntactically (this is generally a Good Thing).

\item \textbf{truncate long lines} instead of wrapping them.

\end{itemize}

If you use \texttt{M-x compile} to run jbuild, you will want to have
{\tt jikes} produce error messages Emacs can parse. To do this supply
the \texttt{+E} option in the definition of \texttt{JIKES} in your
\texttt{\$RVM\_HOST\_CONFIG} file. For example:

\begin{example}
\tt{}export JIKES="/usr/bin/jikes +E"
\end{example}

\subsubsection{vi}%
        \label{options:vi/vim}%
        \cindex[vi]{\texttt{vi} and \texttt{vim}}

If you are more comfortable with {\tt vi}, we recommended that you
use \xlink{\texttt{vim}, a {\tt vi} clone}[ (\texttt{\VimURL})]{\VimURL}.
\texttt{vim}
contains all of {\tt vi}'s commands and is fully backward compatible,
but is much more configurable than {\tt vi}.  \link{Hints for {\tt vi}
diehards who absolutely refuse to use {\tt vim} are at the end
of this subsection}{options:vi}.

\paragraph{vim}\label{options:vim}


The file \texttt{RVM\_ROOT/etc/jikes-rvm.vim} contains
customizations for editing Jikes RVM source code.  It customizes Java
mode, C++ mode, and C mode.  Since this file will probably improve
with time, I recommend you load it in indirectly from the source tree
rather than cutting and pasting its contents into your \texttt{.vimrc} file.

To set yourself up to load it indirectly, add the contents
of ``\texttt{RVM\_ROOT/etc/dot-vimrc.vim}'' to a file in your home
directory named ``\texttt{.vimrc}''.

\paragraph{vi}\label{options:vi}

Standard {\tt vi} options that would approximate Java\TMweb{} formatting are:

\begin{verbatim}
set shiftwidth=2  " for indenting and shifting
set autoindent    " automatically indent new lines to the start of previous
\end{verbatim}
and the approximation for wrapping long lines is
\begin{verbatim}
set wrapmargin=6  " to allow autowrap text at 74th column
\end{verbatim}

% LocalWords:  Javadoc param

