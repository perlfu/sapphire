This subsection briefly describes the overall structure of \jrvm.\@
Details of the various subsystems are provided in subsequent
sections.  

\subsection{Major Components of \jrvm}


\jrvm{} can be divided into the following major components:
\begin{description}
\item[Core runtime] (thread scheduler, class loader, library support,
verifier, etc.) This component is responsible for 
managing all the underlying data
structures required to execute applications and interfacing with
libraries.

\item[Compilers] (baseline, optimizing, JNI) This component is
responsible for generating executable code from bytecodes.

\item[Memory managers] This component is responsible for the
allocation and collection of objects during the execution of an
application. 

\item[Adaptive optimization system] This component is responsible
for profiling an executing application
and judiciously using the optimizing compiler to
improve its performance.
\end{description}

More details of each of these components are provided in the following sections.

\subsection{Package Structure}
The 2.2.0 release introduced packages into the system.  Prior releases
did not use packages explicitly, which resulted in all classes being in
the unnamed package.  

%There are currently nine packages in \jrvm.\@  All classes are in
%one of these packages
The packages in \jrvm{} include:
%% XXX TODO Add other packages here.
\begin{description}
\item[\texttt{com.ibm.jikesrvm}] Classes for the core runtime, except for library
support.  This package also contains other classes that are not
included in one of the other packages, such as the baseline compilers. 

\item[\texttt{com.ibm.jikesrvm.adaptive}] Classes for the adaptive optimization system

\item[\texttt{com.ibm.jikesrvm.classloader}] Implementation of classloaders
and associated data structures including the VM representation of classes,
methods, etc. 

\item[\texttt{com.ibm.jikesrvm.jni}] Implementation of JNI

\item[\texttt{org.mmtk}] Classes in the newer MMTk
(Memory Management Toolkit) collection of memory managers
%% TODO: Document:
%% package org.mmtk.plan;
%% package org.mmtk.policy;
%% package org.mmtk.utility.alloc;
%% package org.mmtk.utility.deque;
%% package org.mmtk.utility.gcspy.drivers;
%% package org.mmtk.utility.gcspy;
%% package org.mmtk.utility.heap;
%% package org.mmtk.utility.options;
%% package org.mmtk.utility.scan;
%% package org.mmtk.utility.statistics;
%% package org.mmtk.utility;
%% package org.mmtk.vm.gcspy;
%% package org.mmtk.vm;

\item[\texttt{\MMpackage\-.\mminterface{}}] Classes related to
memory management that deal with the interface to the VM

\item[\texttt{com.ibm.jikesrvm.opt}] Classes related to
the \link{optimizing compiler}%
%[  (See Section~\Ref{} (page~\Pageref) for more details)]%
{section:optdetails}, except for IR-related classes

\item[\texttt{com.ibm.jikesrvm.opt.ir}] Classes related to the IR
(intermediate representation) of the \link{optimizing compiler.}%
%[ See Section~\Ref{} (page~\Pageref) for more details.]%
{section:optdetails}

\item[\texttt{com.ibm.jikesrvm.OSR}] Classes related to On-Stack-Replacement. 

\item[\texttt{com.ibm.jikesrvm.quick}] Classes related to
the \link{Quick Compiler}{section:quick}.

\item[\texttt{com.ibm.jikesrvm.eclipse.jdt.launching}] Classes related
to launching Jikes RVM as a subsidiary VM under Eclipse. 
%% XXX Link me!

\item[\texttt{com.ibm.jikesrvm.eclipse.ui}] Classes related
to the control panel for the Jikes RVM plugin for Eclipse.

\item[\texttt{org.vmmagic.pragma}] Classes for the \link{VM Magic Pragmas (Pragmata)}{uninterruptible_code}


\item[\texttt{org.vmmagic.unboxed}] Types for directly representing
machine addresses and other such concepts in Java.


\end{description}

The distributed directory structure
does not follow the Java\TMweb{} convention that the source file directory tree
match the package structure, {\it i.e.},
there is no {\tt com/ibm/jikesrvm} directory anywhere under \texttt{\$RVM\_ROOT/rvm}.  Instead, the source directory structure follows a more logical 
structure.  The boot image builder
copies the source files from the \texttt{\$RVM\_ROOT/rvm} tree into a build
directory.  The scripts that perform this copy create the directory
structure required by Java semantics and place classes in appropriate
directories.

This approach avoids the need to change the directory structure as the
package structure evolves to become more fine-grained. If we
eventually arrive at a more fine-grained package structure (so that
each package contains a reasonably small number of classes) we may
convert the source tree to a package-oriented structure to facilitate
using standard Java IDE's to edit Jikes RVM source code.
