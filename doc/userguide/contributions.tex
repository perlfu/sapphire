This section contains guidelines on the process of
submitting a contribution to MMTk.

\index{contributions}

\subsection{MMTk Contributions}

We encourage contributions to MMTk in any of these forms: additional
components or collectors, generalization of architecture,
and instrumentation infrastructure.  Because of the sometimes significant
maintenance cost of memory managers, we expect submitted contributions
to adhere to the following guidelines:


\begin{enumerate}

\item {\bf Coding style}  
MMTk judiciously uses modularity and abstraction to lower maintenance
cost and increase code clarity.  We expect contributions, especially
large ones, to follow the same discipline.  Specifically, we highly
encourage reuse of existing components or contributing patches to
augment the functionality of existing components.  Since MMTk is
designed to be a separable module within the VM, MMTk contributions
should adhere to certain modularity constraints.  Namely,
communication with the rest of the VM should occur only through the
classes {\tt VM\_Interface} and {\tt MM\_Interface} and certain magic classes
({\tt VM\_Address}, {\tt VM\_Word}, {\tt VM\_Offset}, and {\tt VM\_Extent}).

\item {\bf Being up-to-date}  
Contributions should be with respect to SVN head or something
reasonably close at least for the affected files.

\item { \bf Stability}  
New collectors or components are expected to be stable with respect to
all GC tests on multiple platforms and on SMP's with both strong ({\it
e.g.}\ Intel\Rweb{}) and weak ({\it e.g.}\ Power) memory models.  Contributions
that are not collectors but provide new functionality such as
instrumentation should provide their own regression tests.
We expect submitted collectors to work for all regression tests.
As a submitter, a good rule of thumb is that contributors should begin
testing with the following test suites: bytecodeTests, gctest, 
SPECjvm\Rweb{}98, and SPECjbb\Rweb{}2000.  \link{See
the \SectionName{Regression Tests} section}[ (\Ref,
page~\Pageref)]{sec:regression} for more details. 

\item {\bf Timely Maintenance}  
The contributor of a significant contribution is expected to be the
advocate of the system and will fix the submitted components if and
when defects are discovered.  We expect serious failures (failure rate
> 15\%) to be fixed within a few days or a week.  Other failures (low
failure rate or non-deterministic failures) should be fixed before a
release.  The core team will strive to announce planned releases with
at least 3 weeks notice.

\item {\bf Advocacy Change}  
If the contributor does not wish to maintain the contributed
component, then the core team may or may not choose to take over
responsibility of the system.  The former is more likely if the
submitted system is general and of value to multiple Jikes users.  
In those cases, a new maintainer may volunteer.  If not and if the
stability of the system degrades, the system may eventually be
removed.

\end{enumerate}
