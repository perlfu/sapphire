%% This is the text of the Common Public License version 1.0, put
%% into LaTeX format by Steven Augart on 5 May 2004.
%% This is the body of a subsection in LaTeX.
%% So, the highest-level sectioning command we'll use here is
%%\subsubsection*{}. 

%%
%% We use the * form of the sectioning commands since we need to
%% assign our own numbers -- they are part of the legal text of this document.
%%
\begin{center}
{\bf Common Public License - v 1.0}
\end{center}

THE ACCOMPANYING PROGRAM IS
PROVIDED UNDER THE TERMS OF THIS COMMON PUBLIC LICENSE (``AGREEMENT'').
ANY USE, REPRODUCTION OR DISTRIBUTION OF THE PROGRAM CONSTITUTES
RECIPIENT'S ACCEPTANCE OF THIS AGREEMENT.


\subsubsection*{1.  DEFINITIONS}

``Contribution'' means:

\begin{itemize}

\item[a)] in the case of the initial Contributor, the initial code and
documentation distributed under this Agreement, and

\item[b)] in the case of each subsequent Contributor:

\begin{itemize}

\item[i)]	 	changes to the Program, and


\item[ii)]		additions to the Program;


where such changes and/or additions to the Program originate from and are distributed by that particular Contributor.  A
Contribution 'originates' from a Contributor if it was added to the
Program by such Contributor itself or anyone acting on such
Contributor's behalf. Contributions do not
include additions to the Program which: (i) are separate modules of
software distributed in conjunction with the Program under their own
license agreement, and (ii) are not derivative works of the Program. 
\end{itemize}
\end{itemize}

``Contributor'' means any person or entity that distributes the Program.

``Licensed Patents '' mean patent claims licensable
by a Contributor which are necessarily infringed by the use or sale of
its Contribution alone or when combined with the Program. 





``Program'' means the Contributions distributed in accordance with this Agreement.





``Recipient'' means anyone who receives the Program under this Agreement, including all Contributors.


\subsubsection*{2.  GRANT OF RIGHTS}

\begin{itemize}
\item[a)]	Subject to the terms of this Agreement, each Contributor hereby grants Recipient a non-exclusive, worldwide, royalty-free copyright license to reproduce,
prepare derivative works of, publicly display, publicly perform,
distribute and sublicense the Contribution of such Contributor, if any,
and such derivative works, in source code and object code form.


\item[b)] 	Subject to the terms of this Agreement, each Contributor hereby grants Recipient a non-exclusive, worldwide, royalty-free
patent license under Licensed Patents to make, use, sell, offer to
sell, import and otherwise transfer the Contribution of such
Contributor, if any, in source code and object code form. This patent
license shall apply to the combination of the Contribution and the
Program if, at the time the Contribution is added by the Contributor,
such addition of the Contribution causes such combination to be covered
by the Licensed Patents. The patent license shall not apply to any
other combinations which include the Contribution. No hardware per se
is licensed hereunder. 


\item[c)] Recipient understands that although each
Contributor grants the licenses to its Contributions set forth herein,
no assurances are provided by any Contributor that the Program does not
infringe the patent or other intellectual property rights of any other
entity. Each Contributor disclaims any liability to Recipient for
claims brought by any other entity based on infringement of
intellectual property rights or otherwise. As a condition to exercising
the rights and licenses granted hereunder, each Recipient hereby
assumes sole responsibility to secure any other intellectual property
rights needed, if any. For example, if a third party patent license is
required to allow Recipient to distribute the Program, it is
Recipient's responsibility to acquire that license before distributing
the Program.


\item[d)] Each Contributor represents that to its knowledge
it has sufficient copyright rights in its Contribution, if any, to
grant the copyright license set forth in this Agreement. 

\end{itemize}


\subsubsection*{3.  REQUIREMENTS}


A Contributor may choose to distribute the Program in object code form under its own license agreement, provided that:

\begin{itemize}

\item[a)] it complies with the terms and conditions of this Agreement; and


\item[b)]	its license agreement:


\begin{itemize}
\item[i)]	effectively disclaims
on behalf of all Contributors all warranties and conditions, express
and implied, including warranties or conditions of title and
non-infringement, and implied warranties or conditions of
merchantability and fitness for a particular purpose;


\item[ii)] effectively excludes on
behalf of all Contributors all liability for damages, including direct,
indirect, special, incidental and consequential damages, such as lost
profits;


\item[iii)]
states that any provisions which differ from this Agreement are offered
by that Contributor alone and not by any other party; and


\item[iv)] states that source code for the Program is
available from such Contributor, and informs licensees how to obtain it
in a reasonable manner on or through a medium customarily used for
software exchange.

\end{itemize}
\end{itemize}


When the Program is made available in source code form:

\begin{itemize}
\item[a)]	it must be made available under this Agreement; and

\item[b)]	a copy of this Agreement must be included with each copy of the Program.
\end{itemize}


Contributors may not remove or alter any copyright notices contained within the Program.  

Each Contributor must identify itself as the
originator of its Contribution, if any, in a manner that reasonably
allows subsequent Recipients to identify the originator of the
Contribution. 

\subsubsection*{4.  COMMERCIAL DISTRIBUTION}

Commercial distributors of software may accept
certain responsibilities with respect to end users, business partners
and the like. While this license is intended to facilitate the
commercial use of the Program, the Contributor who includes the Program
in a commercial product offering should do so in a manner which does
not create potential liability for other Contributors. Therefore, if a
Contributor includes the Program in a commercial product offering, such
Contributor (``Commercial Contributor'') hereby agrees to defend and
indemnify every other Contributor (``Indemnified Contributor'') against
any losses, damages and costs (collectively ``Losses'') arising from
claims, lawsuits and other legal actions brought by a third party
against the Indemnified Contributor to the extent caused by the acts or
omissions of such Commercial Contributor in connection with its
distribution of the Program in a commercial product offering. The
obligations in this section do not apply to any claims or Losses
relating to any actual or alleged intellectual property infringement.
In order to qualify, an Indemnified Contributor must: a) promptly
notify the Commercial Contributor in writing of such claim, and b)
allow the Commercial Contributor to control, and cooperate with the
Commercial Contributor in, the defense and any related settlement
negotiations. The Indemnified Contributor may participate in any such
claim at its own expense.

For example, a Contributor might include the
Program in a commercial product offering, Product X. That Contributor
is then a Commercial Contributor. If that Commercial Contributor then
makes performance claims, or offers warranties related to Product X,
those performance claims and warranties are such Commercial
Contributor's responsibility alone. Under this section, the Commercial
Contributor would have to defend claims against the other Contributors
related to those performance claims and warranties, and if a court
requires any other Contributor to pay any damages as a result, the
Commercial Contributor must pay those damages.


\subsubsection*{5.  NO WARRANTY}

EXCEPT AS EXPRESSLY SET FORTH IN THIS AGREEMENT,
THE PROGRAM IS PROVIDED ON AN ``AS IS'' BASIS, WITHOUT WARRANTIES OR
CONDITIONS OF ANY KIND, EITHER EXPRESS OR IMPLIED INCLUDING, WITHOUT
LIMITATION, ANY WARRANTIES OR CONDITIONS OF TITLE, NON-INFRINGEMENT,
MERCHANTABILITY OR FITNESS FOR A PARTICULAR PURPOSE. Each Recipient is solely responsible for determining the appropriateness of using and distributing the Program and assumes all risks associated with its exercise of rights under this Agreement,
including but not limited to the risks and costs of program errors,
compliance with applicable laws, damage to or loss of data, programs or equipment, and unavailability or interruption of operations.  

\subsubsection*{6.  DISCLAIMER OF LIABILITY}


EXCEPT AS EXPRESSLY SET
FORTH IN THIS AGREEMENT, NEITHER RECIPIENT NOR ANY CONTRIBUTORS SHALL
HAVE ANY LIABILITY FOR ANY DIRECT, INDIRECT, INCIDENTAL, SPECIAL,
EXEMPLARY, OR CONSEQUENTIAL DAMAGES (INCLUDING WITHOUT LIMITATION LOST PROFITS),
HOWEVER CAUSED AND ON ANY THEORY OF LIABILITY, WHETHER IN CONTRACT,
STRICT LIABILITY, OR TORT (INCLUDING NEGLIGENCE OR OTHERWISE) ARISING
IN ANY WAY OUT OF THE USE OR DISTRIBUTION OF THE PROGRAM OR THE
EXERCISE OF ANY RIGHTS GRANTED HEREUNDER, EVEN IF ADVISED OF THE
POSSIBILITY OF SUCH DAMAGES.


\subsubsection*{7.  GENERAL}


If any provision of this
Agreement is invalid or unenforceable under applicable law, it shall
not affect the validity or enforceability of the remainder of the terms
of this Agreement, and without further action by the parties hereto,
such provision shall be reformed to the minimum extent necessary to
make such provision valid and enforceable.


If Recipient institutes patent litigation against
a Contributor with respect to a patent applicable to software
(including a cross-claim or counterclaim in a lawsuit), then any patent
licenses granted by that Contributor to such Recipient under this
Agreement shall terminate as of the date such litigation is filed. In
addition, if Recipient institutes patent litigation against any entity
(including a cross-claim or counterclaim in a lawsuit) alleging that
the Program itself (excluding combinations of the Program with other
software or hardware) infringes such Recipient's patent(s), then such
Recipient's rights granted under Section 2(b) shall terminate as of the
date such litigation is filed. 


All Recipient's rights under this Agreement shall
terminate if it fails to comply with any of the material terms or
conditions of this Agreement and does not cure such failure in a
reasonable period of time after becoming aware of such noncompliance.
If all Recipient's rights under this Agreement terminate, Recipient
agrees to cease use and distribution of the Program as soon as
reasonably practicable. However, Recipient's obligations under this
Agreement and any licenses granted by Recipient relating to the Program
shall continue and survive. 


Everyone
is permitted to copy and distribute copies of this Agreement, but in
order to avoid inconsistency the Agreement is copyrighted and may only
be modified in the following manner. The Agreement Steward reserves the
right to publish new versions (including revisions) of this Agreement from time to time.
No one other than the Agreement Steward has the right to modify this
Agreement. IBM is the initial Agreement Steward. IBM may assign the
responsibility to serve as the Agreement Steward to a suitable separate
entity. Each new version of the Agreement will
be given a distinguishing version number. The Program (including
Contributions) may always be distributed subject to the version of the
Agreement under which it was received. In addition, after a new version
of the Agreement is published, Contributor may elect to distribute the
Program (including its Contributions) under the new version.  Except
as expressly stated in Sections 2(a) and 2(b) above, Recipient receives
no rights or licenses to the intellectual property of any Contributor
under this Agreement, whether expressly, by implication, estoppel or
otherwise.    All rights in the Program not expressly granted under this Agreement are reserved.


 This Agreement is governed by the laws of the
State of New York and the intellectual property laws of the United
States of America. No party to this Agreement will bring a legal action
under this Agreement more than one year after the cause of action
arose. Each party waives its rights to a jury trial in any resulting
litigation.
