This section provides an overview of Jikes\TMweb{} RVM as well as
information about how best to use this document.

\subsection{Welcome to Jikes RVM}

Jikes\TMboth{} RVM is a Research Virtual Machine 
 developed at the 
\xlink{IBM}{\IBMURL}\Rboth{}
\xlink{T.J.\ Watson Research Center}{\WatsonURL}.  Key
features of the system include
\begin{itemize}
\item the entire virtual machine (VM) is implemented in the
  Java\TMboth{} programming language,
\item the VM utilizes two compilers and no interpreter,\footnote{The
Quick Compiler is an experimental third compiler}
\item a family of parallel, type-exact garbage collectors,
\item a lightweight thread package with compiler-supported preemption,
\item an aggressive optimizing compiler, and 
\item a flexible online adaptive compilation infrastructure.
\end{itemize}

A significant body of information about Jikes RVM 
(formerly known as 
\xlink{\jp}{\JalapenoHomeURL}) appears 
in our published
papers.  For overviews of the system's structure, including the runtime system,
optimizing compiler, and adaptive systems, \xlink{see the list of published papers}[, available from the \jrvm{} web page:
\begin{quote}
\texttt{\RVMPubsURL}
\end{quote}
]{\RVMPubsURL}.

The best paper for a general introduction to \jrvm{} is the 
\xlink{IBM Systems Journal, January 2000
paper
\T~\cite{jalapeno-ibmsj-00}
}{\SystemJournalPaperURL}.  
For a historical overview of the project, see the
\xlink{IBM Systems Journal, April 2005 paper
\T~\cite{}
}{\SystemsJournalPaperHistoryURL}.
For an introduction to the memory managers (MMTk), see the 
\xlink{2004 ICSE\begin{iftex}~\cite{mmtk-icse-04}\end{iftex}}
{\MMTKICSEPaperURL} and the
\xlink{2004 SIGMETRICS\begin{iftex}~\cite{mmtk-sigmetrics-04}\end{iftex}}
{\MMTKSIGMETRICSPaperURL} papers.
For an overview of the adaptive system, see the
\xlink{2000 OOPSLA\begin{iftex}~\cite{jalapeno-adaptive-00}\end{iftex}}
{\OOPSLAPaperURL}  and IBM Technical Report
\xlink{IBM TR
Nov'04\begin{iftex}~\cite{adaptive-tr-04}\end{iftex}}
{\IBMTRURL} 
papers, respectively.

We have given several tutorials on Jikes RVM that you may find
useful. The PACT'01 tutorial covers all of Jikes RVM.\@  The PLDI'02 and
OOPSLA'02 tutorials focus on the optimizing compiler.  The tutorial
slides are available at
\begin{quote}
\xlink{{\tt \RVMSlidesURL}}{\RVMSlidesURL}
\end{quote}

Jikes RVM is a bleeding-edge research project.  You will find that
some of the code does not live up to product quality standards.  Don't
hesitate to help rectify this by contributing clean-ups, bug fixes,
and missing documentation to the project.

Many academic groups have adopted Jikes RVM as their primary research
infrastructure, resulting in
\xlink{publications}{\RVMUsersPubsURL}
in leading
conferences, such as PLDI, POPL, OOPSLA, and SIGMETRICS.\@  In the first year
of its open source release, the VM was downloaded by over
1500 unique IP addresses, including over ninety
universities around the world. A list
of some of the \xlink{users}{\RVMUsersURL} is available.
\xlink{Teaching resources}{\RVMTeachingResourcesURL} using
Jikes RVM are also available.

\subsection{About this document}

This document provides Jikes\TMweb{} RVM information that is not
covered in our published papers.  For high-level overviews,
algorithms, and structures, you will find the published papers to be
the best starting place. This document supplements the Jikes RVM
papers, focusing on implementation details of how to build, run,
and add functionality to the system.

This document is available as both PostScript\Rboth{} and HTML.\@  \xlink{You will find the
HTML version more useful, as it includes hyperlinks}[.  The HTML version is
available at:
\begin{example}
\RVMUserGuideURL
\end{example}
].

Each released Jikes RVM tarball holds a PostScript version of this
document in its top directory, as ``{\tt userguide.ps}''.


The \jrvm{} web page includes Javadoc\TMboth{}
\xlink{API documentation}{\RVMJavadocURL}. 
This HTML should be the primary reference for individual classes. The
level of detail provided in the Javadoc is highly variable, but most
Jikes RVM classes have at least a minimal description.

You may find sections of this user's guide missing, incomplete or
otherwise confusing. We intend this document to live as a continual
work-in-progress, hopefully growing and maturing as community members
edit and add to the guide.  Please accept this invitation to
contribute.

Please send feedback, bug fixes, and text contributions to the 
\xlink{{\tt JikesRVM-researchers} mailing list}{\RVMResearcherMailingListURL}.  
Constructive criticism will be cheerfully accepted. 

