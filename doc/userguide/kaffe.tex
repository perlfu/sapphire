\newcommand{\rvmbuild}[1]{\$RVM\_BUILD\texttt{/#1}}%
\newcommand{\gft}{\texttt{GenerateFromTemplate}}%
\newcommand{\gid}{\texttt{GenerateInterfaceDeclarations}}%
\newcommand{\biw}{\texttt{BootImageWriter}}%
%
%
Starting with Jikes RVM version 2.3.2 you can use the Kaffe open-source (GPL)
virtual machine to boot Jikes RVM under Linux, as an alternative to
the Sun-derived JDK.  This eliminates the last piece of non-free
software that used to be required to build Jikes RVM.  As of
March 27, 2004, you can use a combination of Kaffe and a Jikes RVM
\texttt{BaseBase}\textit{*} build to create Jikes RVM images that
include the optimizing compiler.

You can \xlink{download Kaffe}[ from {\tt \KaffeURL}]{\KaffeURL} or,
under SuSE Linux, install the {\tt kaffe} RPM.  

\subsection{Caveats}

If you want to build under Kaffe, please remember that building Jikes
RVM with Kaffe as the host VM is not nearly as well-tested as using
the Sun VM.  (You can help to change this!)

There are some caveats here:

\begin{itemize}

\item We've run all of our tests against Kaffe 1.1.4 and against the
post-1.1.4 CVS head.

%% \item So far, we have only successfully built {\tt BaseBase}{\it *}
%% (the {\tt prototype} configuration) images doing a pure Kaffe build.
%% If you want to build an image that will run Jikes RVM's optimizing JIT
%% compiler, you will have to go through a two-stage boot process.


\item We haven't run the regression test suite through on Kaffe yet.
  Some aspects of it use features of the Sun JDK, and will probably
  need to be ported to use entirely free tools.

\end{itemize}

To make a baseline-compiled build, use one of the {\tt
\$RVM\_\-ROOT/rvm/config/i686-pc-linux-gnu.kaffe{\it *}} files as your
starting point.  Now go ahead and \link{follow the regular
installation directions}[.  See Section~\Ref,
page~\Pageref]{section:installation}.

We would very much like help with addressing any of the problems
listed in the caveats above.  We would also very much appreciate
patches for building Jikes RVM under other free Java systems.  We are
currently working on being able to use Jikes RVM itself to write the
boot image.  This does not yet work, because of namespace issues.

\subsection{Build a Jikes RVM Image with the Optimizer's Adaptive System}

\subsubsection{Template Expansion --- why there is a bug workaround}

  There is a bug in how Kaffe 1.1.4 runs the template expander (\gft)
  when building the \jrvm{} optimizing compiler.  (We are working with
  the Kaffe developers to track down that bug.)

  The bug is discussed in 
  \xlink{\htmlonly{this}\texonly{a March, 2004} 
         message to the \texttt{jikesrvm-core} mailing list}{%
    http://www-124.ibm.com/pipermail/jikesrvm-core/2004-March/000704.html}.


Because of this bug, if you want to build an image that uses the
optimizing compiler (and in order to get good performance from Jikes
RVM, you'll have to do precisely that), you will need to use a
two-phase build process.

\subsubsection{Instructions}

First, build a \texttt{BaseBase}\textit{*} image as discussed above.
You'll probably want to use the \texttt{prototype} configuration,
which is always an alias for the most stable baseline-compiled
configuration.  

This is the end of the first phase.

Rename the first phase's \${RVM\_BUILD} to something like
\$\{RVM\_ROOT\}\texttt{/DonorRVMBuild}.    

Now create a new configuration file for the second phase build, this
time following the example in {\tt
\$RVM\_\-ROOT/rvm/config/exp/i686-pc-linux-gnu.kaffe-with-help.augart}.    
Set DONOR\_RVM\_BUILD in your new configuration file to point to
\$\{RVM\_ROOT\}\texttt{/DonorRVMBuild}.    

%% In your new config file, set {DONOR\_RVM\_ROOT} to be the old RVM\_ROOT
%% (this will usually be the same as the current RVM\_ROOT), and set
%% DONOR\_RVM\_BUILD to be the RVM\_BUILD for the \texttt{prototype}
%% image you just made.

The \jrvm{} you just built will be the Host VM when you're running
\gft{}.  This is the second stage of the boot process.  Because the
Host \jrvm{} is just compiled with the baseline compiler, and because
you're expanding a lot more information, it will take much longer to
run \texttt{jbuild.expand} during this second-phase build --- on this
author's current computer (IBM Thinkpad T23, Pentium III/1133, 1 GB
RAM), about 120 seconds, versus less than three seconds during the first build.

%% If you plan to hack on Jikes RVM, you will probably be making a lot of
%% builds.  You can use the newly made optimizing build as the new Donor VM.
%% Rename the second stage's \${RVM\_BUILD} to something like
%% \$\{RVM\_ROOT\}\texttt{/DonorRVMBuild}, and set 
%% DONOR\_RVM\_BUILD in your configuration file to point to
%% \$\{RVM\_ROOT\}\texttt{/DonorRVMBuild}.  

Set RVM\_HOST\_CONFIG and RVM\_TARGET\_CONFIG to the new configuration
file, run \texttt{jconfigure}, and run \texttt{jbuild}, just as
before.   On the Thinkpad T23 configuration, this took 8433 seconds (two
hours and 20 minutes) to build a Jikes RVM \texttt{development}
(\texttt{FullAdaptiveGenMS}) image; this compares with 25 minutes to
build such an image under the Sun JDK.

\paragraph{Outstanding Bug} Do not use a {\tt prototype-opt} build or
a \texttt{BaseAdaptiveCopyMS} build as
your DONOR\_RVM\_BUILD.  On Saturday, March 27, 2004, a bug
was reported on the \texttt{jikesrvm-core} list that a
\texttt{prototype-opt} build (an alias for a
\texttt{BaseAdaptiveGenMS} build) crashed with a
\texttt{NullPointerException} in the garbage collector while expanding
one of the templates.  A \texttt{BaseAdaptiveCopyMS} build also
crashed.  This author recommends using a straight \texttt{prototype} build
until the bug is resolved.  Luckily, you don't have to expand the
templates very often.


\subsection{What does the Donor VM (Host VM) do?}%
\index{Donor VM}%
\index{Host VM}%


You must remember that Jikes RVM is written almost entirely in Java.
To build Jikes RVM, we need to be able to run the Jikes RVM compiler
on Jikes RVM's class files.  And in order to run that compiler, we
need to have an already running Host VM.

We use a running Host VM at three stages in the Jikes RVM
building process:

\begin{enumerate}

%
\item \rvmbuild{\IndexTexttt{jbuild.expand}} runs
	\gft\IndexTexttt{GenerateFromTemplate}, a Java program.
This is a macro processor\index{macro processor}
with recursion, looping,
and so on.   In an x86 \jrvm{} build that includes the adaptive system and
optimizing compiler, \gft{} produces 115,900 lines of Java code, over
4 MB worth.    In an x86 \jrvm{} build that just includes the baseline
compiler, \gft{} still emits 16,416 lines of code, over 600 KB
worth.    \gft's output is in \rvmbuild{\IndexTexttt{RVM.generatedSources}}.

%% Kaffe 1.1.4 succeeds in this phase for a baseline compiled build, but
%% is slightly off for an optimizing build.  That is why, to build Jikes
%% RVM entirely with free software tools, you will need to run a
%% two-stage build the first time.  Once you have a built instance of
%% Jikes RVM, you can use that as your host VM for this phase in future
%% builds.

\item \rvmbuild{\IndexTexttt{jbuild.interfaceDeclarations}} 
launches the Java program \gid\IndexTexttt{GenerateInterfaceDeclarations}, whose product is
``\texttt{RVM.scratch/InterfaceDeclarations.h}''.
\texttt{InterfaceDeclarations.h} is a C++ header file containing the layout of
the Jikes RVM boot record.  It also contains additional declarations
that C and C++ programs may need.

Jikes RVM cannot run \gid{} at this time.  That is because \gid{}
needs to know how Jikes RVM will lay out its boot record, and the only
way to do that is for the host to load Jikes RVM's classes and run it
in a sort of zombie state, just enough to get it to lay out the
VM\_BootRecord class in memory and report back on the offsets of the
various fields in that class.  We enter that zombie state by calling
\texttt{VM.initForTool()}, which exists only for this one use. 
 If you run \gid{} under \jrvm{},
then \texttt{VM.initForTool()} reinitializes the running VM.  The
runtime system immediately forgets that it already loaded a currently
executing program, and quickly aborts with a VM internal error.


\item \rvmbuild{\IndexTexttt{jbuild.linkBooter}} runs the Java program
\IndexTexttt{BootImageWriter}.  This program loads Jikes RVM's classes
and, like \gid, half-animates the just-loaded \jrvm{} into a zombie
state --- this time, via \IndexTexttt{VM.initForBootImageWriter()}.  Once
again, a running instance of \jrvm{} does not like to become into a
zombie.

\end{enumerate}

%% \subsection{Other free VMs}

%% Right now, the other free VMs we've tried to work with as the donor VM
%% (host VM) are:

%% \begin{itemize}

%% \end{itemize}
