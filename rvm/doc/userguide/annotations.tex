The Jikes\trademark RVM optionally supports reading and using
Soot-style class file 
annotations.  Such annotations can specify that a null check or bounds
check is redundant for a particular byte code instruction.  These
annotations are produced by the {\bf Soot} class file optimizer
available from
\xlink{\SOOTURL}{\SOOTURL}.  

When the annotations command-line option is true, class file
annotations are processed by \xlink{{\tt
VM\_Method.java}}{\VMMethodURL} and stored internally in Jikes RVM.
During compilation (either baseline or optimizing), the compiler 
queries the {\tt VM\_Method} class for a particular bytecode to
determine if the generation of a null or bounds check can be
suppressed. 

Processing Soot-style annotations is not enabled, by default.  To
enable this support specify {\tt ``annotations=true''} to the
appropriate compiler.  For example, use {\tt -X:irc:annotations=true}
for non-adaptive images and {\tt -X:aos:irc:annotations=true} for
adaptive images.

\JikesTMFooter