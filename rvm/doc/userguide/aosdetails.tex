A comprehensive discussion of the design and implementation of the
adaptive optimization system is given in the
\xlink{OOPSLA 2000 paper by Arnold, Fink, Grove, Hind and
Sweeney}{\OOPSLAPaperURL}.  
Since the writing of the OOPSLA paper the following major changes
have been made to the adaptive system:
\begin{itemize}
\item To improve the accuracy of the sampling mechanism, we insert
yield points at method epilogues, as well method entries and loop
backedges. 

\item The method samples are no longer decayed by the decay
organizer.  However, the call edge samples are still decayed.

\end{itemize}

The remainder of this section provides a road map to the key
classes that comprise the adaptive optimization system.
Most of the code for the adaptive system lives under
{\tt \$RVM\_ROOT/rvm/src/vm/adaptive}.   One exception is the class
\xlink{{\tt VM\_RuntimeCompiler}}{\VMRuntimeCompilerURL}, which lives in
{\tt \$RVM\_ROOT/rvm/src/vm/compilers/harness/runtime/adaptive}.  
This file contains the {\tt compile} method that is called when a
method needs to be {\em initially\/} compiled.  

The files under {\tt \$RVM\_ROOT/rvm/src/vm/adaptive} are organized
into the following five directories:
\begin{description}
\item [controller:] classes related to the controller thread, which is
the decision-making component in the adaptive system

\item [database:] classes that contain information related to decisions
made by the adaptive system, as well as cumulative profile data

\item [recompilation:]  classes related to the recompilation thread

\item [runtimeMeasurements:] classes related to gathering online
profile information

\item [utility:]  miscellaneous adaptive system classes, such as
options, logging facilities, and data structures.
\end{description}
Further details are provided in the following subsections.

\subsection{Controller Directory}
This directory contains classes related to 
the decision-making component of the adaptive system.
Two key classes are 
\xlink{{\tt VM\_Controller}}{\VMControllerURL} and 
\xlink{{\tt VM\_ControllerThread}}{\VMControllerThreadURL}.  
The former contains static data that
is relevant to many components of the adaptive system.  Some data
items are references to controller, organizer, and
(re-)compilation threads, the queues which these thread use to 
communicate, and command-line options.  The boot method of this class
boots the adaptive system and creates the \xlink{{\tt
VM\_ControllerThread}}{\VMControllerThreadURL} 
object. 

The \xlink{{\tt VM\_ControllerThread}}{\VMControllerThreadURL} creates
the other adaptive system 
threads, related to profiling and recompilation.  It then remains in a
loop looking for events placed on the {\tt controllerInputQueue} by
the one or more organizers.  The processing of these events is
performed by other classes in this directory.

\subsection{Database Directory}
This directory contains classes that store information related to the
adaptive system.  We use the term ``database'' to mean an online
repository, not a full-fledged database. This directory contains two
subdirectories: 
{\tt callGraph} and {\tt methodSamples}, which contain information
about a partial call graph (used by adaptive inlining) and a history
of methods that are sampled.  Both classes are populated with data based
on the yield-point sampling mechanism.

\subsection{Recompilation Directory}
This directory contains the \xlink{{\tt
VM\_CompilationThread}}{\VMCompilationThreadURL}, which 
processes
events placed on {\tt compilationQueue} by the {\tt
VM\_ControllerThread}, and invokes the optimizing compiler using the
information passed in the event.  Another class in this directory is
\xlink{{\tt VM\_CompilerDNA}}{\VMCompilerDNAURL}, which contains the cost/benefit
parameters that drive the \xlink{{\tt
VM\_ControllerThread}}{\VMControllerThreadURL}'s recompilation 
decisions. This directory also contains a subdirectory called {\tt
instrumentation}, which contains classes related to inserting
instrumentation into an optimized method.  This enables the gathering
of more detailed profile information, in addition to the
yieldpoint-based profile.

\subsection{RuntimeMeasurements Directory}
In addition to two interfaces and a static class ({\tt
VM\_RuntimeMeasurements}),
this directory contains three subdirectories:
\begin{description}
\item [instrumentation:] miscellaneous classes related to
data collected from the instrumentation of methods.  

\item [listeners:] classes that contain methods that 
are called when application samples,
based on yield points, are taken.  These samples occur when a Java
thread calls into the Jikes RVM scheduler to perform a context switch
to another Java thread.  The sample is taken before the thread switch occurs.
See the 
\xlink{{\tt threadSwitch}}{\VMThreadthreadSwitchURL} method in
{\tt \$RVM\_ROOT/rvm/src/vm/scheduler/VM\_Thread} for details.
There may be many active listeners at one
time.  For example, we may take regular method samples to determine
recompilation candidates (\xlink{{\tt
VM\_MethodListener}}{\VMMethodListenerURL}) and call edge 
samples to aid in inlining decisions (\xlink{{\tt
VM\_EdgeListener}}{\VMEdgeListenerURL}).  

\item [organizers:]  classes related to separate Java threads that
process samples or other raw profile data.  There may be many
organizers active at the same time.
For example, one organizer may process method samples 
(\xlink{{\tt
VM\_MethodSampleOrganizer}}{\VMMethodSampleOrganizerURL}) while
another is processing 
call edge samples (\xlink{{\tt
VM\_AIByEdgeOrganizer}}{\VMAIByEdgeOrganizerURL}). 
\end{description}

\JikesTMFooter

\subsection{Utility Directory}
The adaptive system contains a logging mechanism that allows it to
record actions it takes to a log file.  The file \xlink{{\tt
VM\_AOSLogging}}{\VMAOSLoggingURL} 
contains a series of methods that provide the ability to log
decisions.  The level of logging is controllable on the command line.

Option processing for the adaptive system is also found in this
directory.  It is similar to the option processing for the optimizing
compiler.



