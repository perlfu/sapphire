
Currently, the RVM has two types of configurations:
the adaptive configurations contain the adaptive optimization system (AOS),
and the non-adaptive configurations that do not.  
This section describes the nonstandard RVM command-line directives that 
provide the mechanism to specify adaptive optimization system and optimizing
compiler options.
Section~\ref{subsection:nonadaptive:cmdline} describes how command-line 
directives are specified in a non-adaptive configuration, and 
Section~\ref{subsection:adaptive:cmdline} describes how command-line 
directives are specified in a adaptive configuration.

%%%%%%%%%%%%%%%%%%%%%%%%%%%%%%%%
\subsection{Command-Line Directives in Non-adaptive Configurations}
\label{subsection:nonadaptive:cmdline}

This section describes how nonstandard RVM command line options are 
specified for a non-adaptive configuration of RVM.
In a non-adaptive configuration, the command line options modifies
the behavior of the initial runtime compile when this compiler
is the optimizing compiler.

To pass an option to the optimizing compiler, the option is prefixed
with {\tt -X:irc:}.
For example, to perform global array bounds check elimination on demand
when a method is initially compiled with the optimizing compiler,
use the {\tt -X:irc:global\_bounds=true} directive.

Please note that a particular non-adaptive configuration (such as
OptBasecopyingGC) may not use the 
optimizing compiler as the initial runtime compiler, but rather use the 
baseline compiler.
Therefore, the validity of the prefix to access the optimizing compiler
as the initial runtime compiler depends on how the configuration is built.

All of the above nonstandard VM directives must occur before 
the application class name and application's command-line options.

For the following discussion, we assume that the appropriate prefix has been
prepended to the option and only discuss the option.

To see descriptions of command-line arguments to the optimizing system, use
the {\tt help} option; that is, {\tt -X:irc:help}.  
As of this writing, this command produces the following output:

\include{opt_options}

The {\tt printOptions} command-line optimizing compiler option will print
the current setting of optimizing compiler's options.  Please note that the 
order of the {\tt printOptions} command-line directive with respect to other
optimizing compiler command-line directives is important.  
When a {\tt printOptions} directive is found, the setting of the optimizing 
compiler options will reflect only those optimizing compiler directives
that have preceded the {\tt printOptions} directive.  

%%%%%%%%%%%%%%%%%%%%%%%%%%%%%%%%%%%%%%%%%%%%%%%%%%%%%%%%%%%%%%%%%%%%%
\subsection{Command-Line Directives in Adaptive Configurations}
\label{subsection:adaptive:cmdline}

This section describes how nonstandard RVM command line options are 
specified for an adaptive configuration.
In an adaptive configuration, the command line options modifies
the behavior of either the adaptive optimization system or 
the optimizing compiler.
A command line directive is constructed by concatenating an option 
with a prefix which identifies the desired destiny for that option.

All options in an adaptive configuration are prefixed with {\tt -X:aos}.
To pass an option to the adaptive optimization system, use the {\tt -X:aos:} 
prefix. 
For example, to set the logging level of AOS to one, use the directive
{\tt -X:aos:logging\_level=1}.
Unlike a nonadaptive configuration, an adaptive configuration may 
conceptually have many optimizing compilers that are available at runtime,
each with its own set of option values.
We present a mechanism to address each conceptual optimizing compiler.
To pass options to the opt compiler that recompiles a methods
use the {\tt -X:aos:opt[?]} prefix where the {\tt ?} is optional and 
if specified is an integer that identifies the optimization level.
For example, {\tt -X:aos:opt2:global\_bounds=true} performs 
global array bounds check elimination on demand when a method is 
optimized at optimization level 2.
If no optimization level is specified, the option applies to all 
optimization levels of the optimizing compiler that recompiles methods.
For example, {\tt -X:aos:opt:global\_bounds=true} performs 
global array bounds check elimination on demand whenever a method is 
recompiled with optimization.
Like a nonadaptive configuration, the optimizing compiler may be used
as the initial runtime compiler in an adaptive configuration.
If so, options are passed to the initial runtime compile by prefixing
each option with {\tt -X:aos:irc:}.
For example, to perform global array bounds check elimination on demand
when a method is initially compiled with the optimizing compiler,
use the {\tt -X:aos:irc:global\_bounds=true} directive.
See Section~\ref{subsection:nonadaptive:cmdline} for a discussion of the
optimizing compiler command-line options that are available.

Please note that a particular adaptive configuration may not use the 
optimizing compiler as the initial runtime compiler, but rather use the 
baseline compiler.
Therefore, the validity of the prefix to access the optimizing compiler
as the initial runtime compiler depends on the configuration.

Finally, the prefix {\tt -X:aos:share[?]:o=v} is a short hand for passing a
option value pair, {\tt o=v}, to both the AOS and to the different "conceptual"
optimizing compilers.  If the optimization level is specified, then only that 
optimization level for recompilation is affected.  Otherwise, the option is set
for the initial runtime optimizing compiler and all optimization levels for
method recompilation.

All of the above nonstandard VM directives must occur before 
the application class name and application's command-line options.

For the following discussion, we assume that the appropriate prefix has been
prepended to the option and only discuss the option.

%%%%%%%%%%%%%%%%%%%%%%%%%%%%%%%%%%%%%%%%%%%%%%%%%%
\subsubsection{Adaptive Optimization System (AOS) Command-Line Directives}

To see a description of the command-line directives to the
AOS, use {\tt -X:aos:help}.
As of this writing, this command produces the following output:

\include{adaptive_options}

The {\tt primary\_strategy} option determines what strategy is used to compile
methods.
The default strategy is {\tt adaptive} which allows a method to be
recompiled multiple times at different optimization levels.
The other strategies allow an adaptive configuration to behave as a
just-in-time compiler (JIT) by determining what compiler will compile a method.
For example, the {\tt optonly} strategy compiles a method once with the 
optimizing compiler.  To optimize compile a method at optimization level 1,
use the {\tt -X:aos:irc:O1} option.
(Note that to obtain the functionality of a JIT at optimization level 1, 
{\tt -X:primary\_strategy=optonly -X:aos:irc:O1}, in a nonadaptive configuration
would be achieved with the command-line option {\tt -X:irc:O1}.)
Currently, the {\tt quickonly} primary strategy is not supported.


Another interesting combination of command-line option that allow single level
adaptive recompilation is 
{\tt -X:aos:adaptive\_recompilation=false -X:aos:default\_opt\_level=1} 
which will recompile hot methods at optimization level 1 only.

%%%%%%%%%%%%%%%%%%%%%%%%%%%%%%%%%%%%%%%%%%%%%%%%%%%%%%%
\subsubsection{Implementing AOS command-line options}

This section provides some information on various
implementation details for AOS command-line options.

\index{AOS command-line options}
\index{VM\_AOSOptions class}
The command-line options to AOS are
stored as fields in an object of type {\tt VM\_AOSOptions}.
The RVM build process generates the {\tt VM\_AOSOptions.java} 
file automatically from a template.  

\index{BooleanOptions.dat}
\index{ValueOptions.dat}
To add or modify the command-line options in {\tt VM\_AOSOptions.java},
you must modify either {\tt BooleanOptions.dat}, {\tt ValueOptions.dat},
{\tt ShareBooleanOptions.dat}, or {\tt ShareValueOptions.dat}.
You should describe your desired command-line option in a format 
described below in Section~\ref{section:optdetails}.
The options in the {\tt ShareBooleanOptions.dat} and {\tt ShareValueOptions.dat}
files are defined as both AOS and optimizing compiler options.

%%%%%%%%%%%%%%%%%%%%%%%%%%%%%%%%%%%%%%%%%%%%%%%%%%%%%%%%%%%%%
\subsection{Adding RVM Nonstandard Command-Line Options}

This section states how nonstandard RVM command-line options can be added
to RVM.  
Nonstandard RVM command-line options are those options that are specific to RVM.
The format of a nonstandard RVM option is {\tt -X:o=v} where {\tt o} is the 
option and {\tt v} is the value it is to be set to.
Adherence to this format is important to keep command-line options processing
from becoming unwieldy.

RVM command-line options are processed in two places: {\tt RunBootImage.C} and
{\tt VM\_CommandLineArgs.java}.  
{\tt RunBootImage.C} is called first before the RVM boot image is loaded,
and {\tt VM\_CommandLineArgs.java} is called after.
In addition to processing any option which does not require RVM boot image
to be loaded (such as {\tt help} and {\tt version}), 
{\tt RunBootImage.C} processes any nonstandard option that impacts either 
heap size, message output, or where to find the boot image.
To allow unrestricted RVM option orders and 
since command line processing stops at the first option that is not 
recognized as a RVM option, all other options must be recognized by
{\tt RunBootImage.C} and passed on.
