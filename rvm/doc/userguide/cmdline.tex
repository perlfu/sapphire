Currently, Jikes\trademark RVM has two types of configurations:
the adaptive configurations that contain the adaptive optimization
system (AOS), and the non-adaptive configurations that do not.  This
section describes the non-standard Jikes RVM command-line options that
provide the mechanism to specify AOS, baseline compiler, and
optimizing compiler options for the different types of configurations;
and how to add a new non-standard command-line option.

\subsection{Command-Line Options in Non-Adaptive Configurations}
\label{subsection:nonadaptive:cmdline}

This section describes how non-standard Jikes\trademark RVM command
line options are specified for a non-adaptive configuration of Jikes
RVM.  In a non-adaptive configuration, the command line options modify
the behavior of the baseline or optimizing compiler.  Different
prefixes can be used to direct the option to one of following:
the initial runtime compiler ({\tt -X:irc:}),
the baseline compiler ({\tt -X:base:}) 
and the optimizing compiler ({\tt -X:opt:}). 
Some prefixes are only allowed with some configurations.
Depending on the configuration selected, the initial runtime compiler is 
either the baseline or the optimizing compiler. 

Consider the case where the initial runtime compiler is the optimizing
compiler. In that case either of the prefixes {\tt -X:irc:} or {\tt
-X:opt:} can be used to pass options to the optimizing compiler.  For
example, to perform global array bounds check elimination on demand
when a method is initially compiled with the optimizing compiler, use
the {\tt -X:irc:global\_bounds=true} directive.  In such a
configuration the baseline compiler will still be present and will be
used for some compilations.  The prefix {\tt -X:base:} must be used to
pass an option to the baseline compiler in configurations when the
initial runtime compiler is the optimizing compiler.

When the initial runtime compiler is the baseline compiler, the
optimizing compiler will not be part of the system, so the prefix {\tt
-X:opt:} should not be used. Also, with such a configuration, options
specified with the prefix {\tt -X:irc:} must be options recognized by
the baseline compiler.

All of the above non-standard VM options must occur before the
application class name and application's command-line options.

For the following discussion, we assume that the appropriate prefix
has been prepended to the option and only discuss the option.

%%%%%%%%%%%%%%%%%%%%%%%%%%%%%%%%%%%%%%%%%%%%%%%%%%%%%%%%
\subsubsection{Optimizing Compiler Command-Line Options}
\label{section:nonadaptive:optimizing:options}

To see descriptions of command-line options to the optimizing compiler,
use the {\tt help} option with the {\tt -X:opt:} prefix 
to generate the following output:

\T \begin{tiny}
\input{opt_options}
\T \end{tiny}

Note that when the initial runtime compiler is the optimizing compiler, 
the {\tt -X:irc} and the {\tt -X:opt} command line prefixes are equivalent.

%%%%%%%%%%%%%%%%%%%%%%%%%%%%%%%%%%%%%%%%%%%%%%%%%%%%%%
\subsubsection{Baseline Compiler Command-Line Options}
\label{section:nonadaptive:baseline:options}

To see descriptions of the command-line options to the baseline
compiler, use the {\tt help} option with the prefix {\tt -X:base:} 
to generate the following output:

\T \begin{small}
% From VM_BASEOptions.template
\begin{verbatim}
-X:base[:help]			Print brief description of baseline compiler's command-line arguments
-X:base:printOptions		Print the current values of the active baseline compiler options

Boolean Options (-X:base:<option>=true or -X:base:<option>=false)
option                               Description
annotations                            Act on annotations in class files
preload_as_boot                        Apply boot options to preload_class
verbose                                Print method name at start of compilation
mc                                     Print final machine code

Value Options (-X:base<option>=<value>)
option                         Type    Description
preload_class                  String  Class to preload upon 1st OPT compilation

Selection Options (set option to one of an enumeration of possible values)

Set Options (option is a set of values)
method_to_print                Only apply print options against methods whose name contains this string
method_to_break                Invoke breakStub for jdp's benefit after compiling methods whose name contains this string
\end{verbatim}


\T \end{small}
 
Note that when the initial runtime compiler is the baseline compiler, 
the {\tt -X:irc} and the {\tt -X:base} command line prefixes are equivalent.

%%%%%%%%%%%%%%%%%%%%%%%%%%
\subsubsection{Discussion}

The {\tt printOptions} command-line option for both the baseline and 
optimizing compilers will print the current setting of compiler's options.  
Please note that the order of the {\tt printOptions} command-line directive 
with respect to other compiler command-line options is important.  
When a {\tt printOptions} directive is found, the setting of the  
compiler options will reflect only those compiler options
that have preceded the {\tt printOptions} directive.  

%%%%%%%%%%%%%%%%%%%%%%%%%%%%%%%%%%%%%%%%%%%%%%%%%%%%%%%%%%%%
\subsection{Command-Line Options in Adaptive Configurations}
\label{subsection:adaptive:cmdline}

This section describes how non-standard Jikes\trademark RVM command
line options are specified for an adaptive configuration.  In an
adaptive configuration, the command line options modifies the behavior
of the adaptive optimization system, the optimizing compiler or the
baseline compiler.  A command-line directive is constructed by
concatenating an option with a prefix which identifies the desired
destiny for that option.

All options in an adaptive configuration are prefixed with {\tt
-X:aos}.  To pass an option to the adaptive optimization system, use
the {\tt -X:aos:} prefix.  For example, to set the logging level of
AOS to one, use the directive {\tt -X:aos:logging\_level=1}.  Unlike a
nonadaptive configuration, an adaptive configuration may conceptually
have many optimizing compilers that are available at runtime, each
with its own set of option values.  We present a mechanism to address
each conceptual optimizing compiler.  To pass options to the opt
compiler that recompiles a method use the {\tt -X:aos:opt[?]} prefix
where the {\tt ?} is optional and if specified is an integer that
identifies the optimization level.  For example, 
{\tt -X:aos:opt2:global\_bounds=true} performs global array bounds check
elimination on demand when a method is optimized at optimization level
2.  If no optimization level is specified, the option applies to all
optimization levels of the optimizing compiler that recompiles
methods.  For example, {\tt -X:aos:opt:global\_bounds=true} performs
global array bounds check elimination on demand whenever a method is
recompiled with optimization.  Like a nonadaptive configuration, there
is an initial runtime compiler.  In the default adaptive
configurations, the initial runtime compiler is the baseline compiler.
Options are passed to the initial runtime compiler by prefixing each
option with {\tt -X:aos:irc:}.  For example, to perform global array
bounds check elimination on demand when a method is initially compiled
with the optimizing compiler, use the {\tt
-X:aos:irc:global\_bounds=true} directive.  See
Section~\ref{subsection:nonadaptive:cmdline} for a discussion of the
optimizing and baseline compiler command-line options that are
available.

Finally, the prefix {\tt -X:aos:share[?]:o=v} is a short hand for
passing a option value pair, {\tt o=v}, to both the AOS and to the
different "conceptual" optimizing compilers.  If the optimization
level is specified, then only that optimization level for
recompilation is affected.  Otherwise, the option is set for AOS and
all optimization levels for method recompilation.

All of the above non-standard VM options must occur before 
the application class name and application's command-line options.

For the following discussion, we assume that the appropriate prefix has been
prepended to the option and only discuss the option.

%%%%%%%%%%%%%%%%%%%%%%%%%%%%%%%%%%%%%%%%%%%%%%%%%%%%%%%%%%%%%%%%%%%%%%%
\subsubsection{Adaptive Optimization System (AOS) Command-Line Options}

To see a description of the command-line options to the AOS, use 
{\tt -X:aos:help}.  As of this writing, this command produces the
following output:

\T \begin{tiny}
\input{adaptive_options}
\T \end{tiny}

The {\tt primary\_strategy} option determines what strategy is used to
compile methods.  The default strategy is {\tt adaptive} which allows
a method to be recompiled multiple times at different optimization
levels.  The other strategies allow an adaptive configuration to
behave as a just-in-time compiler (JIT) by determining what compiler
will compile a method.  For example, the {\tt optonly} strategy
compiles a method once with the optimizing compiler. This has the
effect of making the initial runtime compiler be the optimizing
compiler (and no recompilations will take place), so subsequent
options in the command line that are prefixed with {\tt -X:aos:irc:}
will be passed to the optimizing compiler. Since no recompilation will
take place options prefixed with {\tt -X:aos:opt} will have no effect.
For example to optimize compile a method at optimization level 1, when
the {\tt optonly} strategy is specified, use the {\tt -X:aos:irc:O1}
option.  (Note that to obtain the functionality of a JIT at
optimization level 1, {\tt -X:primary\_strategy=optonly
-X:aos:irc:O1}, in a nonadaptive configuration would be achieved with
the command-line option {\tt -X:irc:O1}.) 

%%%%%%%%%%%%%%%%%%%%%%%%%%%%%%%%%%%%%%%%%%%%%%%%%%%%%%%%
\subsection{JMTk Command-Line Options}
\label{section:jmtkoptions}

To see descriptions of command-line options to JMTk,
use the {\tt help} option with the {\tt -X:gc:} prefix 
to generate the following output:

\T \begin{tiny}
\input{jmtk_options}
\T \end{tiny}

%%%%%%%%%%%%%%%%%%%%%%%%%%%%%%%%%%%%%%%%%%%%%%%%%%%%%%%%
\subsection{Adding Jikes RVM Non-Standard Command-Line Options}
This section states how non-standard Jikes\trademark RVM command-line
options can be added to Jikes RVM.  Non-standard Jikes RVM
command-line options are those options that are specific to Jikes RVM.
The format of a non-standard Jikes RVM option is {\tt -X:o=v} where
{\tt o} is the option and {\tt v} is the value it is to be set to.
Adherence to this format is important to keep command-line options
processing from becoming unwieldy.

Jikes RVM command-line options are processed in two places: 
{\tt RunBootImage.C} and {\tt VM\_CommandLineArgs.java}.  
{\tt RunBootImage.C} is called first before the boot image is loaded, and
{\tt VM\_CommandLineArgs.java} is called after.  In addition to
processing any option which does not require the Jikes RVM boot image
to be loaded (such as {\tt help} and {\tt version}), 
{\tt RunBootImage.C} processes any non-standard option that impacts either
heap size, message output, or where to find the boot image.  To allow
unrestricted order of Jikes RVM options and because command-line processing
stops at the first option that is not recognized as a Jikes RVM
option, all Jikes RVM options must be recognized by {\tt RunBootImage.C}
and passed on.

\JikesTMFooter
