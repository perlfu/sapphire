
\label{section:cmdline}%
\index{command line arguments}%
%
This section describes how non-standard Jikes\TMweb{} RVM command
line options are specified. Command line options can be used to modify
the behavior of the runtime system, adaptive optimization system, the
optimizing compiler, the baseline compiler, or the memory manager.  A
command-line directive is constructed by concatenating an option with
a prefix which identifies the desired destination for that option.

All of the non-standard VM options must occur before 
the application class name and application's command-line options.

The order of non-standard command line argument is significant. Each
command line option is processed in order, as typed on the command
line. For example, using a {\tt printOptions} directive for a
subsystem will print the current value of the subsystem's options and
this will reflect only those command line arguments that have preceded
the {\tt printOptions} directive.

%%%%%%%%%%%%%%%%%%%%%%%%%%%%%%%%%%%%%%%%%%%%%%%%%%%%%%%%
\subsection{VM Command-Line Options}
\label{section:vmoptions}

To see descriptions of command-line options to the core virtual
machine, use the {\tt help} option with the {\tt -X:vm:} prefix to
generate the following output:

\T \begin{tiny}
\input{vm_options}
\T \end{tiny}


%%%%%%%%%%%%%%%%%%%%%%%%%%%%%%%%%%%%%%%%%%%%%%%%%%%%%%%%%%%%%%%%%%%%%%%
\subsection{Adaptive Optimization System (AOS) Command-Line Options}

To see a description of the command-line options to the AOS, use 
{\tt -X:aos:help}.  As of this writing, this command produces the
following output:

\T \begin{tiny}
\input{adaptive_options}
\T \end{tiny}

All options in an adaptive configuration are prefixed with {\tt
-X:aos}.  To pass an option to the adaptive optimization system, use
the {\tt -X:aos:} prefix.  For example, to set the logging level of
AOS to one, use the directive {\tt -X:aos:logging\_level=1}. An
adaptive configuration may conceptually have many optimizing compilers
that are available at runtime, each with its own set of option values.
We present a mechanism to address each conceptual optimizing compiler.
To pass options to the opt compiler that recompiles a method use the
{\tt -X:recomp[?]} prefix where the {\tt ?} is optional and if
specified is an integer that identifies the optimization level.  For
example, {\tt -X:recomp2:global\_bounds=true} performs global array
bounds check elimination on demand when a method is recompiled at
optimization level 2.  If no optimization level is specified, the
option applies to all optimization levels of the optimizing compiler
that recompiles methods.  For example, {\tt
-X:recomp:global\_bounds=true} performs global array bounds check
elimination on demand whenever a method is recompiled with
optimization.  In the default adaptive configurations, the initial
runtime compiler is the baseline compiler.  Options are passed to the
initial runtime compiler by prefixing each option with {\tt -X:irc:}.  

The {\tt enable\_recompilation} option determines whether or not the
adaptive system will actually adaptively recompile methods.  By
default, this is set to true. By setting this to false, Jikes RVM can
simulate a JIT-only system by compiling each method on first
invocation with a specified {\tt initial\_compiler}. For example, 
by saying {\tt -X:aos:enable\_recompilation=false
-X:aos:initial\_compiler=opt}, one can force all methods to be
optimized on first invocation. 
This has the effect of making the initial runtime compiler
be the optimizing compiler (and no recompilations will take place), so
subsequent options in the command line that are prefixed with {\tt
-X:irc:} will be passed to the optimizing compiler. Since no
recompilation will take place options prefixed with {\tt -X:recomp:}
will have no effect.  For example to optimize compile a method at
optimization level 1, use the {\tt -X:irc:O1} option. 

%%%%%%%%%%%%%%%%%%%%%%%%%%%%%%%%%%%%%%%%%%%%%%%%%%%%%%
\subsection{Baseline Compiler Command-Line Options}
\label{section:nonadaptive:baseline:options}

To see descriptions of the command-line options to the baseline
compiler, use the {\tt help} option with the prefix {\tt -X:base:} 
to generate the following output:

\T \begin{small}
% From VM_BASEOptions.template
\begin{verbatim}
-X:base[:help]			Print brief description of baseline compiler's command-line arguments
-X:base:printOptions		Print the current values of the active baseline compiler options

Boolean Options (-X:base:<option>=true or -X:base:<option>=false)
option                               Description
annotations                            Act on annotations in class files
preload_as_boot                        Apply boot options to preload_class
verbose                                Print method name at start of compilation
mc                                     Print final machine code

Value Options (-X:base<option>=<value>)
option                         Type    Description
preload_class                  String  Class to preload upon 1st OPT compilation

Selection Options (set option to one of an enumeration of possible values)

Set Options (option is a set of values)
method_to_print                Only apply print options against methods whose name contains this string
method_to_break                Invoke breakStub for jdp's benefit after compiling methods whose name contains this string
\end{verbatim}


\T \end{small}
 
Note that when the initial runtime compiler is the baseline compiler, 
the {\tt -X:irc} and the {\tt -X:base} command line prefixes are equivalent.

%%%%%%%%%%%%%%%%%%%%%%%%%%%%%%%%%%%%%%%%%%%%%%%%%%%%%%%%
\subsection{Optimizing Compiler Command-Line Options}
\label{section:nonadaptive:optimizing:options}

To see descriptions of command-line options to the optimizing compiler,
use the {\tt help} option with the {\tt -X:opt:} prefix 
to generate the following output:

\T \begin{tiny}
\input{opt_options}
\T \end{tiny}

Note that when the initial runtime compiler is the optimizing compiler, 
the {\tt -X:irc} and the {\tt -X:opt} command line prefixes are equivalent.

%%%%%%%%%%%%%%%%%%%%%%%%%%%%%%%%%%%%%%%%%%%%%%%%%%%%%%%%
\subsection{JMTk Command-Line Options}
\label{section:jmtkoptions}

To see descriptions of command-line options to JMTk,
use the {\tt help} option with the {\tt -X:gc:} prefix 
to generate the following output:

\T \begin{tiny}
\input{jmtk_options}
\T \end{tiny}

%%%%%%%%%%%%%%%%%%%%%%%%%%%%%%%%%%%%%%%%%%%%%%%%%%%%%%%%
\subsection{Adding Jikes RVM Non-Standard Command-Line Options}
The supporting code for virtually all of the non-standard
Jikes\TMweb{} RVM command-line options (and much of the content of
this section of the userguide) is machine generated from
option description files.  Each of the major sub-systems (VM, AOS, GC,
Baseline Compiler, Optimzing Compiler) has its own set of description
files and corresponding name space of options. 

At a high level, adding a new command line argument can be done just
by adding a new record to the appropriate option description file.  We
strongly discourage adding command line options by any other means as
doing so leads to more work in maintaining the help messages, option
processing, and userguide. Although the details of the template and
option description files varies slightly from one subsystem to the
other, the general ideas are the same.  

\IndexttClass{OPT\_Options}%
As an example, here is a discussion on how to add a command line
option to the optimizing compiler.
Jikes RVM stores command-line options to the optimizing compiler 
as fields in an object of type {\tt OPT\_Options}.
The Jikes RVM build process generates the {\tt OPT\_Options.java} 
file automatically from a template.  

\Indextt{BooleanOptions.dat}%
\Indextt{ValueOptions.dat}%
To add or modify the command-line options in {\tt OPT\_Options.java},
you must modify either {\tt BooleanOptions.dat} or 
{\tt ValueOptions.dat}.  You should describe your desired
command-line option in a format described below.
Your option will be generated the next time you build the
system.

%%%%%%%%%%%%%%%%%%%%%%%%%%%%%%%%%%
\subsubsection{\texttt{BooleanOptions.dat}}

The \IndexTexttt{BooleanOptions.dat} file defines boolean options for
the optimizing compiler.  The file describes each command-line option 
by a two-line record, and each record is separated
by a blank line.  Long lines can be continued using ``$\backslash$''.
{\bf NOTE:} blank lines {\em are} important!
Lines starting with ``\#'' are ignored.

The first line must have the following format
\begin{quote}
\begin{verbatim}
FULL_NAME OPT_LEVEL DEFAULT_VALUE {SHORT_NAME}
\end{verbatim}
\end{quote}
where
\begin{itemize}
\item {\tt FULL\_NAME} gives the name of the boolean field in {\tt OPT\_Options.java}
\item {\tt OPT\_LEVEL} gives the minimum optimization level that automatically sets this field true
\item {\tt DEFAULT\_VALUE} of {\tt true} or {\tt false}
\item {\tt SHORT\_NAME} is an optional field which defines a mnemonic by which the command-line processor recognizes this option.
\end{itemize}

The second line of each record must hold a short textual description
of the semantics of the option.  The system will print this
description when so instructed by the command-line option {\tt
-X:opt:help}.

For example, the two line record in {\tt BooleanOptions.dat} that
defines the option of whether to perform local scalar replacement is
\begin{verbatim}
LOCAL_SCALAR_REPLACEMENT 1 true local_sr
Perform local scalar replacement
\end{verbatim}

%%%%%%%%%%%%%%%%%%%%%%%%%%%%%%%%
\subsubsection{ValueOptions.dat}

The \IndexTexttt{ValueOptions.dat} file defines non-boolean options for
the optimizing compiler.  The file describes each command-line option 
with a three-line record, and each record is separated
by a blank line.  As with \texttt{BooleanOptions.dat},
long lines can be continued using ``$\backslash$'' and
blank lines are once again significant.
Lines starting with ``\#'' are ignored.

The first line must have the following format
\begin{quote}
\begin{verbatim}
TAG FULL_NAME TYPE DEFAULT_VALUE {SHORT_NAME}
\end{verbatim}
\end{quote}
where
\begin{itemize}
\item {\tt TAG} is 'E' for an Enumeration type, and 'V' for a value type.  Further instructions for Enumeration types appear below.
\item {\tt FULL\_NAME} gives the name of the field in {\tt OPT\_Options.java}
\item {\tt TYPE} is one of 'byte', 'int', or 'String', and gives the primitive datatype for the value in OPT\_Options.java
\item {\tt DEFAULT\_VALUE} is the default value for the option
\item {\tt SHORT\_NAME} is an optional field which defines a mnemonic by which the command-line processor recognizes this option.
\end{itemize}

The second line of each record must hold a short textual description
of the semantics of the option.  The system will print this
description when so instructed by the command-line option {\tt
-X:opt:help}.

The third line of each record is used for enumeration options, and must
be left blank for non-enumeration options.

For example, the three-line record in {\tt ValueOptions.dat}
that defines the maximum inlining depth when using static inlining
heuristics is
\begin{verbatim}
V IC_MAX_INLINE_DEPTH int 5
Static inlining heuristic: Upper bound on depth of inlining
<blank line>
\end{verbatim}

Enumeration options provide a mechanism to define an option in terms of 
a small fixed set of choices.  For an enumeration option, the third line
of the record must contain a specification for each value that the
enumeration can take.  Each such specification must have the following
format:
\begin{verbatim}
"ITEM_NAME QUERY_NAME CMD_NAME"
\end{verbatim}
where
\begin{itemize}
\item {\tt ITEM\_NAME} gives the name of the enumeration value in {\tt OPT\_Options.java}
\item {\tt QUERY\_NAME} gives the name of an accessor function which returns {\tt true} iff the enumeration takes the value {\tt ITEM\_NAME}.
\item {\tt CMD\_NAME} is the name to pass on the command-line to set the enumeration to this value.
\end{itemize}
The quotes are important, and the specifications should be
space-separated.

For example, the optimizing compiler supports a choice of three
options for floating-point optimization rules.  The three-line record
describing these options is:
\begin{verbatim}
E FP_MODE byte FP_STRICT
Selection of strictness level for floating point computations
"FP_STRICT strictFP strict" \
"FP_ALLOW_FMA allowFMA allow_fma" \
"FP_LOOSE allowAssocFP allow_assoc"
\end{verbatim}
Notice how the third line was broken up by using ``$\backslash$''.

So, by default, Jikes RVM uses the {\em strict} floating-point
semantics.  To use the option that allows fused multiply-add
({\texonly{\sc} FMA})
instructions, specify ``{\tt -X:irc:allow\_fma}'' on the command-line.
Given an {\tt OPT\_Options} object called {\tt options}, your code can
query if {\texonly{\sc} FMA} is allowed by testing ``{\tt op\-tions\-.al\-low\-F\-M\-A()}''.

