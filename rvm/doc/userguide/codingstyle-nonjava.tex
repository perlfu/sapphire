Besides Java , you may
encounter other types of source code if you poke into some dark
corners of Jikes RVM.  

There are several types of source code in the Jikes RVM system.  They
include:


\subsection{Java} 
($\approx 1045$ source files as of this writing)  This is
the vast majority of the code.  The names of Java source files always end in
`{\tt .java}''.   We discuss Java coding style in
section~\ref{section:javacodingstyle}. 

We have found it necessary to preprocess the Java code; the
preprocessor is discussed in section~\ref{section:preprocessor}.


\subsection {C}

($\approx 26$ source files)  The file names always
end in ``{\tt .c}''.  We are committed to GNU C.  One of the core team
members uses the latest GNU C (3.3 as of this writing), but many users
are still running GNU C 2.95 and 2.96.  So if you use constructs that
would not be legal in GNU C 2.95, then please surround them with
appropriate preprocessor conditionals.  It is OK to use constructs
found through GCC 2.95.

Our C style is straight Kernighan and Ritchie, with four-space
indenting.  

\subsection{C++} 

($\approx 14$ source files)  The file names always end in
``{\tt .C}''.   The material discussed for C applies here as well.

\subsection{Bourne Shell (Bash)}

We use the GNU Bourne-Again Shell (Bash) exclusively in our shell
scripts.  Our build system is driven by one massive Bash script, {\tt
rvm/bin/jconfigure}, which in turn generates other scripts in the {\tt
RVM\_BUILD} directory.

\subsubsection{Indenting}

We use four-space indenting for Bash; you'll find older parts of the
Bash code that use two-space and three-space indenting.   We limit
lines to 80 columns unless necessary for syntactic reasons to do otherwise.

\subsubsection{Declaring functions}

We use the {\tt function} keyword to declare functions:
\begin{verbatim}
function emitCopier () {
\end{verbatim}
rather than eliding it:
\begin{verbatim}
emitcopier () {
\end{verbatim}

\subsubsection{Exit in case of Build Error}

An important consideration for the build process is that if trouble
happens while buiding a part of the system, the build should abort
rather than continuing on.  We have encountered several problems with this,
where the build process continued on despite trouble building the GNU
Classpath library, and users then got an RVM that did not work.  

The Bash source code used in the build system is protected with 
{\tt set~-e}, which causes the shell to immediately exit in case of
trouble, and (on Bash versions that support it) with a {\tt trap}
against the {\tt ERR} pesudo-signal.

\paragraph{Subshells}

{\tt -e} and {\tt ERR} do not apply to commands executed within
subshells.  So, if you want to execute Make in a particular directory,
{\it directory-name}, it's easier to use:
\begin{example}
\tt{}make -C {\it directory-name} {\it{}make-target}
\end{example}
instead of:
\begin{example}
\tt{}( cd {\it directory-name} && make {\it{}make-target})
\end{example}
If you want to use a subshell in your build-system code, then try:
\begin{example}
\tt{}({\it subshell-commands }) || false
\end{example}
which will make the right thing happen, or:
\begin{example}
\tt{}if ! ({\it subshell-commands }); then 
    echo >&2 "$ME: Something bad happened."
    exit 1
fi

\end{example}

\subsection{GNU Makefiles} 

We are committed to GNU Make.  All new makefiles should be named {\tt
GNUmakefile}, just as discussed in the Sun style guide mentioned in
section~\ref{section:javacodingstyle}.  We have gone over to pure Bash
(we used to use Korn shell, for historical reasons); authors of
Makefiles can safely use any Bash constructs.

