This section describes a set of coding style guidelines that we
recommend for all code added to the RVM  system.  

Regrettably, much code in the current system does not follow any consistent
coding style.  This unfortunate residue of the system's evolution 
make editing sometimes unpleasant, and prevent javadoc from formatting comments
in
many files.  To alleviate this problem, we present this style guide 
(which consists of a small tweak of the style guide advanced by Sun) 
for new code. 

\index{javadoc}
Most code in the optimizing compiler has been formatted to at least obey 
the indentation rule, 80 columns, and javadoc comments.  Most other RVM code
has not been so formatted. We ask that all new code introduced
into the system, at least for the optimizing compiler, follow the 
guidelines unless compelling factors dictate otherwise.  

\subsection {Coding style description}

The RVM coding style guidelines are defined as a reference to an Sun
Microsystems ``Code Conventions for the Java\trademark Programming Language'',
with a few exceptions listed below.  The Sun coding
conventions can be found at 
\xlink{http://java.sun.com/docs/codeconv} 
{http://java.sun.com/docs/codeconv} in HTML,
postscript, and PDF.  Most of the style guide is intuitive; 
however, please read through the document or look at its sample code.

We have adopted one revision to the Sun code conventions:
\begin{enumerate}
\index{indenting}
\item {\bf Two space indenting} The Sun coding convention suggests 4
space indenting, however with 80 column lines and 4 space indenting,
there is very little room left for code.  Thus, we recommend using 2
spaces indenting.

\end{enumerate}

\subsection {Javadoc requirements}
\index{javadoc}

All files should contain descriptive comments
in Javadoc form (
\xlink{{\tt http://java.sun.com/j2se/javadoc/index.html}}
{http://java.sun.com/j2se/javadoc/index.html}
) so
that documentation can be generated automatically.  Of course,
additional non-javadoc source code comments should appear as
appropriate.
For javadoc, at a minimum,

\begin{enumerate}
\item All classes and methods should have a block comment describing
them
\item All methods contain a short description of their arguments
(using {\tt @param}), the return value (using {\tt @return}) and the
exceptions they may throw (using {\tt @throws}).
\item Each class should include {\tt @see} and {\tt @link} 
references as appropriate.
\end{enumerate}

\subsection {Useful tools/hints}
\index{editing source code}
\index{vi}
\index{emacs}

This section describes helpful hints for conforming with the style
guide.  Below are suggestions on how to setup the two most common
editors, emacs and vi. 
%\remark{If we find a pretty-print code processor, we
%can describe it in this section.}

\subsubsection{emacs} 

The following tells {\tt emacs} to indent 2 spaces:
\begin{verbatim}
;; You have to do it in this complicated way because of the
;; strange way the cc-mode initializes the value of `c-basic-offset'.
(add-hook 'c-mode-hook (lambda () (setq c-basic-offset 2)))
(add-hook 'java-mode-hook (lambda () (setq c-basic-offset 2)))
\end{verbatim}
If you want {\tt emacs} to truncate long lines instead of wrapping them, add
the following to your c/java mode hook:
\begin{verbatim}
(setq truncate-lines 't)
\end{verbatim}

\subsubsection{vi}\label{options:vi/vim}

If you are more comfortable with {\tt vi}, it is recommended that you
use a {\tt vi} clone called {\tt vim} (\xlink{{\tt http://www.vim.org/}}{http://www.vim.org}).  It
contains all of {\tt vi}'s commands and is fully backward compatible,
but is much more configurable than {\tt vi}.  Hints for {\tt vi}
diehards who absolutely refuse to use {\tt vim} are provided at the end
of this subsection (\ref{options:vi}).

\paragraph{vim}\label{options:vim}

Add the following to your {\tt .vimrc} for formatting:
\begin{verbatim}
set shiftwidth=2           " for indenting and shifting
set expandtab              " to replace tab characters by spaces
set smarttab               " to allow the use of <Tab> for indenting
set formatoptions-=t2croq  " reset formatting
set formatoptions+=croq    " format comments
set textwidth=0            " don't wrap text
set wrapmargin=0           " ditto
" Java mode setup
augroup java
   autocmd!
   autocmd BufEnter *.java set cindent
   autocmd BufEnter *.java set cinoptions=>s,e0,n0,f0,{0,}0,^0,:s,=s,ps,ts,c3,+s,(0,u0,)20,*30,gs,hs
   autocmd BufEnter *.java set cinwords=if,else,while,do,for,switch,static,new
   autocmd BufLeave *.java set nocindent
augroup END
\end{verbatim}
If you want {\tt vim} to truncate long lines instead of wrapping them, add
the following to your {\tt .vimrc}:
\begin{verbatim}
set formatoption+=t " to allow autowrap text
set textwidth=74    " to allow autowrap text at 74th column
\end{verbatim}

\paragraph{vi}\label{options:vi}

Standard {\tt vi} options that would approximate Java formatting are:
\begin{verbatim}
set shiftwidth=2  " for indenting and shifting
set autoindent    " automatically indent new lines to the start of previous
\end{verbatim}
and the approximation for wrapping long lines is
\begin{verbatim}
set wrapmargin=6  " to allow autowrap text at 74th column
\end{verbatim}

% LocalWords:  Javadoc param

