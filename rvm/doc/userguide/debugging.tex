Previous releases of Jikes\TMweb\ RVM included a home grown
debugger called {\bf jdp}.  The jdp debugger is no longer included in
\jrvm{}.  This is because, although the jdp debugger provided some
useful functionality for debugging Jikes RVM itself, it was always a
little unstable and was not really intended to support source level
debugging for programs running on top of Jikes RVM.  Prior to the
2.2.0 release, the Jikes RVM core team decided to deprecate jdp
because we believe that the cost of maintaining it outweighs its
usefulness.


The future plan for debugging Jikes RVM consists of two main pieces of
work:

\label{JDWP}
\index{JDWP}
\begin{itemize}
\item Implement \xlink{\textbf{JDWP}}{\JDWPURL} support in Jikes RVM. If Jikes RVM
implemented the standard \xlink{Java Debugging Wire Protocol
(JDWP)}{\JDWPURL}, then a 
number of debugger front ends could be used with Jikes RVM.  This
would support most debugging tasks, but would obviously not work if
Jikes RVM itself was crashing or corrupted.

\item Low level debugging of Jikes RVM itself would be performed using
the GNU {\bf gdb} debugger.  Work could be done to either or both of
{\tt gdb} and Jikes RVM to make this more pleasant.  For example,
Jikes RVM could be extended to generate basic {\tt stabs} information
for the boot image.
\end{itemize}

Contributions from the community along either of these fronts would be
greatly appreciated.  Some prototyping of \xlink{JDWP}{\JDWPURL} support for Jikes RVM
was done at IBM during the summer of 2002; we can make this code
available if anyone is interested in completing it.


