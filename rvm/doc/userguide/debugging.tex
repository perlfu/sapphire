Previous releases of Jikes\JikesTMFootnote\ RVM included a home grown
debugger called jdp. The jdp debugger provided some useful
functionality for debugging Jikes RVM itself, but has always been a
little unstable and was not really intended to support source level
debugging for programs running on top of Jikes RVM. Prior to the 2.2.0
release, the Jikes RVM core team decided to deprecate jdp because we
believe that the cost of maintaining it outweighs its usefulness.

The future plan for debugging Jikes RVM consists of two main pieces of
work. 
\begin{itemize}
\item Implement JDWP support in Jikes RVM. If Jikes RVM implemented
the standard debugging wire protocol (JDWP), then a number of debugger
front ends could be used with Jikes RVM.  This would support most
debugging tasks, but would obviously not work if Jikes RVM itself was
crashing or corrupted.
\item Low level debugging of Jikes RVM itself would be done using
gdb. Work could be done either in gdb or Jikes RVM to make this
marginally more pleasant.  For example, Jikes RVM could be extend to
generate basic stab information for the bootimage.
\end{itemize}
Contributions from the community along either of these fronts would be
greatly appreciated.  Some prototyping of JDWP support for Jikes RVM
was done at IBM during the summer of 2002; we can make this code
available if anyone is interested in completing it.


