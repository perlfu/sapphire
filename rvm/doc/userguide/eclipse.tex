\newcommand{\eclipseURL}{http://www.eclipse.org}
\newcommand{\antURL}{http://ant.apache.org}

\index{Eclipse IDE, running under Jikes RVM}
Since version 2.2.1, Jikes RVM purports to run
\xlink{Eclipse}{\eclipseURL}.  This is really a technology preview and
for that reason we start with some caveats, the most obvious being
that production-mode use of Eclipse with Jikes RVM
is not recommended at this time.  Furthermore:
\begin{itemize}
\item We strongly recommend you run a
recent---7.3 or newer---version of RedHat Linux for Intel IA32.
\item Use either *Adaptive* or BaseBase*  Jikes RVM builds when trying
to run Eclipse.  The *Opt* images will attempt to optimize all
executed methods, which will result in bad performance.
\item Use only the *SemiSpace builds of Jikes RVM.  We have seen
various limitations of the other garbage collectors, such as severe
fragmentation and implementation-specific resource limits.
\item Under Linux, use the Eclipse builds for the GTK windowing
toolkit.  There are Linux builds for the Motif toolkit, but we never
test these and work on Eclipse seems to focus on the GTK version.

\item We strongly recommend Eclipse 2.1.1, currently the latest release
from \xlink{{\tt \eclipseURL}}{\eclipseURL}; this is what we use and
test at Watson.  We have also successfully run Eclipse 2.1.0; Eclipse 2.0.x releases may also work.  Do not try to
use Eclipse 1.x versions.

\item The Eclipse debugging support for Java programs --- which relies
on talking JDWP to the VM --- will not work because
Jikes RVM has yet to implement this protocol.  (It's on our TODO list, though.)

\item Running Eclipse requires that you build Jikes RVM with
RVM\_FOR\_SINGLE\_VIRTUAL\_PROCESSOR equal to 0. The
select-interception mechanism (which is required to run Eclipse on
Jikes RVM) currently does not work if this variable is set to 1 (see
defect \# 3583).

\item As of this writing (August 11, 2003), 
Jikes RVM defect \# 2046 is still outstanding:
Jikes RVM's {\tt VM\_IdleThread} chews up CPU time when waiting for
something to happen.  This idle thread ends up competing for CPU time
with any activities you're engaged in outside of the Jikes RVM process
running Eclipse.  In other words, your web browser will slow down,
etc.  We plan to have this fixed soon.

\end{itemize}

Given these caveats, Eclipse has been running on Jikes RVM at
Watson in various configurations since the summer of 2002, and
more-or-less all of the standard Eclipse functionality appears to work
(The single biggest exception is debugging support, as mentioned
above).  We have successfully developed and run Java programs, checked
projects out of CVS repositories, used the refactoring support, run
the Web-based help system, applied updates to Eclipse, and so on.
Furthermore, performance is generally not too bad, at least when
adaptive builds of Jikes RVM, such as FastAdaptiveSemiSpace, 
are used.  Most importantly, we very much
want users to try running Eclipse on Jikes RVM: we eagerly
solicit feedback regarding bugs and would be especially grateful for
any contributions that enhance the usability of Eclipse on Jikes RVM.  

Setting up Eclipse and Jikes RVM is relatively straightforward; you
have to install Jikes RVM and Eclipse themselves first, and, after
that, there are just a few steps:
\begin{enumerate}
\index{Ant, the Java-based build tool}
\item Install Ant---the Java-based build tool.  You can get this from
\xlink{{\tt \antURL}}{\antURL}. 
\item Add two variables to your Jikes RVM configuration file:
 \begin{description}
  \index{ANT\_CMD}
 \item[ANT\_CMD] is the full pathname of the executable Ant command
  \index{ECLIPSE\_INSTALL\_DIR}
 \item[ECLIPSE\_INSTALL\_DIR] is the root of your Eclipse install
 \end{description}
\item In a fresh build directory, run {\tt ./jbuild.plugin} to install
support for Eclipse to use Jikes RVM
\end{enumerate}

\index{rvmeclipse}
 Once you have done this, you will be able to run Eclipse on Jikes RVM
using the {\tt \$RVM\_ROOT/rvm/bin/rvmeclipse} command.  This wrapper
calls the normal Eclipse command, instructing it to use Jikes RVM to
run Eclipse.  Set RVM\_ROOT and RVM\_BUILD before you
invoke {\tt rvmeclipse}.
