\begin{center}
{\bf RVM Frequently Asked Questions}
\end{center}


\subsection{Basic RVM Questions}

\subsubsection{What is RVM?}

The Jikes Research Virtual Machine for Java is an open-source virtual machine
implementation, written in Java, intended to serve as infrastructure for
programming language research.

\subsubsection{Was RVM once called \jp? What's the difference?}

Yes. There is no difference.  Call a pepper by any other name, and does it
not still burn your mouth?

\subsection{Getting RVM}

See the RVM web pages on DeveloperWorks.

\subsection{Licenses}

The RVM implementation is licensed open-source under the Common Public
License.  There are separate, more restrictive licenses for the binary and 
source to the Java standard libraries for RVM.  See the DeveloperWorks
web pages for more details.

\subsection{Building RVM}

\subsubsection{Why is bootimage writing so slow?} 
\begin{description}
\item [Q:] 
Boot image writing seems to be the most time-consuming step in the
RVM build process.  Has anybody tried to or thought about
creating an incremental boot image writer? It might speed up the
development cycle quite a bit (I guess the standard reply to this will
be that I better keep the stuff I'm working on out of the boot image).

\item [A:]
Incremental boot image building is not a trivial problem.  One big
issue is: if we change the implementation of one class in the boot image,
what other parts of the VM image must be invalidated?  One example: which
methods must be recompiled to reflect the new implementation?  We have no
mechanism in place to trace these kinds of dependencies.  There are other
examples, too.  In summary: incremental boot image writing would be nice,
but it's not easy to support, and it hasn't been at the top of our
priorities.
\end{description}

\subsubsection{Which jikes should I use?}
At Watson, we're currently using jikes v1.13 to compile the RVM source on
both Linux and AIX.  We've had reports from users that v1.14 has problems
on Linux.  In order to build the rvmrt.jar library, we applied patch 62 to
the jikes build to fix a jikes problem with variable scoping.

\subsection{System Structure}

\subsubsection{Why doesn't the RVM source use packages?}

This is a historical artifact.  In the early days of the project, we did
not want to be constrained by a package structure to a particular
directory structure.  With the current build process, this is not an
issue, and we may incrementally add package structure to the
implementation over time.

\subsection{Optimizing Compiler}

\subsubsection{What is the OptTestHarness?}

The OptTestHarness is a driver program to run the optimizing compiler even
on a BaseBase boot image.  This driver is useful for optimizing compiler
development, since you can use the driver to selectively compile
individual methods with certain options.
