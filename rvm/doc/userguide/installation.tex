This section gives instructions on how to install and run Jikes\TMweb{} RVM.
If your system meets certain prerequisites, the install process is
quite simple; if not, you may need to do some preparatory work.  We
present the simple case first.  We then present details for installing
on other platforms, and finally discuss further installation steps for
running Eclipse on Jikes RVM.

\subsection{Installing on Red Hat Linux 7.3, 8.x, and 9.x, or SuSE Linux 8.x for~Intel~x86}

Red Hat\TM{} Linux\R{} for Intel\R{} IA32 (Intel x86) has emerged as the 
primary platform for Jikes RVM.  SuSE\R{} Linux is also in daily use
by one of the Jikes RVM core team members.
If you have either a developer or full install of Red~Hat
Linux~7.3 or later, your system has almost everything you need
to run Jikes RVM.  The following instructions describe how to install
and build Jikes RVM for this platform:
\begin{enumerate}

\index{Blackdown Java Virtual Machine}
\index{Java Virtual Machine (JVM)}
\index{JVM (Java Virtual Machine)}
\item Download and install a Java\TMweb{} Virtual Machine if you have not
already done so.  We currently recommend the Blackdown VM v1.4.1. 
For versions of Jikes RVM prior to 2.2.2, we recommend
the Blackdown VM, v1.3.1.  You can download these VMs from 
\xlink{{\tt \BlackdownURL}} {\BlackdownURL}, or, if you have SuSE
Linux 8.2, install the two RPMs {\tt BlackdownJava2-JRE} and {\tt java2}. 
It is possible to use other 1.3 or 1.4 JVMs to build Jikes RVM,
however building Jikes RVM is a good way to find bugs in a JVM.
\index{IBM JVM on Linux}
JVMs that are known to fail on Linux/IA32 include the IBM\Rweb{}~1.3.1 and
1.4.0.  People have reported some success with the IBM~1.4.1 (let us
know if you have success/problems with IBM~1.4.1). Also, please note
that Red Hat~8.x does have a {\tt java} command, but that does not invoke a
JVM and will not suffice for building Jikes RVM.

\index{Jikes compiler}
\item Download and install the Jikes (Java source to bytecode)
compiler, if you have not already 
done so (it is included in the standard SuSE install, but not in
the standard Red Hat install).  You can 
obtain Jikes from \xlink{the Jikes developerWorks site}{\jikesURL}.
Use the pre-built RPM file for {\tt jikes-1.18} or higher.   
If you're using SuSE Linux~8.2 or newer, Jikes~1.18 is on the install
CD in the package {\tt jikes}.


{\tt jikes-1.15} in
particular --- installed by default with SuSE Linux 8.1 --- has a
byte-code generation bug that will cause Jikes RVM to crash with an
{\tt ArrayIndexOutOfBounds} exception.  {\tt jikes-1.13}, as of the last
time we tried it (August 8, 2003) works fine, and sometimes generates
better error messages than {\tt jikes-1.18}.

\item Download Jikes RVM from developerWorks and unpack it.  We shall
hereinafter call the location of the RVM distribution {\tt \$RVM\_ROOT/rvm}.

\index{GNU Classpath: Java class libraries}
\index{Classpath, GNU: Java class libraries}
\item You will need version \classpathversion{} of the GNU Classpath
libraries.  The Jikes RVM build process can automatically download, 
configure, and build the appropriate version of classpath for you.
(This has historically sometimes led to confusion on the part of the
builder.  We expect to have these glitches ironed out before the next
release, but if you have any difficulty please see section~\ref{manual-classpath-root}.)

\item Edit a configuration file so that it fits your machine.  One of
the {\tt i686-pc-linux} files in {\tt \$RVM\_ROOT/rvm/config} will probably be
more-or-less right to begin with.  On Linux, most standard commands
(and Jikes) are in {\tt /usr/bin}.  Hereinafter, we will call your edited
file {\tt \$RVM\_ROOT/rvm/config/i686-pc-linux-gnu.mine}

\index{environment variables}
\item Setup your environment with definitions required by Jikes RVM.
There are four of these that you must use; in addition,
{\tt \$RVM\_ROOT/rvm/bin} should be in your path.
\begin{description}

\item[RVM\_ROOT] is the location of the Jikes RVM distribution.  It
must be {\bf RVM\_ROOT} as given above.

\item[RVM\_HOST\_CONFIG] is the configuration file of the machine
building Jikes RVM.  It must be
{\tt \$RVM\_ROOT/rvm/config/i686-pc-linux-gnu.mine}.

\item[RVM\_TARGET\_CONFIG] is the configuration of the machine running
Jikes RVM.  It must be {\tt \$RVM\_ROOT/rvm/config/i686-pc-linux-gnu.mine}.

\item[RVM\_BUILD] is the location where Jikes RVM is to be built.  It
can be anywhere.
\end{description}

\item Setup a build directory with ``{\tt jconfigure prototype}'', \link{or
use some other configuration if you prefer.}[  See Section~\Ref
for information about the various configurations.]{configs}

\item Build Jikes RVM: go to RVM\_BUILD and type ``{\tt ./jbuild}''

\end{enumerate}

\newcommand{\gccURL}{ftp://ftp.gnu.org/gnu/gcc}
\newcommand{\glibcURL}{ftp://ftp.gnu.org/gnu/glibc}
\newcommand{\makeURL}{ftp://ftp.gnu.org/gnu/make}
\newcommand{\bashURL}{ftp://ftp.gnu.org/gnu/bash}
\newcommand{\bisonURL}{ftp://ftp.gnu.org/gnu/bison}
\newcommand{\tarURL}{ftp://ftp.gnu.org/gnu/tar}
\newcommand{\autoconfURL}{ftp://ftp.gnu.org/gnu/autoconf}
\newcommand{\automakeURL}{ftp://ftp.gnu.org/gnu/automake}
\newcommand{\wgetURL}{ftp://ftp.gnu.org/gnu/wget}
\newcommand{\cvsURL}{http://www.cvshome.org}
\newcommand{\linuxPPCJDKURL}{http://www.ibm.com/java/jdk/linux/index.html}
\newcommand{\linuxKernelURL}{http://www.kernel.org}

\subsection{The Hard Way}

 For any platform that is not SuSE\Rweb{} Linux\Rweb{} 8.2 or newer, or Red Hat\TMweb{} Linux 7.3 or newer for Intel\Rweb{}
IA32, installation can be more complicated.  We present a list of system
prerequisites, and then a series of install steps that assumes you
have those.

\subsubsection{System Prerequisites}

\newcommand{\SeeBelow}{{\small \em (See Below)}}
\begin{Label}{prereqs}
\begin{table}[h]
%% This table can be way too wide to fit on a page.  If it does get to
  %% be too wide, then please uncomment the call to resizebox below,
  %% and its maching closing brace.  At that point, the page will look
  %% bad under XDVI; you'll have to look at the PostScript instead to
  %% see the effects.
%\resizebox{\linewidth}{!}{%
%\begin{center}
\begin{tabular}{|l|l|l|} \hline
\hline {\em software} & {\em version} & {\em site} \\ 
\hline \multicolumn{3}{|c|}{\em All platforms}                     \\ \hline
GNU make       & 3.79+    & \xlink{\tt \makeURL}{\makeURL}         \\ 
%% Including the amplification below makes the table much too wide. --augart
{\tt bash} % (The Bourne-Again Shell) 
        & 2.05a+ \SeeBelow   & \xlink{\tt \bashURL}{\bashURL}         \\ 
GNU tar        & 1.13+    & \xlink{\tt \tarURL}{\tarURL}           \\ 
{\tt gcc}            & 2.95+    & \xlink{\tt \gccURL}{\gccURL}           \\
{\tt unzip}          & 5.50+    & \xlink{\tt \unzipURL}{\unzipURL}       \\
Jikes\TMweb{} Compiler & 1.18+ or 1.13 & \xlink{\tt \jikesURLHyphenated}{\jikesURL}       \\
Yacc or Bison &	{\it any} & \xlink{\tt \bisonURL}{\bisonURL} \\
\hline \multicolumn{3}{|c|}{\em If automatically Building GNU Classpath} \\ \hline
{\tt automake}       & 1.6.3+ \SeeBelow  & \xlink{\tt \automakeURL}{\automakeURL} \\
{\tt cvs}            & {\it any}    \SeeBelow  & \xlink{\tt \cvsURL}{\cvsURL} \\
{\tt wget}       & {\it any}    \SeeBelow  & \xlink{\tt \wgetURL}{\wgetURL} \\
{\tt autoconf}       & 2.53+  \SeeBelow  & \xlink{\tt \autoconfURL}{\autoconfURL} \\
\hline \multicolumn{3}{|c|}{\em Linux/IA32}                      \\ \hline
kernel         & 2.4+ \SeeBelow{} & \xlink{\tt \linuxKernelURL}{\linuxKernelURL} \\
JDK            & Blackdown 1.3.1 or 1.4.1 & \xlink{\tt \BlackdownURL}{\BlackdownURL} \\
glibc          & 2.2+ \SeeBelow & \xlink{\tt \glibcURL}{\glibcURL} \\ 
\hline \multicolumn{3}{|c|}{\em AIX\TMweb{}/PowerPC\TMweb{}}                     \\ \hline

AIX            & 4.3+     & %   \center{---}                      
\\
JDK            & IBM\Rweb{} DK 1.3.0 or 1.4.0 & \xlink{\tt \AIXJdkURL}{\AIXJdkURL} \\ 
\hline \multicolumn{3}{|c|}{\em Linux/PowerPC}                      \\ \hline
JDK            & IBM DK 1.3.0    & \xlink{\tt \linuxPPCJDKURL}{\linuxPPCJDKURL} \\
\hline\hline 
\end{tabular}
%\end{center}%and end the resizebox:
%}%ended the resizebox
\caption{System prerequisites for Jikes RVM}
\end{table}
\end{Label}

\begin{itemize}
\item {\tt automake} and {\tt autoconf} are only required \link{to build the
GNU Classpath libaries from GNU Classpath's CVS}{manual-classpath-root}.  If you instead download a
tagged pre-built GNU Classpath release (we recommend this), then you don't need
these.\texonly{  Please see section~\ref{manual-classpath-root}.}

As of this writing (October 16, 2003), the version of {\tt
  jBuildClasspathJar} available through the CVS head uses
{\tt wget} to retrieve GNU Classpath's tarball via FTP.  If this continues
to work, then the Jikes RVM build process won't require {\tt
  automake}, {\tt autoconf}, or {\tt cvs}.

\item {\tt glibc} for Linux/IA32 must use the GS register for
thread-local state.  See the build instructions for {\tt glibc} for details;
if that scares you, use a recent Red Hat or SuSE Linux distribution.

\item It is possible to use 2.2 Linux kernels with multiprocessor
support disabled in Jikes RVM.  We do not recommend this; if you
insist, look in {\tt jconfigure} for details.
\item We have also used the Sun\Rweb{} JDK 1.4.1 to build Jikes RVM.

\item We have had a problem with Bash 2.05 (not
  2.05a or 2.05b): it occasionally hangs on AIX while running
  jconfigure, because of what appears to be a bug in its handling of
  the here-document redirection operator ({\tt $<<$- EOF}).  Unfortunately, the
  version of Bash distributed with AIX 5.1 in /usr/contrib/bin is
  2.05.   Bash 2.05a, 2.05b, 2.02 and 2.03 seem to work adequately.

\end{itemize}

\subsubsection{Installation Overview}\label{sec:installDetails}

To install and build Jikes\TMweb{} RVM, two items are required
\begin{itemize}
\item The Jikes RVM source distribution.  This is available as a
compressed tar file {\tt \RVMTarFile}.  You can also work with the
contents of this repository with CVS from the 
\xlink{public repository}{\RVMCVSURL}.

\item The GNU 
\xlink{Classpath}{\classpathURL} libraries. 
\end{itemize}

Each item is distributed under a different license.  The license for
the first item is provided in Appendix~\ref{appendix:licenses}.  The
GNU Classpath license is available at \xlink{{\tt
\classpathURL}}{\classpathURL}. 

The first item is available  from the Jikes RVM
\xlink{download}{\RVMDownloadURL} page. The second item is available at
\xlink{{\tt \classpathURL}}{\classpathURL}.
You can either explicitly download the appropriate version of GNU
Classpath and set {\tt CLASSPATH\_ROOT} in your config file or let
the Jikes RVM build process download the right version for you
automatically.  (Please look at section~\ref{manual-classpath-root}
before you make your decision.)

With these files downloaded, you will set up 
a working directory holding the Jikes RVM source files, standard
library jar, and tools needed to build Jikes RVM. 

Jikes RVM can be configured in various ways. Multiple versions of the system,
corresponding to different configurations, can be generated from 
one working directory. See Section~\ref{configs} for information about the 
various 
configurations.
\index{configurations}
The Jikes RVM  {\em boot image} and other files generated during the 
configuration process
\index{boot image}
are stored in a {\em build directory}, which is logically separate from 
the working directory. 
\index{build directory}

To install Jikes RVM  you must do the following:
\begin{enumerate}
\item Set up a working directory.
\item Set various environment variables.
\item Edit Jikes RVM environment scripts.
\item Choose a configuration and run the configuration script to write
the appropriate directory and configuration specific files to the
build directory.
\item Build an executable version of Jikes RVM.
\end{enumerate}

The remainder of this section describes the process in greater detail.

\JikesTMFooter

\subsubsection{Installation Steps}

\begin{enumerate}
\item {\bf Set up a working directory.}

First extract the \jrvm{} source distribution into a
directory such as  
{\tt \$HOME/rvmRoot}.
\begin{verbatim}
% cd $HOME
% mkdir rvmRoot
% cd rvmRoot
% zcat jikesrvm-[version].tar.gz | tar xvf - 
\end{verbatim}

\index{environment variables}
\index{PATH}
\item {\bf Set up environment variables.}

You need to set up the following shell environment variables:

\begin{description}

\item[{\tt RVM\_ROOT}]  \index{RVM\_ROOT} the directory that contains
  the extracted distribution


\item[{\tt RVM\_BUILD}] \index{RVM\_BUILD} the directory where you would like the build
process to generate an executable Jikes RVM configuration

\item[{\tt RVM\_HOST\_CONFIG}] the configuration file used to specify
the software environment on which the system is generated; i.e., where the
boot image is generated.  \index{RVM\_HOST\_CONFIG}


\item[{\tt RVM\_TARGET\_CONFIG}] the configuration file used to specify
the software environment where the system support is generated; i.e., where
the ``booter'' and ``C runtime'' will be generated.  \index{RVM\_TARGET\_CONFIG}



\item[{\tt PATH}] your path should contain {\tt \$RVM\_ROOT/rvm/bin} in
order to pick up various scripts and utilities

\end{description}

We recommend you set up these variables in your shell configuration
file.  For example, for {\tt csh}, you might insert the
following into your {\tt .cshrc} file:

\begin{verbatim}
setenv RVM_ROOT $HOME/rvmRoot       # <--define your working directory 
setenv RVM_BUILD $HOME/rvmBuild     # <--define your current build directory 
setenv PATH $RVM_ROOT/rvm/bin:$PATH
setenv RVM_HOST_CONFIG $RVM_ROOT/rvm/config/powerpc-ibm-aix4.3.3.0
setenv RVM_TARGET_CONFIG $RVM_ROOT/rvm/config/powerpc-ibm-aix4.3.3.0
\end{verbatim}

{\em Note:} You should define each of these environment variables as an
{\em absolute} path.  The builder template expansion process will crash
and burn if you use a {\tt ..} in these paths.

For a Linux-Intel environment, the exports
would be replaced with the following:

\begin{verbatim}
setenv RVM_HOST_CONFIG $RVM_ROOT/rvm/config/i686-pc-linux-gnu
setenv RVM_TARGET_CONFIG $RVM_ROOT/rvm/config/i686-pc-linux-gnu
\end{verbatim}

These two variables point to the same file when the type of system  
doing the build is the same as where you are going 
the execute Jikes RVM.  To cross build a system
e.g., build on AIX\TMweb/PowerPC\TMweb\ for a
Linux/IA32 platform, see the section on Cross 
Platform Building.

\item {\bf Edit configuration scripts.}

You must edit a script in the {\tt \$RVM\_ROOT/rvm/config}  directory to set 
up variables used by the installation process.  
If someone else at your site has already installed Jikes RVM, they have
probably already done this step for you.  Consult your local Jikes RVM guru.

You must edit the file(s) that define the host and target configuration
environments in the {\tt \$RVM\_ROOT/rvm/config} directory.  
You do not need to {\em source} these variables in your working shell; 
variables in this file will be picked up by the installation scripts.  

The host and target configuration files have two sections.  In the
first section, you specify the operating system, architecture, and
whether or not the platform will support SMP-builds of Jikes RVM. 
For operating system, define one of RVM\_FOR\_LINUX or RVM\_FOR\_AIX
to be 1.  For architecture define either
RVM\_FOR\_IA32 or RVM\_FOR\_POWERPC to be 1.  For SMP status, set
RVM\_FOR\_SINGLE\_VIRTUAL\_PROCESSOR to 0 (SMP supported) or 1 (SMP not
supported).  The following are the typical settings for
RVM\_FOR\_SINGLE\_VIRTUAL\_PROCESSOR:
\begin{description}
\item[AIX/PowerPC] {\tt 0 }
\item[Linux/PowerPC] {\tt 1}
\item[Linux/IA32] {\tt 0}\footnote{You must use 1 here if you do not
have a 2.4 kernel and {\tt glibc} compiled to use the GS segment register to
access pthread-specific state.  We do not recommend such a setup.}
\end{description}                

The second section in the configuration file is used to define how to
find tools that Jikes RVM needs. You must set the following variables:

\begin{description}

\item {\tt HOST\_JAVA\_HOME} the base directory for JDK JVM.  If you
have a nonstandard JDK, you may have to define a number of variables
whose default values are defined with reference to {\tt HOST\_JAVA\_HOME}.

\index{CLASSPATH\_ROOT}
\label{manual-classpath-root}
\item {\tt CLASSPATH\_ROOT} the {\bf parent} directory of the GNU
  Classpath source code directory, if you have decided to manually
  download the Classpath libraries. 

{\bf The setting of {\tt CLASSPATH\_ROOT} is tricky}, so we will go
into some depth here.  For this example, we'll assume you downloaded
{\tt classpath-0.06.tgz}.  Create a directory to hold it.  You can
name that directory anything; we'll name it 
{\it {\tt \$RVM\_ROOT/}classpath-0.06-build}.  
A command like 
{\tt /usr/gnu/tar -C
  {\rm \it {\tt \$\{RVM\_ROOT\}/}\-classpath-0.06-build} xzf classpath-0.06.tgz} 
will create the directory
 {\rm \it {\tt \$\{RVM\_ROOT\}}/\-classpath-0.06-build}{\tt /\-classpath-0.06}.  

You {\bf must rename} {\tt \$\{RVM\_ROOT\}/{\it
    classpath-0.06-build}/\-classpath-0.06} to {\tt \$\{RVM\_ROOT\}/{\it
    classpath-0.06-build}/\-classpath}.  Now set your {\tt
  CLASSPATH\_ROOT} to {\tt \$\{RVM\_ROOT\}/\-{\it classpath-0.06-build}}
({\bf not} to {\tt \$\{RVM\_ROOT\}/{\it
    classpath-0.06-build}/\-classpath}).


The first time you run Jikes RVM's build process, it will
automatically build the appropriate files in a directory named {\tt
\$CLASSPATH\_ROOT}/{\it your-machine-type}.   You can build GNU Classpath
yourself in  {\tt \$CLASSPATH\_ROOT}/{\it your-machine-type} ahead of
time.  If you build Classpath in a differently named build directory 
(such as the source directory {\tt \$CLASSPATH\_ROOT/classpath}), the
Jikes RVM build will try to build Classpath again, not knowing that
you've already built it.

As of this writing,
building GNU Classpath from scratch takes about three minutes on an
AMD Athlon 700 processor with 512 MB of RAM.

\item {\tt GNU\_MAKE} the GNU {\tt make} executable

\item {\tt JIKES} the Jikes\TMweb{} compiler executable ({\tt jikes}).

\item {\tt CC} how to invoke the C compiler.

\item {\tt CPLUS} how to invoke the C++ compiler.

\item {\tt LDSHARED} how to link a shared C++ library.

\item {\em various basic Unix\R{} utilities} e.g., {\tt grep}, {\tt xargs}, etc.

\end{description}

\index{Contributions: Useful Thoughts}
We would like to use GNU {\tt autoconf} to automate this
step.  If you want to contribute to Jikes RVM, then this would be a
great project to choose.

\index{configurations}
\index{jconfigure script}
\item {\bf Choose a configuration and populate your build directory.}
You will use the {\tt jconfigure} script (in {\tt \$RVM\_ROOT/rvm/bin}) to
populate your build ({\tt \$RVM\_BUILD}) directory with files.  You must
first choose a Jikes RVM configuration.

For quick turnaround time while modifying Jikes RVM, you will usually
want to use the baseline compiler to build the Jikes RVM boot image.
A typical configuration that builds very quickly (but performs poorly)
is {\tt prototype}: a non-adaptive system that uses the
baseline compiler everywhere.

To obtain reasonable performance from Jikes RVM, you will need to use
the optimizing compiler to build the boot image.  This takes longer,
but generates a Jikes RVM image with reasonable runtime performance.
We recommend either the {\tt development} or {\tt production} images
depending on whether or not you want VM assertion checking enabled. (A
discussion  of Jikes RVM configurations appears in
Section~\ref{configs}.) 

Depending on your purposes you may want to choose another
configuration that uses a different memory manager (See
Section~\ref{ssec:choosinggc}.).

Run the {\tt jconfigure} script to set up the {\tt \$RVM\_BUILD}
directory for the configuration you desire.  This step creates
build scripts for your configuration and otherwise formats your
{\tt \$RVM\_BUILD} directory.
The {\tt jconfigure} script takes one argument, the name of the
configuration desired: 

\begin{verbatim}
% jconfigure <configuration>
\end{verbatim}

For example, to configure a build directory for the 
{\tt prototype} configuration, type the following command:

\begin{verbatim}
% jconfigure prototype
\end{verbatim}

\index{jbuild script}
\index{boot image}
\index{RVM\_BUILD}
\item {\bf Build an executable version of Jikes RVM.}  

Use the {\tt jbuild} script, located in the {\tt \$RVM\_BUILD} directory,
to build an executable system.  This script copies source files into
{\tt \$RVM\_BUILD/RVM.classes}, preprocesses these files, generates
some code with template expansions, builds an executable C program to
start Jikes RVM, and writes the Jikes RVM boot image.  The boot
image is the binary image of a ready-to-go instance of Jikes RVM.

The {\tt jbuild} script must be run from the {\tt \$RVM\_BUILD}
directory. It prints a copious report of its operation which you may
save for future reference by redirecting standard out and err.

\begin{verbatim}
% cd $RVM_BUILD
% jbuild
\end{verbatim}


After the {\tt jbuild} script has completed successfully you should be able 
to run Jikes RVM.  (See Section~\ref{section:running}.)

Note: The jbuild process may produce warning messages; these should not
affect system viability.

\end{enumerate}

\AIXPPCJikesTMFooter

\subsection{Jikes RVM Configurations}\label{configs}
\index{configuration names}
\index{build configurations}

This section describes Jikes\TMweb{} RVM's build configurations.
The various build configurations are defined by files in {\tt
\$RVM\_ROOT/rvm/config/build}.

\subsubsection{Logical Configurations}
\index{logical configurations}
There are a large number of possible Jikes RVM configurations.
Therefore we define four ``logical'' configurations that are most
suitable for casual or novice users of the system.  The four
configurations are: 
\begin{description}
\item[prototype] A simple, fast to build, but low performance
configuration of Jikes RVM.  This configuration does not include the
optimizing compiler or adaptive system.  Most useful for rapid
prototyping of the core virtual machine.
\item[prototype-opt] A simple, fast to build, but low performance
configuration of Jikes RVM.  Unlike prototype, this configuration does
include the optimizing compiler and adaptive system. Most useful for
rapid prototyping of the core virtual machine, adaptive system, and
optimizing compiler. 
\item[development] A fully functional configuration of
Jikes RVM with reasonable performance that includes the adaptive
system and optimizing compiler. This configuration takes longer to
build than the two prototype configurations.
\item[production] The same as the development configuration,
except all assertions are disabled.  This is the highest performance
configuration of Jikes RVM and is the one to use for benchmarking and
performance analysis. Build times are similar to the development
configuration. 
\end{description}
The mapping of logical to actual configurations may vary from release
to release.  In particular, it is expected that the choice of garbage
collector for these logical configurations may be different as JMTk
evolves. 

\subsubsection{Configurations in Depth}

Most standard Jikes RVM configuration files loosely follow the
following naming scheme:
\begin{verse}
       \Mmeta{boot image compiler} 
       \Mlbr{} \Mlitch{Base} \Mor \Mlitch{Adaptive} \Mrbr 
       \Mmeta{garbage collector}
\end{verse}

\index{boot image compiler}
\index{runtime compiler}
where
\begin{itemize}
\item the \Mmeta{boot image compiler} is the compiler used to compile
Jikes RVM's boot image. 
\item
 \Mlbr{}~\Mlitch{Base}~\Mor~\Mlitch{Adaptive}~\Mrbr denotes whether or
 not the adaptive system and optimizing compiler are included in the
 build.  
\item the {\em garbage collector} is the garbage collection scheme used.
\end{itemize}

The following garbage collection suffixes are available:

\begin{description}
\item[\Mlitch{NoGC}] no garbage collection is performed
\item[\Mlitch{SemiSpace}] a copying semi-space collector
\item[\Mlitch{MarkSweep}] a mark-and-sweep (non copying) collector
\item[\Mlitch{GenCopy}] a classic copying generational collector with a copying
  higher generation
\item[\Mlitch{GenMS}] a copying generational collector with a non-copying
  mark-and-sweep mature space
\item[\Mlitch{CopyMS}] a hybrid non-generational collector with a copying space
  (into which all allocation goes), and a non-copying space into which
  survivors go
\item[\Mlitch{RefCount}] a reference counting collector with synchronous
  (non-concurrent) cycle collection
\end{description}

For example, to specify a compiler with a baseline-compiled boot image
that will compile classes loaded at runtime using the baseline 
compiler and that uses a non-generational semi-space copying garbage
collector, use the name {\em BaseBaseSemiSpace}.

Some files augment the standard configurations as follows:
\begin{itemize}
\item The word 
\Mlitch{Full} at the beginning of the configuration name identifies a
configuration such that all the Jikes RVM classes are included in the
boot image and are compiled by the optimizing compiler.  (By default
only a small subset of these classes are included in the boot image.)
\item The word \Mlitch{Fast} at the beginning of the configuration name identifies a
Full configuration where all assertion checking has been turned off.
\end{itemize}
A boot image with either of these modifications will likely to run
faster than without, but take longer to build.

\index{adaptive configurations}
In configurations that include the adaptive system (denoted by
Adaptive in their name), methods are initially compiled by one
compiler (by default the baseline compiler) and then online profiling
is used to automatically select hot methods for recompilation by the
opt compiler at an appropriate optimization level. Further details are
provided in Section~\ref{section:aosdetails}.

For example, to configure a build directory for an adaptive
configuration, where the optimizing compiler is used to compile the
boot image and the semi-space garbage collector is used, use the
following command:

\begin{verbatim}
% jconfigure OptAdaptiveSemiSpace
\end{verbatim}

To view a list of configurations see 
{\tt \$RVM\_ROOT/rvm/config/build}.  Follow the examples in this
directory to define your own configurations with different options.  See
the {\tt jconfigure} file for a list of all options the builder
understands, or type:

\begin{verbatim}
% jconfigure --help
\end{verbatim}

\JikesTMFooter

\subsection{Cross Platform Building}

The Jikes\TMweb{} RVM build process consists of two major phases:
the building of a 
{\em boot image}, and the building of a {\em boot loader}.
The boot image is built using a Java\TMweb{} program executed within a host
JVM and is therefore platform-neutral.  By contrast, the boot loader
is written in C, and must be compiled on the target platform.

Because building the boot image can be time-consuming,
you may prefer to build the boot image
on a faster machine than the target platform.  To cross build, simply
set your 
RVM\_HOST\_CONFIG and RVM\_TARGET\_CONFIG environment variables to
be different files.

For example, to build the prototype configuration for AIX\TMweb\ 
{\em on a Linux host}:
\begin{verbatim}
% setenv RVM_ROOT $HOME/rvmRoot
% setenv RVM_BUILD $HOME/rvmBuild
% setenv PATH $RVM_ROOT/rvm/bin:$PATH
% setenv RVM_TARGET_CONFIG=$RVM_ROOT/rvm/config/powerpc-ibm-aix4.3.3.0
% setenv RVM_HOST_CONFIG=$RVM_ROOT/rvm/config/i686-pc-linux-gnu
% jconfigure prototype
% cd $RVM_BUILD
% jbuild
\end{verbatim}

This phase of the build process will complete with the words ``{\tt
  please run me on AIX}''.


The build process is then completed by building just the boot loader {\em
  on an AIX host}:

\begin{verbatim}
% setenv RVM_ROOT $HOME/rvmRoot
% setenv RVM_BUILD $HOME/rvmBuild
% setenv PATH $RVM_ROOT/rvm/bin:$PATH
% jbuild -booter
\end{verbatim}

After the {\tt jbuild -booter} script has completed successfully you should be able 
to run Jikes RVM. 

The building of the boot loader must occur in the same directory as
the rest of the build.  This can either be done transparently via a
network file system, or by copying the build directory from the first
host to the target.  Of course {\tt RVM\_ROOT}, {\tt RVM\_BUILD }
and {\tt PATH} need not be explicitly set each time: they could have
been set in your {\tt .cshrc}, {\tt .bashrc}, or other shell startup file.

More advanced users can experiment with the {\tt RVM\_BUILD\_COPY}
environment variable.  If this is set, then the 
{\tt jbuild.linkBooter} phase of the build process is replaced by the
execution of {\tt `\$RVM\_BUILD\_COPY`}.  This opens up a lot of
possibilities, such as copying the build directory to a target
machine and executing {\tt jbuild.linkBooter} remotely on the target
via {\tt rsh} or {\tt ssh}.  By setting {\tt RVM\_BUILD\_COPY}
appropriately on the host platform, cross-platform building can become
a stream-lined process.

\JikesAIXTMFooter

\JavaTMFooter

\subsection{Building Documentation}

The {\tt \RVMTarFile} file contains a PostScript version of this userguide
in {\tt \$RVM\_ROOT/rvm/doc}.  Additionally, the 
\xlink{developerWorks web page}{\RVMHomeURL} keeps an online version of
the userguide and Javadoc API, corresponding to the latest HEAD of the CVS
repository.

If you would like to recover the userguide or Javadoc for an older release
of RVM, you can rebuild the documentation locally.  See the Makefile in
{\tt \$RVM\_ROOT/rvm/doc/userguide} for rules on how to build the
HTML userguide using
\xlink{{\tt hyperlatex}}{\HyperlatexURL}.  To build the Javadoc pages, use
the {\tt jdoc.sh} script in {\tt \$RVM\_ROOT/rvm/bin}; this script takes as
its one command-line argument the directory to output the Javadoc HTML.
