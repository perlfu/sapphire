\newcommand{\AtManualClasspathRootDiscussion}[1]{at
 \link{the discussion of setting up \texttt{CLASSPATH\_ROOT}#1}[; it's in
 Section~\ref{sec:manual-classpath-root}, item~\Ref,
 sub-item~\texttt{CLASS\-PATH\_\-ROOT}, on page~\Pageref]{manual-classpath-root}.}%
This section gives instructions on how to install and run Jikes\TMweb{} RVM.\@
If your system meets certain prerequisites, the install process is
simple; if not, you may need to do preparatory work.  We
present the simple case first, then present details for installing
on other platforms.  
\link{Subsequently,
 we shall discuss further installation steps for
running Eclipse on Jikes RVM}%
[
%% NOTE: Commenting out the following line in favor of the subsequent one
%% works around an apparent bug in Hyperlatex's handling of \Ref and \Pageref.
% (Section~\Ref, page~\Pageref)%
 (Section~\ref{section:eclipse}, page~\pageref{section:eclipse})%
]{section:eclipse}.\@

\subsection{Installing on Red Hat Linux 7.3, 8.x, and 9.0, or SuSE Linux 8.x and 9.x for~Intel~x86}

Red~Hat\TMboth{} Linux\Rboth{} for Intel\Rboth{} IA-32 and AMD\Rboth{} x86-32 has emerged as the 
primary platform for Jikes\TMweb{} RVM.\@ 
%% %% No longer true for me --Steve Augart, 28 September 2004
%% SuSE\Rboth{} Linux is also in daily use
%% by one of the Jikes RVM core team members.
If you have either a developer or full install of Red~Hat
Linux~7.3 or later, your system has almost everything you need
to run Jikes RVM.\@  The following instructions describe how to install
and build Jikes RVM for this platform:
\begin{enumerate}

\index{Blackdown Java Virtual Machine}%
\index{Java Virtual Machine (JVM)}%
\index{JVM (Java Virtual Machine)}%
\item Download and install a Java\TMweb{} Virtual Machine if you have not
already done so.  The most well-tested one is the Blackdown VM,
v1.4.1, with a close runner-up being the Sun VM, v1.4.2.
For versions of Jikes RVM prior to 2.2.2, we recommend
the Blackdown VM, v1.3.1.  You can download the Blackdown VMs from 
\xlink{{\tt \BlackdownURL}}{\BlackdownURL}, or, if you have SuSE
Linux~8.2, install the RPMs {\tt Black\-downJa\-va2-JRE} and {\tt java2}. 
It is possible to use other 1.3 or 1.4 JVMs to build Jikes RVM;
however, building Jikes RVM is a good way to find bugs in a JVM.\@
Making a {\tt Full} build of Jikes RVM will make most JVMs exit abnormally.

\index{Kaffe Virtual Machine}%
Starting with Jikes RVM version 2.3.2 you can also use the Kaffe
open-source (GPL) virtual machine to boot Jikes RVM.  You can download
Kaffe from \xlink{{\tt \KaffeURL}}{\KaffeURL} or, under SuSE Linux, install the 
{\tt kaffe} RPM.

Please see 
\link{the \SectionName{%
        Bootstrapping with Kaffe} appendix}[ (Appendix~\Ref, page~\Pageref.)]{appendix:kaffe}\htmlonly{.}


\index{IBM JVM on Linux}%
JVMs that are known to fail on Linux/x86-32 include the IBM\Rweb{}~1.3.1 and
1.4.0.  People have reported some success with the IBM~1.4.1 (let us
know if you have success/problems with IBM~1.4.1). Also, please note
that Red Hat Linux 8.x does have a {\tt java} command, but that command does not invoke a
JVM and will not suffice for building Jikes RVM.\@

\index{Jikes compiler}%
\item Download and install the Jikes (Java source to Java Virtual
  Machine instruction byte code)
compiler, if you have not already 
done so (it is included in the standard SuSE Linux install, but not in
the standard Red Hat Linux install).  You can 
obtain Jikes from \xlink{the Jikes developerWorks site}{\jikesURL};
use the pre-build RPM file.
If you're using SuSE Linux versions 8.2 through 9.1, Jikes~1.18 or 1.19 is on the install
CD in the package {\tt jikes}.  SuSE Linux 9.2 comes with Jikes 1.21,
which will not work; you'll have to manually download it.  

We have had mixed experiences with different releases of the Jikes
compiler. Versions 1.13 and 1.18 are known to work for building Jikes
RVM. Versions 1.14 through 1.17, 1.20, 1.21, and 1.22 are known {\em not}\/ to
work.  Version 1.19 has worked for many people, but we have had
problems reported on OSX. 


\begin{small}
{\tt jikes-1.15} in
particular --- installed by default with SuSE Linux 8.1 --- has a
byte-code generation bug that will cause Jikes RVM to crash with an
{\tt ArrayIndexOutOfBounds} exception.  Jikes 1.21 and 1.22 dump core
with an assertion failure when they builds Jikes RVM.

{\tt jikes-1.13}, as of the last
time we tried it (September 28, 2004) works fine, and sometimes generates
better error messages than {\tt jikes-1.18} and {\tt jikes-1.19}.
\end{small}

\item Download Jikes RVM from the \xlink{Jikes RVM
download page}{\RVMDownloadURL}, and unpack it.   You can either
  download a tarball via FTP or download directly from CVS.  We shall
hereinafter call the location of the unpacked RVM distribution {\tt
  \$RVM\_\-ROOT\-/\-rvm}.   


\index{GNU Classpath: Java class libraries}%
\index{Classpath, GNU: Java class libraries}%
\item You will need version \classpathversion{} of the GNU
Classpath libraries.\footnote{0.10 and 0.11 also work; you'll have to
  set the configuration variable {\tt RVM\_WITH\_CLASSPATH\_0\_10=1} or
  {\tt RVM\_WITH\_CLASSPATH\_0\_11=1} to use them, as appropriate.}
If you're running on OSX, you will probably encounter some trouble
while building Classpath 0.12; on that platform, Classpath 0.10 builds
just fine. 

The Jikes RVM build process can automatically download, 
configure, and build the appropriate version of Classpath for you, on
platforms other than OSX.

This has historically sometimes led to confusion on the part of the
person installing Jikes RVM.\@  
%% We expect to have these glitches ironed
%% out before the next release.  
If you have any difficulty, look \AtManualClasspathRootDiscussion{}

\item Edit a configuration file so that it fits your machine.  One of
the {\tt i686-pc-linux} files in {\tt \$RVM\_\-ROOT/rvm/config} will probably be
more-or-less right to begin with.  On Linux, most standard commands
(and Jikes) are in {\tt /usr/bin}.  Hereinafter, we will call your edited
file {\tt \$RVM\_\-ROOT/rvm/config/i686-pc-linux-gnu{\it .mine}}

\index{environment variables}%
\item Setup your environment with definitions required by Jikes RVM.\@
There are four of these that you must use; in addition,
{\tt \$RVM\_\-ROOT/rvm/bin} should be in your path:

\begin{description}

\item[\varName{RVM\_\-ROOT}] is the location of the Jikes RVM distribution.  It
must be \varName{{\bf RVM\_\-ROOT}} as given above.

\item[\varName{RVM\_HOST\_CONFIG}] is the configuration file of the machine
building Jikes RVM.\@  It must be
{\tt \$RVM\_\-ROOT/rvm/config/i686-pc-linux-gnu{\it .mine}}.

\item[\varName{RVM\_TARGET\_CONFIG}] is the configuration of the
  machine running Jikes RVM.\@  It will normally be {\tt
  \$RVM\_\-ROOT/rvm/config/i686-pc-linux-gnu{\it .mine}}.  If you
  leave it unset, the build configuration program {\tt jconfigure}
  will assume it is the same as \varName{RVM\_HOST\_CONFIG}.

\item[\varName{RVM\_BUILD}] is the location where Jikes RVM is to be built.  It
can be anywhere.

\end{description}

\item Setup a build directory with ``{\tt jconfigure prototype}'', \link{or
use some other configuration if you prefer.}[  See Section~\Ref{} on page~\Pageref{}
for information about the various configurations.]{configs}

\item Build Jikes RVM: go to \varName{RVM\_BUILD} and type ``{\tt ./jbuild}''

\end{enumerate}

\newcommand{\gccURL}{ftp://ftp.gnu.org/gnu/gcc}
\newcommand{\glibcURL}{ftp://ftp.gnu.org/gnu/glibc}
\newcommand{\makeURL}{ftp://ftp.gnu.org/gnu/make}
\newcommand{\bashURL}{ftp://ftp.gnu.org/gnu/bash}
\newcommand{\bisonURL}{ftp://ftp.gnu.org/gnu/bison}
\newcommand{\tarURL}{ftp://ftp.gnu.org/gnu/tar}
\newcommand{\autoconfURL}{ftp://ftp.gnu.org/gnu/autoconf}
\newcommand{\automakeURL}{ftp://ftp.gnu.org/gnu/automake}
\newcommand{\wgetURL}{ftp://ftp.gnu.org/gnu/wget}
\newcommand{\cvsURL}{http://www.cvshome.org}
\newcommand{\linuxPPCJDKURL}{http://www.ibm.com/java/jdk/linux/index.html}
\newcommand{\linuxKernelURL}{http://www.kernel.org}

\subsection{The Hard Way}

 For any platform that is not SuSE\Rweb{} Linux\Rweb{} 8.2 or newer, or Red Hat\TMweb{} Linux 7.3 or newer for Intel\Rweb{}
IA-32/AMD\Rweb{} x86-32, installation can be more complicated.  We present a list of system
prerequisites, and then a series of install steps that assumes you
have those.

\xname{installation_Prerequisites}
\subsubsection{System Prerequisites}

\newcommand{\SeeBelow}{{\small \em (See Below)}}
\begin{Label}{prereqs}
\begin{table}[htbp]
%% This table can be way too wide to fit on a page.  If it does get to
  %% be too wide, then please uncomment the call to resizebox below,
  %% and its maching closing brace.  At that point, the page will look
  %% bad under XDVI; you'll have to look at the PostScript instead to
  %% see the effects.
%\resizebox{\linewidth}{!}{%
%\begin{center}
\label{prereq-table}\begin{tabular}{|l|l|p{.57\textwidth}|} \hline
\hline {\em software} & {\em version} & {\em site} \\ 
\hline \hline \multicolumn{3}{|c|}{\em All platforms}                     \\ \hline
GNU make       & 3.79+    & \xlink{\tt \makeURL}{\makeURL}         \\ 
%% Including the amplification below makes the table much too wide. --augart
{\tt bash} % (The Bourne-Again Shell) 
        & 2.05a+ \SeeBelow   & \xlink{\tt \bashURL}{\bashURL}         \\ 
GNU tar        & 1.13+    & \xlink{\tt \tarURL}{\tarURL}           \\ 
{\tt gcc}            & 2.95+    & \xlink{\tt \gccURL}{\gccURL}           \\
{\tt unzip}          & 5.50+    & \xlink{\tt \unzipURL}{\unzipURL}       \\
Jikes\TMweb{} Compiler & 1.19, 1.18 or 1.13 & \xlink{\tt \jikesURL}{\jikesURL}       \\
Yacc or Bison &	{\it any} & \xlink{\tt \bisonURL}{\bisonURL} \\
Xalan-Java \textit{(optional)} & {\it any}    \link{\SeeBelow}{xalan-why}  & \xlink{\tt \xalanURL}{\xalanURL} \\
\hline
\hline \multicolumn{3}{|c|}{\em If \link{automatically Building GNU
        Classpath via {\tt wget}}{wget-why} (the default)} \\ \hline

{\tt wget} & {\it any}    \link{\SeeBelow}{wget-why}  & \xlink{\tt \wgetURL}{\wgetURL} \\

%% \hline \hline \multicolumn{3}{|c|}{\em If automatically Building GNU
%%         Classpath from CVS (the default until October~2003)} \\ \hline
%% {\tt automake}       & 1.6.3+ \link{\SeeBelow}{auto-and-cvs}  & \xlink{\tt \automakeURL}{\automakeURL} \\
%% {\tt cvs}            & {\it any}    \link{\SeeBelow}{auto-and-cvs}  & \xlink{\tt \cvsURL}{\cvsURL} \\
%% {\tt autoconf}       & 2.53+  \link{\SeeBelow}{auto-and-cvs}  & \xlink{\tt \autoconfURL}{\autoconfURL} \\
\hline \hline \multicolumn{3}{|c|}{\em Linux/x86-32}                      \\ \hline
kernel         & 2.4+ \SeeBelow{} & \xlink{\tt \linuxKernelURL}{\linuxKernelURL} \\
JDK            & Blackdown 1.3.1 or 1.4.1 or 1.4.2 & \xlink{\tt \BlackdownURL}{\BlackdownURL} \\
glibc          & 2.2+ \SeeBelow & \xlink{\tt \glibcURL}{\glibcURL} \\ 
\hline \hline \multicolumn{3}{|c|}{\em AIX\TMweb{}/PowerPC\TMweb{}}                     \\ \hline

AIX            & 4.3+     & %   \center{---}                      
\\
JDK            & IBM\Rweb{} DK 1.3.0 or 1.4.0 or 1.4.2 & \xlink{\tt \AIXJdkURL}{\AIXJdkURL} \\ 
\hline \hline \multicolumn{3}{|c|}{\em Linux/PowerPC}                      \\ \hline
JDK            & IBM DK 1.3.0    & \xlink{\tt \linuxPPCJDKURL}{\linuxPPCJDKURL} \\
\hline \hline \multicolumn{3}{|c|}{\em Mac OS X}                     \\ \hline

\texttt{dlcompat} library & 2003-06-29 or later
& \xlink{\texttt{http://www.o\-pendar\-win.org/\-pro\-jects/\-dl\-compat/}}{http://www.opendarwin.org/projects/dlcompat/} \\
\texttt{August2003gccUpdater.dmg} & \textit{n/a} & Apple Developer Connection web site        \\
\texttt{Java141Developer.dmg} & \textit{n/a} & Apple Developer Connection web site    \\
\texttt{CHUD\_3.0.1.dmg} & \textit{n/a} & Apple Developer Connection web site \\
\texttt{Dec2002DevToolsCD.dmg} & \textit{n/a} & Apple Developer Connection web site \\

% \hline

\multicolumn{3}{|c|}{
\texonly{\hrulefill}  \link*{Please also see \SectionName{OSX
installation} below}[\textit{Mac OS~X is further discussed in subsection~\Ref{}, page~\Pageref{}, below \texonly{\hrulefill} }]{OSX-install}} 
\\
\hline\hline 
\end{tabular}
%\end{center}%and end the resizebox:
%}%ended the resizebox
\caption{System Prerequisites for Jikes RVM}
\end{table}
\end{Label}

\T The notes in this subsection are annotations to
\T Table~\ref{prereq-table}, \SectionName{System Prerequisites for
\T Jikes RVM}, on page~\pageref{prereq-table}:
\T
\begin{itemize}
%% \begin{Label}{auto-and-cvs}
%% \item \texttt{automake}, \texttt{autoconf}, and \texttt{cvs}
%% are only required \link{to build the GNU Classpath libaries 
%%         from GNU Classpath's CVS}{manual-classpath-root}.  This is an
%% obscure option, intended mainly for classpath hackers.
%% If you instead \link{download a GNU Classpath tarball release}{manual-classpath-root}
%% via FTP (we recommend this), or let the \jrvm{} build process retrieve and build
%% GNU Classpath for you, then you don't need
%% these.\texonly{  Look \AtManualClasspathRootDiscussion{ before
%%  you make your decision}}
%% \end{Label}

\begin{Label}{wget-why}
%% As of this writing (October 16, 2003), the version of 
\item {\tt jBuildClasspathJar} %% available through the CVS head now 
uses \texttt{wget} to retrieve GNU Classpath's tarball via FTP.\@   If
you \link{retrieve Classpath yourself}{manual-classpath-root} then you
won't need \texttt{wget}.

%% If this continues
%% to work, then the Jikes RVM build process won't require {\tt
%%   automake}, \texttt{autoconf}, or \texttt{cvs}.
\end{Label}

\begin{Label}{xalan-why}
%% As of this writing (October 16, 2003), the version of 
\item The Xalan-Java XSLT processor and the Xerces XML processor are an
optional
part of the Jikes RVM run-time library.  If you are using any programs
that use XML,
then we recommend adding them, since the XML support in Classpath is
not mature.  They are also necessary if you wish to run Eclipse.  
We have tested using Xalan-Java 2.6.0 (the latest version as of this
writing); visit \xlink{\texttt{\xalanURL}}{\xalanURL} and download the ``Xalan
Java binary distribution'', which has Xerces packaged with it.  The
file to download is \texttt{\xalanFile}.   You can just unpack it
without having to build it. 
\end{Label}

\item \texttt{glibc} for Linux/x86 must use the GS register for
thread-local state.  
% See the build instructions for \texttt{glibc} for details;
You can build \texttt{glibc} yourself, being sure to specify the flag
``\texttt{--enable-kernel=2.4.0}'';
if that scares you, use a recent Red Hat or SuSE Linux distribution,
or use Debian/Sarge with a 2.6 kernel.  More information about why
this is important and how to check on this is
\link{in the FAQ}[.  See section~\Ref on Page~\Pageref]{singleProcessorQuestion}.

\item It is possible to use 2.2 Linux kernels with multiprocessor
support disabled in Jikes RVM.\@  We do not recommend this; if you
insist, look in \texttt{jconfigure} for details.
\item We have also used the Sun\Rweb{} JDK 1.4.1 to build Jikes RVM.\@

\item We have had a problem with Bash 2.05 (not
  2.05a or 2.05b): it occasionally hangs on AIX while running
  \texttt{jconfigure}, because of what appears to be a bug in its handling of
  the here-document redirection operator (\texttt{$<<$- EOF}).  Unfortunately, the
  version of Bash distributed with AIX 5.1 in /usr/contrib/bin is
  2.05.   Bash 2.05a, and 2.05b seem to work adequately.  
  \texttt{jconfigure} will test for a working version of Bash and warn
  you of problems.

\end{itemize}

\subsubsection{Installation Overview}\label{sec:installDetails}

To install and build Jikes\TMweb{} RVM, two items are required
\begin{itemize}
\item The Jikes RVM source distribution.  This is available as a
compressed tar file \texttt{\RVMTarFile}.  You can also work with the
contents of this repository with CVS from the 
\xlink{public repository}{\RVMCVSURL}.

\item The GNU 
\xlink{Classpath}{\classpathURL} libraries. 
\end{itemize}

Each item is distributed under a different license.  The license for
the first item is \link{provided in the \SectionName{%
        Jikes RVM Licenses} appendix}[ (Appendix~\Ref, page~\Pageref.)]{appendix:licenses}\htmlonly{.}
The GNU Classpath license is available at 
\xlink{\texttt{\classpathURL}}{\classpathURL}. 

The first item is available  from the \xlink{Jikes RVM
download page}{\RVMDownloadURL}. The second item is
available \xlink{from the GNU Classpath web site}[, at \texttt{\classpathURL}]{\classpathURL}.
You can either explicitly download the appropriate version of GNU
Classpath and set \texttt{CLASSPATH\_ROOT} in your config file or let
the Jikes RVM build process download the right version for you
automatically.  Please look \AtManualClasspathRootDiscussion{}

With these files downloaded, you will set up 
a working directory holding: the Jikes RVM source files; the 
library sources; and the tools needed to build Jikes RVM.\@ 

\index{configurations}%
Jikes RVM can be configured in various ways.  Multiple versions of the system,
corresponding to \link{different configurations}{configs}, can be generated from 
one working directory.   \texonly{See Section~\ref{configs} (page~\pageref{configs}) for information about the various configurations.}

The Jikes RVM {\em boot image\index{boot image}} and other files generated during the 
configuration process
are stored in a {\em build directory\index{build directory}},
 which is logically separate from the working directory.   Each
distinct configuration of \jrvm{} you build will have its own 
{\em build directory}. 

To install Jikes RVM  you must do the following:
\begin{enumerate}
\item Set up a working directory.
\item Set various environment variables.
\item Edit Jikes RVM environment scripts.
\item Choose a configuration and run the configuration script to write
the appropriate directory and configuration specific files to the
build directory.
\item Build an executable version of Jikes RVM.\@
\end{enumerate}

The remainder of this section describes the process in greater detail.

\subsubsection{Installation Steps}%
\label{sub-sub-sec:InstallationSteps}%
\label{sec:manual-classpath-root}%
\label{single-virtual-processor-subsubsection}

\begin{enumerate}
\item {\bf Set up a working directory.}

First extract the \jrvm{} source distribution into a
directory such as  
\texttt{\$HOME/rvmRoot}:
\begin{verbatim}
% cd ; mkdir rvmRoot
% tar -C rvmRoot zxvf jikesrvm-[version].tar.gz 
\end{verbatim}

\index{environment variables}%
\item {\bf Set up environment variables.}

You need to set up the following shell environment variables:

\begin{description}

\item[\IndexTexttt{RVM\_\-ROOT}] the directory that contains
  the extracted distribution

\item[\IndexTexttt{RVM\_BUILD}]
the directory where you would like the build
process to generate an executable Jikes RVM configuration

\item[\IndexTexttt{RVM\_HOST\_CONFIG}]
the configuration file used to specify
the software environment on which the system is generated; {\it i.e.}, where the
boot image is generated.  

\item[\IndexTexttt{RVM\_TARGET\_CONFIG}] the
configuration file used to specify the software environment where the
system support is generated; {\it i.e.}, where the ``booter'' and ``C
runtime'' will be generated. 

\item[\IndexTexttt{PATH}] your path should contain \texttt{\$RVM\_\-ROOT/\-rvm/\-bin} in
order to pick up various scripts and utilities

\end{description}

We recommend you set up these variables in your shell configuration
file.  For example, for \texttt{csh}, you might insert the
following into your \texttt{.cshrc} file:

\begin{verbatim}
setenv RVM_ROOT $HOME/rvmRoot       # <--define your working directory 
setenv RVM_BUILD $HOME/rvmBuild     # <--define your current build directory 
setenv PATH $RVM_ROOT/rvm/bin:$PATH
setenv RVM_HOST_CONFIG $RVM_ROOT/rvm/config/powerpc-ibm-aix4.3.3.0
setenv RVM_TARGET_CONFIG $RVM_ROOT/rvm/config/powerpc-ibm-aix4.3.3.0
\end{verbatim}

{\em Note:} You should define each of these environment variables as an
{\em absolute} path.  The builder template expansion process will crash
and burn if you use a \texttt{..} in these paths.

For a Linux\Rweb{}-Intel\Rweb{} environment, the exports
would be replaced with the following:

\begin{verbatim}
setenv RVM_HOST_CONFIG $RVM_ROOT/rvm/config/i686-pc-linux-gnu
setenv RVM_TARGET_CONFIG $RVM_ROOT/rvm/config/i686-pc-linux-gnu
\end{verbatim}

These two variables point to the same file when the type of system  
doing the build is the same as where you are going 
the execute Jikes\TMweb{} RVM.\@  To cross build a system
({\it e.g.}, to build on Linux/x86 for an AIX\TMweb{}/PowerPC\TMweb{}
platform), see \link{\SectionName{Cross Platform Building}}[
  (sub-section~\Ref, page~\Pageref)]{cross-platform-building}.

\item {\bf Edit configuration scripts.}

You must edit a script in the \texttt{\$RVM\_\-ROOT/rvm/config}
directory to set  
up variables used by the installation process.  
If someone else at your site has already installed Jikes RVM, they have
probably already done this step for you.  Consult your local Jikes RVM guru.

You must edit the file(s) that define the host and target configuration
environments in the \texttt{\$RVM\_\-ROOT/rvm/config} directory.  
You do not need to {\em source} these variables in your working shell; 
variables in this file will be picked up by the installation scripts.  

\begin{Label}{single-virtual-processor-item}
The host and target configuration files have two sections.  In the
first section, you specify the operating system, architecture, and
whether or not the platform will support SMP-builds of Jikes RVM.\@ 
For operating system, define one of \varName{RVM\_\-FOR\_\-LI\-NUX}, 
\varName{RVM\_\-FOR\_\-OSX},
or \varName{RVM\_\-FOR\_\-AIX} 
to be 1.  For architecture define either
\varName{RVM\_FOR\_IA32} or \varName{RVM\_\-FOR\_\-POWER\-PC} to be 1.  For SMP status, set
\varName{RVM\_\-FOR\_\-SIN\-GLE\_VIR\-TU\-AL\_PRO\-CES\-SOR} to 0 (SMP supported) or 1 (SMP not
supported).  The following are the typical settings for
\varName{RVM\_\-FOR\_\-SIN\-GLE\_\-VIR\-TU\-AL\_\-PRO\-CES\-SOR}:

\begin{description}
\item[AIX/PowerPC] \texttt{0 }
\item[Linux/PowerPC] \texttt{1}
\item[OS~X/PowerPC] \texttt{1}
\item[Linux/x86-32] \texttt{0}\footnote{You must use 1 here if you do not
have a 2.4 kernel and \texttt{glibc} compiled to use the GS segment register to
access pthread-specific state.  We do not recommend such a setup.}
\end{description}                
\end{Label}

The second section in the configuration file is used to define how to
find tools that Jikes RVM needs. You must set the following variables:

\begin{description}

\item[\texttt{HOST\_JAVA\_HOME}] the base directory for JDK JVM.\@  If you
have a nonstandard JDK, you may have to define a number of variables
whose default values are defined with reference to \texttt{HOST\_JAVA\_HOME}.

\label{manual-classpath-root}%
\item[\IndexTexttt{CLASSPATH\_ROOT}] the {\bf parent} directory of the GNU
  Classpath source code directory, if you have decided to manually
  download the Classpath libraries. 

{\bf The setting of \texttt{CLASSPATH\_ROOT} is tricky}, so we will go
into some depth here.  For this example, we'll assume you downloaded
\texttt{classpath-0.12.tgz}.  Create a directory to hold it.  You can
name that directory anything; we'll name it 
{\it \texttt{\$RVM\_\-ROOT/}classpath-0.12}.  
A command like 
``\texttt{/usr/gnu/tar -C
  {\rm \it \texttt{\$\{RVM\_\-ROOT\}/}\-classpath-0.12-build} xzf classpath-0.12.tgz}''
will create the directory
``{\rm \it \texttt{\$\{RVM\_\-ROOT\}}/\-class\-path-0.12-build}\texttt{/\-class\-path-0.12}''.  

You {\bf must rename} ``\texttt{\$\{RVM\_\-ROOT\}/{\it
    class\-path-0.12-build}/\-class\-path-0.12}'' to ``\texttt{\$\{RVM\_\-ROOT\}/\-{\it
    class\-path-0.12-build}/\-class\-path}''.  Now set your {\tt
  CLASS\-PATH\_\-ROOT} to ``\texttt{\$\{RVM\_\-ROOT\}/\-{\it class\-path-0.12-build}}''
({\bf not} to \texttt{\$\{RVM\_\-ROOT\}/\-{\it
    class\-path-0.12-build}/\-class\-path}).


The first time you run Jikes RVM's build process, it will
automatically build the appropriate files in a directory named {\tt
\$CLASSPATH\_ROOT}/{\it your-machine-type}.   You can build GNU Classpath
yourself in  \texttt{\$CLASSPATH\_ROOT}/{\it your-machine-type} ahead of
time.  If you build Classpath in a differently named build directory 
(such as the source directory \texttt{\$CLASSPATH\_ROOT/classpath}), the
Jikes RVM build will try to build Classpath again, not knowing that
you've already built it.

As of this writing,
building GNU Classpath from scratch takes about three minutes on an
AMD Athlon 700 processor with 512 MB of RAM.\@

\item[\texttt{GNU\_MAKE}] the GNU \texttt{make} executable

\item[\texttt{JIKES}] the Jikes\TMweb{} compiler executable (\texttt{jikes}).

\item[\texttt{CC}] how to invoke the C compiler.

\item[\texttt{CPLUS}] how to invoke the C++ compiler.

\item[\texttt{LDSHARED}] how to link a shared C++ library.

\item[{\em various basic Unix\Rboth{} utilities}]
for example, \texttt{grep}, \texttt{xargs}, and so on.

\end{description}

\index{contributions: suggestions: \texttt{autoconf}}%
We would like to use GNU \IndexTexttt{autoconf} to automate this
step.  If you want to contribute to Jikes RVM, then this would be a
great project to choose.

\index{configurations}%
\cindex[jconfigure script]{\texttt{jconfigure} script}%
\item {\bf Choose a configuration and populate your build directory.}
You will use the \texttt{j\-con\-fi\-gure} script (in \texttt{\$RVM\_\-ROOT/rvm/bin}) to
populate your build (\texttt{\$RVM\_BUILD}) directory with files.  You must
first choose a Jikes RVM configuration.

\cindex[prototype configuration]{\texttt{prototype} configuration}%
For quick turnaround time while modifying Jikes RVM, you will usually
want to use the baseline compiler to build the Jikes RVM boot image.
A typical configuration that builds very quickly (but performs poorly)
is \texttt{prototype}: a non-adaptive system that uses the
baseline compiler everywhere.

\cindex[production configuration]{\texttt{production} configuration}%
\cindex[development configuration]{\texttt{development} configuration}%
To obtain reasonable performance from Jikes RVM, you will need to use
the optimizing compiler to build the boot image.  This takes longer,
but generates a Jikes RVM image with reasonable runtime performance.
We recommend either the \texttt{development} or \texttt{production}
configurations, 
depending on whether or not you want VM assertion checking enabled.  (A
\link{discussion of Jikes RVM configurations appears \texorhtml{in
Section~\Ref, on page~\Pageref}{elsewhere}}{configs}.) 

Depending on your purposes, you may want to \link{choose another
configuration that uses a different memory manager}[ (See
Section~\Ref{}, page~\Pageref.)]{ssec:choosinggc}\htmlonly{.}

Run the \texttt{jconfigure} script to set up the \texttt{\$RVM\_BUILD}
directory for the configuration you desire.  This step creates
build scripts for your configuration and otherwise formats your
\texttt{\$RVM\_BUILD} directory.
The \texttt{jconfigure} script takes one argument, the name of the
configuration desired: 

\begin{verbatim}
% jconfigure <configuration>
\end{verbatim}

For example, to configure a build directory for the 
\texttt{prototype} configuration, type the following command:

\begin{verbatim}
% jconfigure prototype
\end{verbatim}

\cindex[jbuild script]{\texttt{jbuild} script}%
\index{boot image}%
\Indextt{RVM\_BUILD}%
\item {\bf Build an executable version of Jikes RVM.}  

Use the \texttt{jbuild} script, located in the \texttt{\$RVM\_BUILD} directory,
to build an executable system.  This script copies source files into
\texttt{\$RVM\_BUILD/RVM.classes}, preprocesses these files, generates
some code with template expansions, builds an executable C program to
start Jikes RVM, and writes the Jikes RVM boot image.  The boot
image is the binary image of a ready-to-go instance of Jikes RVM.\@

The \texttt{jbuild} script must be run from the \texttt{\$RVM\_BUILD}
directory.  When invoked with the \texttt{-trace} flag,
\texttt{jbuild} prints a copious report of its operation, which you may 
save for future reference by redirecting the standard output and
standard error streams.

\begin{verbatim}
% cd $RVM_BUILD
% jbuild
\end{verbatim}


After the \texttt{jbuild} script has completed successfully \link{you should be able 
to run Jikes RVM.}[  (See Section~\Ref{}, page~\Pageref{}.)]{section:running}

Note: The \texttt{jbuild} process may produce warning messages; these should not
affect system viability.  In case of any errors that affect system
viability, \texttt{jbuild} will exit with non-zero (error) status.

\end{enumerate}

\subsubsection{Mac OS X}\label{OSX-install}

\begin{enumerate}

\item As of OS~X~10.3, Apple's developer tools include the Jikes
compiler; if you have the developer tools installed on OS~X~10.3, you
can try building without installing any special libraries etc.\ at
all; it will probably work just fine. You will still need to install
\texttt{wget} if you want automatically retrieve GNU Classpath. If GNU
Classpath fails to compile, see below for more information.

\item
If you want to build Jikes RVM for Mac OS~X, you will need to patch
your copy of GNU Classpath with the patch file located
in \texttt{\$RVM\_\-ROOT/\-rvm/\-etc/\-gnu-classpath-on-mac-osx.patch}.

\item Classpath 0.12 does not seem to build on OSX; Classpath 0.10
builds just fine.

\item
One of the \xlink{\texttt{jikesrvm-researchers} readers contributed
this information on how to install Jikes RVM on a clean OS~X~10.2
box}{\RVMResearcherMailingListArchiveURL/2003-October/002367.html}:
\begin{quote}
\raggedright
I first installed the following: \\
\OneCMStrut{}\texttt{August2003gccUpdater.dmg  Java141Developer.dmg jikes-1.18.tar.bz2 \\
\OneCMStrut{}CHUD\_3.0.1.dmg Dec2002DevToolsCD.dmg     dlcompat-20030629.tar.gz \\
}

The \texttt{*.dmg} files are freely available through the Apple
Developer Connection 
website, and the versions of \texttt{jikes} and \texttt{dlcompat} I grabbed from their
respective websites and installed from source.
\end{quote}

\item
If your Mac OS~X build complains about a missing file  \texttt{dlfcn.h}, then 
you need to have the \texttt{dlcompat} library installed, available from 
\xlink{\texttt{http://www.o\-pendar\-win.org/\-pro\-jects/\-dl\-compat/}}{http://www.opendarwin.org/projects/dlcompat/}
---~the 2003-06-29 release of \texttt{dlcompat} works, according to a posting to
the \texttt{jikesrvm-researchers} mailing list dated 2003-10-26.

\item
If you want to \texttt{make tags} using
\texttt{\$RVM\_ROOT/rvm/GNUmakefile}, you will need \texttt{xargs}
from GNU findutils.

\item
If you want to \texttt{make conditionals} using
\texttt{\$RVM\_ROOT/rvm/GNUmakefile}, you will need GNU sed.

\item
Check the \link{list of OS~X prerequisites}[ on page~\Pageref]{prereqs} for additional
information. 

\end{enumerate}


\subsection{Jikes RVM Configurations}\label{configs}%
\index{configuration names}%
\index{build configurations}

This section describes Jikes\TMweb{} RVM's build configurations.
The various build configurations are defined by files in ``{\tt
\$RVM\-\_\-ROOT\-/\-rvm\-/\-con\-fig\-/\-build}''.

\subsubsection{Logical Configurations}%
\newcommand{\configName}[1]{\texttt{#1}}%
\index{logical configurations}

There are many possible Jikes RVM configurations.
Therefore, we define four ``logical'' configurations that are most
suitable for casual or novice users of the system.  The four
configurations are: 
\begin{description}

\item[prototype] A simple, fast to build, but low performance
configuration of Jikes RVM.\@  This configuration does not include the
optimizing compiler or adaptive system.  Most useful for rapid
prototyping of the core virtual machine.

\item[prototype-opt] A simple, fast to build, but low performance
configuration of Jikes RVM.\@  Unlike \configName{prototype}, this
configuration does include the optimizing compiler and adaptive system. Most useful for
rapid prototyping of the core virtual machine, adaptive system, and
optimizing compiler. 

\item[development] A fully functional configuration of
Jikes RVM with reasonable performance that includes the adaptive
system and optimizing compiler.  This configuration takes longer to
build than the two \configName{prototype} configurations.

\item[production] The same as the \configName{development} configuration,
except all assertions are disabled.  This is the highest performance
configuration of Jikes RVM and is the one to use for benchmarking and
performance analysis. Build times are similar to the \configName{development} 
configuration. 

\end{description}

The mapping of logical to actual configurations may vary from release
to release.  In particular, it is expected that the choice of garbage
collector for these logical configurations may be different as JMTk
evolves. 

\subsubsection{Configurations in Depth}

Most standard Jikes RVM configuration files loosely follow the
following naming scheme:
\begin{verse}
       \Mmeta{boot image compiler} 
       \Mlbr{} \Mlitch{Base} \Mor{} \Mlitch{Adaptive} \Mrbr{}
       \Mmeta{garbage collector}
\end{verse}

\index{boot image compiler}%
\index{runtime compiler}%
where
\begin{itemize}
\item the \Mmeta{boot image compiler} is the compiler used to compile
Jikes RVM's boot image. 
\item
 \Mlbr{}~\Mlitch{Base}~\Mor~\Mlitch{Adaptive}~\Mrbr{} denotes whether or
 not the adaptive system and optimizing compiler are included in the
 build.  
\item the {\em garbage collector} is the garbage collection scheme used.
\end{itemize}

The following garbage collection suffixes are available:

\begin{description}
\item[\Mlitch{NoGC}] no garbage collection is performed
\item[\Mlitch{SemiSpace}] a copying semi-space collector
\item[\Mlitch{MarkSweep}] a mark-and-sweep (non copying) collector
\item[\Mlitch{GenCopy}] a classic copying generational collector with a copying
  higher generation
\item[\Mlitch{GenMS}] a copying generational collector with a non-copying
  mark-and-sweep mature space
\item[\Mlitch{CopyMS}] a hybrid non-generational collector with a copying space
  (into which all allocation goes), and a non-copying space into which
  survivors go
\item[\Mlitch{RefCount}] a reference counting collector with synchronous
  (non-concurrent) cycle collection
\end{description}

For example, to specify a Jikes RVM configuration: 
\begin{enumerate}
\item with a baseline-compiled boot image,
\item that will compile classes loaded at runtime using the baseline 
compiler and
\item that uses a non-generational semi-space copying garbage
collector, 
\end{enumerate}
use the name \Mlitch{BaseBaseSemiSpace}.

Some files augment the standard configurations as follows:

\begin{itemize}

\item The word 
\Mlitch{Full} at the beginning of the configuration name identifies a
configuration such that all the Jikes RVM classes are included in the
boot image.  (By default 
only a small subset of these classes are included in the boot image.)

\Mlitch{FullAdaptive} images 
have all of the included classes already compiled by the optimizing
compiler.  \Mlitch{FullBaseAdaptive} images have the included classes
already compiled by the baseline compiler; the adaptive system will
later recompile any hot methods.

\item The word \Mlitch{Fast} at the beginning of 
the configuration name identifies a \Mlitch{Full} configuration where
all assertion checking has been turned off.

\end{itemize}

\Mlitch{Full} and \Mlitch{Fast} boot images run faster but take
longer to build.

\index{adaptive configurations}%
In configurations that include the adaptive system (denoted by
\Mlitch{Adaptive} in their name), methods are initially compiled by one
compiler (by default the baseline compiler) and then online profiling
is used to automatically select hot methods for recompilation by the
optimizing compiler at an appropriate optimization level.  \link{Further details are
provided in the \SectionName{Adaptive Optimization System} section}[ (Section~\Ref, page~\Pageref.)]{section:aosdetails}\texonly{.}

For example, to configure a build directory for an adaptive
configuration, where the optimizing compiler is used to compile the
boot image and the semi-space garbage collector is used, use the
following command:

\begin{verbatim}
% jconfigure OptAdaptiveSemiSpace
\end{verbatim}

To view a list of configurations, see the directory
``\texttt{\$RVM\_\-ROOT\-/\-rvm\-/\-con\-fig\-/\-build}''.  Follow the
examples there to define your own configurations with different options. 
For a list of all options the builder understands, type:
\begin{verbatim}
% jconfigure --help
\end{verbatim}

\subsubsection{Configurations in Painful Depth}
To go into painful depth, except for the ``logical'' configurations
(\texttt{development}, 
\texttt{prototype}, \texttt{production}, \texttt{prototype-opt}), and
except for a couple of historical ones, standard Jikes\TMweb{} RVM configuration
files follow the following naming scheme:
\begin{verse}
       \MZeroOrOne{\Mlitch{ExtremeAssertions}} \MZeroOrOne{\Mlitch{Fast} \MZeroOrOne{\Mlitch{Opt} \MZeroOrOne{\Mlitch{0}\Mor\Mlitch{1}}} \Mor{} \Mlitch{Full}}\\
 \Mmeta{boot image compiler} \Mlbr{} \Mlitch{Base} \Mor{}
 \Mlitch{Adaptive} \Mrbr{} \\
       \MZeroOrOne{\Mlitch{OTH}} \Mmeta{garbage collector} \MZeroOrOne{\Mlitch{UP}}
\end{verse}

\begin{itemize}
\item \Mlitch{OTH} means that a configuration uses the
        uses the \link{\IndexTexttt{OptTestHarness}}{opttestharness}.  
        \link{\IndexTexttt{OptTestHarness}}{opttestharness}.  

\item The \Mlitch{ExtremeAssertions} configurations have the
          \IndexTexttt{RVM\-\_\-WITH\-\_\-EXTREME\-\_\-ASSERTIONS} flag
          enabled.  This turns on a number of expensive assertions.

\item \Mlitch{UP} means that a configuration has the
          \IndexTexttt{RVM\_FOR\_SINGLE\_VIRTUAL\_PROCESSOR} flag set.


\end{itemize}

You can specify additional configuration variables by:
\begin{enumerate}
\item writing your own
configuration files
\item giving command-line options to
\IndexTexttt{jconfigure} of the form
``\texttt{-D}\Mmeta{config-variable}\texttt{=1}'', or
\item adding the variable assignments to your \varName{RVM\_TARGET\_CONFIG} file.
\end{enumerate}
The more useful ones include:
\begin{description}
\item[\IndexTexttt{RVM\_WITH\_DETERMINISTIC\_THREAD\_SWITCHING}]
Thread switch after $n$ method calls; do not use interrupts to drive
preemption.

\item[\IndexTexttt{RVM\_WITHOUT\_PREEMPTIVE\_THREAD\_SWITCHING}]
  When set to 1, switch between threads only at explicit yield and wait calls.

\item[\IndexTexttt{RVM\_WITH\_ALL\_CLASSES}]  When set to 1, include
  all VM classes in the bootimage, not just those needed for startup.  
  The \Mlitch{Fast} and \Mlitch{Full} configurations set this flag.

\item[\IndexTexttt{RVM\_HPM\_DIR}]  Used for PowerPC Hardware%
  \index{HPM: Hardware Performance Monitors}
  Performance Monitor (HPM) support.  Specifies where the HPM
  libraries are located; they're needed at link time.

\item[\IndexTexttt{RVM\_WITH\_HPM}]%
  \index{HPM: Hardware Performance Monitors}
  A conditional compilation directive.  When set to 1, include the HPM
  (Hardware Performance Monitor) support code.  PowerPC only.

\item[\IndexTexttt{RVM\_WITHOUT\_ASSERTIONS}]
 When set to 1, exclude runtime assertion checks.  The
 \Mlitch{Fast} configurations set this option.

\item[\IndexTexttt{RVM\_WITHOUT\_AOS\_LOG}] When set to 1, turn off \link{AOS
  logging}[ (discussed in Section~\Ref, on page~\Pageref.)]{aos-logging}\texonly{.}

\item[\IndexTexttt{RVM\_FOR\_SINGLE\_VIRTUAL\_PROCESSOR}]
Discussed: \link{above}[
  (SubSubSection~\ref{single-virtual-processor-subsubsection},
  item~\Ref, on page~\Pageref)]{single-virtual-processor-item}; in the
\link{\SectionName{Threads} section}[ (Section~\Ref, page~\Pageref)]{threads:single-virtual-processor};
and \link{in a \SectionName{FAQ} question}[ (Section~\Ref, page~\Pageref)]{singleProcessorQuestion}.

\end{description}

All of the available configuration options are listed in the source
file \texttt{\$RVM\_ROOT\-/bin\/\-j\-con\-fi\-gure}.

\subsection{Cross-Platform Building}\label{cross-platform-building}

The Jikes\TMweb{} RVM build process consists of two major phases:
the building of a 
{\em boot image}, and the building of a {\em boot loader}.
The boot image is built using a Java\TMweb{} program executed within a host
JVM and is therefore platform-neutral.  By contrast, the boot loader
is written in C, and must be compiled on the target
platform.\footnote{True, one might use a cross-compiler to build the
  boot loader on platform \textit{X} for target platform \textit{Y}.
  The Jikes RVM system isn't currently set up for such use because
  nobody has needed to do this yet who cared enough to make the changes.  It could easily be
  modified to do so.}

Because building the boot image can be time-consuming,
you may prefer to build the boot image
on a faster machine than the target platform.  You may also be
porting Jikes RVM to a target platform that lacks tools such as a Java
Development Kit, or whose development environment is otherwise unpleasant. 
To cross-build, simply set your 
\varName{RVM\_HOST\_CONFIG} and \varName{RVM\_TARGET\_CONFIG} environment variables to
be different files.

For example, to build the \texttt{prototype} configuration for AIX\TMweb{} 
{\em on a Linux host:}
\begin{verbatim}
% setenv RVM_ROOT $HOME/rvmRoot
% setenv RVM_BUILD $HOME/rvmBuild
% setenv PATH $RVM_ROOT/rvm/bin:$PATH
% setenv RVM_TARGET_CONFIG=$RVM_ROOT/rvm/config/powerpc-ibm-aix4.3.3.0
% setenv RVM_HOST_CONFIG=$RVM_ROOT/rvm/config/i686-pc-linux-gnu
% jconfigure prototype
% cd $RVM_BUILD
% ./jbuild
\end{verbatim}

This phase of the build process will complete with the words ``{\tt
  please run me on AIX}''.


The build process is then completed by building just the boot loader {\em
  on an AIX host}:

\begin{verbatim}
% setenv RVM_ROOT $HOME/rvmRoot
% setenv RVM_BUILD $HOME/rvmBuild
% setenv PATH $RVM_ROOT/rvm/bin:$PATH
% cd $RVM_BUILD
% ./jbuild -booter
\end{verbatim}

\varName{RVM\_ROOT} and \varName{RVM\_BUILD} do not have to have the
same values on the target that they do on the host.   There's no
problem if (for example) the host refers to a directory as
\texttt{/nfs/\-tar\-get\-sys/\-home/\-bar} and the target refers to the same
directory as \texttt{/home/\-bar}.

After the \texttt{jbuild -booter} script has completed successfully, you should be able 
to run Jikes RVM.\@ 

The building of the boot loader must occur in the same directory as
the rest of the build.  This can either be done transparently via a
network file system, or by copying the build directory from the first
host to the target.  Of course, \texttt{RVM\_\-ROOT}, \texttt{RVM\-\_\-BUILD},
and \texttt{PATH} need not be explicitly set each time; they could
have been set in your \texttt{.cshrc}, \texttt{.bashrc}, or other
shell startup file. 

More advanced users can experiment with the \texttt{RVM\_BUILD\_COPY}
environment variable.  If this is set, then the 
\texttt{jbuild.linkBooter} phase of the build process is replaced by the
execution of ``\texttt{\$RVM\-\_\-BUILD\_COPY}''.  This opens up 
possibilities such as copying the build directory to a target
machine and executing \texttt{jbuild.linkBooter} remotely on the target
via \texttt{rsh} or \texttt{ssh}.  By setting \texttt{RVM\_BUILD\_COPY}
appropriately on the host platform, cross-platform building can become
a stream-lined process.

\subsection{Building Documentation}

The \texttt{\RVMTarFile} file contains a PostScript\Rweb{} version of this user's guide
in \texttt{\$RVM\_\-ROOT/rvm/doc}.  Additionally, the 
\xlink{developerWorks web page}{\RVMHomeURL} keeps an online version of
the user's guide and Javadoc\TMweb{} API, corresponding to the latest HEAD of the CVS
repository.

If you would like to recover the user's guide or Javadoc for an older release
of \jrvm{}, you can rebuild the documentation locally.  See the Makefile in
\texttt{\$RVM\_\-ROOT/rvm/doc/userguide} for rules on how to build the
HTML for the user's guide using\index{Hyperlatex}
\xlink{\texttt{hyperlatex}}[\footnote{You can get \texttt{hyperlatex}
    at \texttt{\HyperlatexURL}}]{\HyperlatexURL}.  To build the
Javadoc pages, use 
the \texttt{jdoc.sh} script in \texttt{\$RVM\_\-ROOT/rvm/bin}; this
script takes as its only command-line argument the directory into
which to output the Javadoc HTML.\@ 

% LocalWords:  SuSE AMD Blackdown JVM VM VMs RPMs downJa va JRE JVMs
