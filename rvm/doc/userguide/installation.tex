So you've gotten the RVM distribution.  Now what?  This section gives
instructions on how to install and run RVM for the new RVM user.

\subsection{System Prerequisites}
To build on any platform you will need the following:
\begin{itemize}
\item GNU make. You can download this from
\xlink{{\tt http://www.gnu.org/software/make/make.html}}
           {http://www.gnu.org/software/make/make.html}.

\item The Jikes\trademark Java\trademark compiler 
You can download this from
\xlink{{\tt http://oss.software.ibm.com/developerworks/opensource/jikes}}
           {http://oss.software.ibm.com/developerworks/opensource/jikes}.
You can use any of the pre-compiled binaries, or build it yourself from the
source. 

\item GNU tar is needed to extract the distribution tar file.  We have
experienced problems with the AIX tar program truncating file names.

\end{itemize}

\subsubsection{AIX/PowerPC}
In addition to the software mentioned above, to install, build, and
run RVM on a AIX/PowerPC environment, 
you will need 
\begin{itemize}
\item a PowerPC processor
\item AIX 4.3 or later, and
\item (recommended) 512MB of memory.
\item the IBM AIX Developer Kit for Java\trademark version 1.3.0.  You can download this from
\xlink{{\tt http://www.ibm.com/java/jdk/aix/index.html}}
           {http://www.ibm.com/java/jdk/aix/index.html}.
\end{itemize}

\subsubsection{Linux/IA32}
In addition to the software mentioned above, to install, build, and
run RVM on a Linux Intel environment, you will need 
\begin{itemize}
\item an Intel Architecture 32 bit processor
\item Linux  (we have run on RedHat 6.0 and 7.0)
\item The system has run in 386MB; we have not established a lower bound.
\item the IBM Developer Kit (at least version 2.1.3).  You can download this from
\xlink{{\tt http://www.ibm.com/java/jdk/Linux}}
           {http://www.ibm.com/java/jdk/Linux}.
\item {\tt ksh}. This must be installed in {\tt /bin/ksh}.
\end{itemize}

\subsubsection{Win32 and Cygwin/IA32}
In addition to the software mentioned above, to install, build, and
run RVM on a Win32/Cygwin Intel environment, you will need 
\begin{itemize}
\item an Intel Architecture 32 bit processor
\item a Win32 operating system (we have only tried it on Windows 2000).
\item Cygwin. You can download this from 
\xlink{{\tt http://www.cygwin.com}}
           {http://www.cygwin.com}. (I used dll version 1.1.8).
\item The system has run in 256MB on Windows 2000; we have not established
a lower bound.
\item the IBM Developer Kit (at least version 2.1.3).  You can download this from
\xlink{{\tt http://www.ibm.com/java/jdk/}}
           {http://www.ibm.com/java/jdk/}.
In a standard ``client for e-business'' you should already have this
installed.
\item ksh.  One alternative is pdksh from 
\xlink{{\tt http://www.cs.mun.ca/\~{}michael/pdksh}}
       {http://www.cs.mun.ca/\~{}michael/pdksh}.
\end{itemize}
Currently this option for running RVM is even less functional than
the Linux/IA32 configuration. It is not currently being maintained or
tested.  Some problems include:
\begin{itemize}
\item We haven't managed to successfully install a handler for hardware traps,
so the first hardware exception (eg a {\tt NullPointerException}) will cause
an ugly core dump.
\item Cygwin itself handles Samba-mounted DFS file
systems and one can thus build from a DFS-mounted shadow
directory. However, something about them (most likely the 
$\backslash\backslash$ ) is
still confusing RVM.  Therefore, your build directories and all class
files for the programs you want to run must be on your local file
system. Failure to do this will result in a class-not-found error.
\item Spaces in directory names aren't handled by jconfigure and the
other scripts.  Use cygwin's mount command to avoid them (in
particular the config file assumes that
C:$\backslash$Program~Files$\backslash$IBM is mounted as /IBM).
\end{itemize}


\subsection{Installation Overview}
The RVM distribution consists of a directory structure, which shadows a
CVS repository at Watson. 
\index{shadow}
We will refer to your own private copy of the 
directory tree as your {\em shadow}.  The shadow holds the RVM 
source files, tools needed to build RVM, and the sources of standard
library files. 
A more complete description of the shadow directory structure
is provided in Section~\ref{rvmroot}.

RVM can be configured in various ways. Multiple versions of RVM,
corresponding to different configurations, can be generated from 
one shadow. See Section~\ref{configs} for information about the various 
configurations.
\index{configurations}
The RVM  {\em boot image} and other files generated during the 
configuration process
\index{boot image}
are stored in a {\em build directory} which is logically separate from 
the shadow. See Section~\ref{rvmbuild} for a detailed description of the
files in the build directory.
\index{build directory}

In order to install RVM  you must:
\begin{enumerate}
\item Obtain the shadow
\item Set various environment variables
\item Edit two RVM environment scripts
\item Choose a configuration and run the configuration script to write
the appropriate directory and configuration specific files to the
build directory
\item Build an executable version of RVM
\end{enumerate}

The remainder of this section describes the process in greater detail.

\subsection{Installation Steps}

\begin{enumerate}
\item Obtain the shadow

Extract the RVM distribution into
a directory such as {\tt \$HOME/jalapeno-1.1}

\index{environment variables}
\index{RVM\_ROOT}
\index{RVM\_BUILD}
\index{PATH}
\item Set up environment variables

You need to set up the following shell environment variables:

\begin{description}
\item [{\tt RVM\_ROOT}] the directory that contains the extracted
distribution 
\item [{\tt RVM\_BUILD}] the directory where you would like the build
process to generate an executable RVM configuration

\item [{\tt RVM\_HOST\_CONFIG}] the configuration file used to specify
the software environment on which the system is generated i.e. where the
boot image is generated.

\item [{\tt RVM\_TARGET\_CONFIG}] the configuration file used to specify
the software environment where the system support is generated. i.e. where
the ``booter'' and ``C runtime'' will be generated.

\item[{\tt PATH}] your path should contain {\tt \$RVM\_ROOT/bin} in
order to pick up various scripts and utilities
\end{description}

The following environment variables specify the machine architecture and
the operating system on which the system is to run.
\begin{description}

\item [{\tt RVM\_TARGET\_ARCH}].specifies on what architecture
the RVM will run e.g.  powerpc, IA32 (for intel 32 bit)

\item [{\tt RVM\_TARGET\_OS}] specifies on what Operating System
the RVM will run e.g. aix, linux

\end{description}

Because the ARCH and OS  variables default to the system on which the ``build'' is
currently executing, they usually do not need to be 
explicitly set.


We recommend you set up the following variables in your shell configuration
file.  For example, for {\tt csh}, you might insert the
following into your {\tt .cshrc} file:

\begin{verbatim}
setenv RVM_ROOT $HOME/jalapeno-1.0   # <--define your shadow directory 
setenv RVM_BUILD $HOME/rvmBuild     # <--define your current build directory 
setenv PATH $RVM_ROOT/bin:$PATH
setenv RVM_HOST_CONFIG $RVM_ROOT/config/powerpc-ibm-aix4.3.3.0
setenv RVM_TARGET_CONFIG $RVM_ROOT/config/powerpc-ibm-aix4.3.3.0
\end{verbatim}

For a Linux-Intel environment, the  exports
would be replaced with the following:

\begin{verbatim}
setenv RVM_HOST_CONFIG $RVM_ROOT/config/i686-pc-linux-gnu
setenv RVM_TARGET_CONFIG $RVM_ROOT/config/i686-pc-linux-gnu
\end{verbatim}

For a Cygwin-Intel environment, the  exports
would be replaced with the following:

\begin{verbatim}
setenv RVM_HOST_CONFIG $RVM_ROOT/config/i686-pc-cygwin
setenv RVM_TARGET_CONFIG $RVM_ROOT/config/i686-pc-cygwin
\end{verbatim}

These two variables point to the same file when the type of system  
doing the build is the same as where  you are going 
the execute the RVM. To cross build a system
e.g build on AIX-powerpc for a Linux-Intel platform, see the section on Cross
Platform Building.

\item Edit configuration scripts

You must edit two scripts in the {\tt \$RVM\_ROOT/config}  directory to set 
up variables used by the installation process.  

First, edit the files that define the host and target configuration
environments in the {\tt \$RVM\_ROOT/config} directory.  You do not need to {\em
source} these variables in your working shell; variables in this file will be 
picked up by the installation scripts.  You must set the following
variables in this file:
\begin{description}
\item {\tt HOST\_JAVA\_HOME} the base directory for JDK JVM
\item {\tt HOST\_JAVA} the executable command for the JDK JVM
\item {\tt HOST\_JAVAC} the {\tt javac} executable for the JDK  
\item {\tt HOST\_REPOSITORIES} the {\tt rt.jar} archive for the JDK  
\item {\tt GNU\_MAKE} the GNU {\tt make} executable
\item {\tt JIKES} the Jikes\trademark {\tt jikes} executable.
\item {\tt CC} how to invoke the C/C++ compiler.
\item {\tt LDSHARED} how to link a shared C++ library.
\end{description}

A few additional environment variables that define the location of
the grep, fgrep and find command are also located in this file and
may need definitions.


\index{configurations}
\index{jconfigure script}
\item Choose configuration and populate your build directory

You will use the {\tt jconfigure} script (in {\tt \$RVM\_ROOT/bin}) to
populate your build ({\tt \$RVM\_BUILD}) directory with files.  You must
first choose a RVM configuration.

For novice users, two configurations are recommended.  (A discussion
of RVM configurations is contained in Section~\ref{configs}.)

\begin{itemize}
\item {\tt BaseBasecopyingGC}: a non-adaptive system that uses the
baseline  compiler everywhere, with the copying collector
\item {\tt OptOptcopyingGC}: a non-adaptive system that uses the
optimizing compiler everywhere, with the copying collector
\end{itemize}

Depending on your purposes (See Section~\ref{sssec:choosinggc}.) you
may want to choose another configuration, e.g.,
\begin{itemize}
\item {\tt OptOptnoncopyingGC}: a non-adaptive system that uses the
optimizing  compiler everywhere, with the noncopying (mark-sweep) collector
\end{itemize}

Run the {\tt jconfigure} script to set up the {\tt \$RVM\_BUILD}
directory for the configuration you desire.  This step creates
build scripts for your configuration and otherwise formats your
{\tt \$RVM\_BUILD} directory.
The {\tt jconfigure} script takes one argument, the name of the
configuration desired: 

\begin{verbatim}
% jconfigure <configuration>
\end{verbatim}

For example, to configure a build 
directory for the {\tt OptOptcopyingGC} configuration, type
the following command:

\begin{verbatim}
% jconfigure OptOptcopyingGC
\end{verbatim}

\index{jbuild script}
\index{boot image}
\index{RVM\_BUILD}
\item Build an executable version of RVM .  

Use the {\tt jbuild} script, located in the {\tt \$RVM\_BUILD} directory,
to build an executable system.  This script copies source files into
{\tt \$RVM\_BUILD/Jalapeno.classes}, preprocesses these files, generates
some code with template expansions, builds an executable C program to
start the RVM, and writes the RVM boot image.  The boot image is the
binary image of a ready-to-go instance of the RVM.

The {\tt jbuild} script must be run from the {\tt \$RVM\_BUILD}
directory. It prints a copious report of its operation which you may
save for future reference by redirecting standard out and err.

\begin{verbatim}
% cd $RVM_BUILD
% jbuild
\end{verbatim}


After the {\tt jbuild} script has completed successfully you should be able 
to run RVM.  (See Section~\ref{section:running}.)

Note: The jbuild for Linux/Intel currently produces approximately 13 error
messages, most pertaining to the fact that Reflection is not implemented.
They do not affect the operation of the system.

\end{enumerate}

\subsection{RVM Configurations}\label{configs}
\index{configuration names}

This section describes the RVM build configurations.

The standard RVM configurations follow the following naming scheme
\begin{verbatim}
       <boot image compiler> <runtime compiler> <garbage collector>
\end{verbatim}

\index{boot image compiler}
\index{runtime compiler}
where
\begin{itemize}
\item the {\em boot image compiler} is the compiler used to compile the RVM boot image.
\item the {\em runtime compiler} is the ``compiler'' used to compile
the classes loaded at runtime.  
\item the {\em garbage collector} is the garbage collection scheme used.
\end{itemize}

The types of compilers -- the baseline compiler, the quick compiler, and 
the optimizing compiler -- are designated by the names {\em Base},
{\em Quick},  and {\em Opt} respectively.  In these configurations,
all classes loaded at runtime are compiled once, by the specified
compiler.  This is different than the adaptive configurations,
discussed in Section~\ref{adaptive-configs}.

The garbage collector names follow the following format
\begin{verbatim}
           <type> [modifier]
\end{verbatim}

A garbage collector may have any of the following types:

\begin{description}
\item[noGC] no garbage collection is performed
\item[copyingGC] a copying semi-space collector
\item[noncopyingGC] a mark-and-sweep (non copying) collector
\item[hybridGC] a hybrid collector, semi-space for the nursery and
mark-and-sweep for the mature space
%%\item[refcountGC] a reference counting garbage collector
\end{description}

To indicate that a garbage collector is a generational garbage collector 
the modifier {\em gen} may be appended to the garbage collector name.

For example, to specify a compiler with a baseline-compiled boot image
that will 
compile classes loaded at runtime using the optimizing compiler and that uses
a non-generational copying garbage collector use the name 
{\em BaseOptcopyingGC}.

The standard configurations may be modified in the following manner. 
\begin{itemize}
\item The word 
{\em Full} at the beginning of the configuration name identifies a 
configuration
such that all the RVM classes are included in the boot image (by default
a small subset of the RVM classes are included in the boot image). 
\item The word
{\em Fast} at the beginning of the configuration name identifies a Full
configuration where all assertion checking has been turned off. 
\end{itemize}
A boot image with
either of these modifications is likely to run faster than without
(the opt compiler will be opt compiled),
but take longer to build.  The {\tt
Fast} configuration is a pseudonym for FastOptOptcopyingGC, and is
recommended for most non-adaptive performance runs.

\subsubsection{Adaptive Configurations} \label{adaptive-configs}
\index{adaptive configurations}
In the non-adaptive configurations, all classes
loaded at runtime are compiled once by the specified
compiler: base or opt.  Another option is to build one configuration,
an adaptive configuration,
where the runtime compiler is either
\begin{itemize}
\item specified on the command line (base or opt), or
\item selected automatically as the application runs
\end{itemize}

The first choice allows an adaptive configuration to provide the same
functionality as a non-adaptive configuration.  One image is built,
and the runtime compiler can be specified at the command line as
follows:
\begin{verbatim}
  jalapeno -X:aos:primary_strategy=baseonly
            or
  jalapeno -X:aos:primary_strategy=optonly
\end{verbatim}

The second choice intially compiles all methods with the
baseline compiler and then automatically selects hot methods for
recompilation by the opt compiler at an appropriate optimization
level. Further details are provided in Section~\ref{section:aosdetails}.

The adaptive configurations follow the following naming scheme
\begin{verbatim}
           [boot image compiler] Adaptive  <garbage collector>
\end{verbatim}

For example, to configure a build 
directory for an adaptive configuration, where the Opt compiler is 
used to compile the boot image (but is not included in the boot image
and assertions are turned on), and the copyingGC garbage collector is
used, use the following command:

\begin{verbatim}
% jconfigure OptAdaptivecopyingGC
\end{verbatim}

Section~\ref{section:running} describes how this image can be used in
the manner mentioned above.

To view a complete list of configurations see 
{\tt \$RVM\_ROOT/bin/jconfigure}.
Some configurations may not be available at your site. 

\subsection{Cross Platform Building}

The RVM build process consists of two major phases: the building of a
RVM {\em boot image}, and the building of a RVM {\em boot loader}.
The boot image is build using a Java\trademark program executed within a host
JVM and is therefore platform-neutral.  By contrast, the boot loader
is written in C, and must be compiled on the target platform.

Because the building of the boot image can be a relatively lengthy
process, it can be advantageous to perform that task somewhere other
than the target platform.  By default, the build process will target
the operating system and architecture on which the {\tt jconfigure}
script is run.  
    Two environment variables can be used to
override this behavior and explicitly specify a different target: {\tt
  RVM\_TARGET\_ARCH } and {\tt RVM\_TARGET\_OS }. The current valid target
architectures are ``powerpc'' and ``IA32'' (intel). The current valid
Operating systems are ``aix'' and ``linux''.

    Two additional environment variables are used to describe the 
environment on which the boot image is to be built RVM\_HOST\_CONFIG  and where
the boot image loader  RVM\_TARGET\_CONFIG is to be built. These variables
each point to a file where additional environmnet variables the location
of software components like javac, Jikes\trademark, and the C compiler. This distribution
contains examples of such files in the {\tt \$RVM\_ROOT/config} directory.   

For example, to build a BaseBasecopyingGC system for AIX {\em on a Linux
host}:
\begin{verbatim}
% setenv RVM_ROOT $HOME/rvmRoot
% setenv RVM_BUILD $HOME/rvmBuild
% setenv PATH $RVM_ROOT/bin:$PATH
% setenv RVM_TARGET_ARCH powerpc
% setenv  RVM_TARGET_OS aix
% setenv RVM_TARGET_CONFIG=$RVM_ROOT/config/powerpc-ibm-aix4.3.3.0
% setenv RVM_HOST_CONFIG=$RVM_ROOT/config/i686-pc-linux-gnu
% jconfigure BaseBasecopyingGC
% cd $RVM_BUILD
% jbuild
\end{verbatim}

This phase of the build process will complete with the words ``{\tt
  please run me on AIX}''.


The build process is then completed by building just the boot loader {\em
  on an AIX host}:

\begin{verbatim}
% setenv RVM_ROOT $HOME/rvmRoot
% setenv RVM_BUILD $HOME/rvmBuild
% setenv PATH $RVM_ROOT/bin:$PATH
% jbuild -booter
\end{verbatim}

After the {\tt jbuild -booter} script has completed successfully you should be able 
to run RVM. 

The building of the boot loader must occur in the same directory as
the rest of the build.  This can either be done transparently via a
network file system, or by copying the build directory from the first
host to the target.  Of course {\tt RVM\_ROOT}, {\tt RVM\_BUILD }
and {\tt PATH} need not be explicitly set each time: they could have
been set in your {\tt .cshrc}.

More advanced users can experiment with the {\tt RVM\_BUILD\_COPY}
environment variable.  If this is set, then the {\tt
  jbuild.linkBooter} phase of the build process is replaced by the
execution of {\tt `\$RVM\_BUILD\_COPY`}.  This opens up a lot of
possibilities, including: copying the build directory to a target
machine and executing {\tt jbuild.linkBooter} remotely on the target
via {\tt rsh} or {\tt ssh}, etc.  By setting {\tt RVM\_BUILD\_COPY}
appropriately on the host platform, cross-platform building can become
a stream-lined process.









