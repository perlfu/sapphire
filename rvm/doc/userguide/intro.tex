This section provides an overview of the Jikes RVM as well as
information for how best to use this document.

\subsection{Welcome to the Jikes RVM}

The Jikes\JikesTMFootnote RVM is a Research Virtual Machine 
 developed at the 
\xlink{IBM}{\IBMURL} 
\xlink{T.J.\ Watson Research Center}{\WatsonURL}.  Key
features of the system include
\begin{itemize}
\item the entire virtual machine (VM) is implemented in the
  Java\JavaTMFootnote  programming language,
\item the VM utilizes two compilers and no interpreter,
\item a family of parallel, type-exact garbage collectors,
\item a lightweight thread package with compiler-supported preemption,
\item an aggressive optimizing compiler, and 
\item a flexible online adaptive compilation infrastructure.
\end{itemize}

A significant body of information about the Jikes RVM 
(formerly known as 
\xlink{\jp}{\JalapenoHomeURL}) appears 
in our published
papers.  For overviews of the system's structure, including the runtime system,
optimizing compiler, and adaptive systems, see the published papers
available from the \jrvm\ web page:
\begin{quote}
\xlink{{\RVMPubsURL}}{\RVMPubsURL}
\end{quote}

The best paper for a general introduction to RVM is 
the 
\xlink{IBM Systems Journal, January 2000
paper
\T~\cite{jalapeno-ibmsj-00}
}{\SystemJournalPaperURL}.  
For introductions to the
optimizing compiler and adaptive system, see the 
\xlink{1999 ACM Java Grande\begin{iftex}~\cite{jalapeno-opt-grande-99}\end{iftex}}
{\JavaGrandePaperURL}
 and 
 \xlink{2000 OOPSLA\begin{iftex}~\cite{jalapeno-adaptive-00}\end{iftex}}
{\OOPSLAPaperURL}  
papers, respectively.

You may also find the PACT'01 tutorial slides on the system useful.
They are available at
\begin{quote}
\xlink{{\RVMSlidesURL}}{\RVMSlidesURL}
\end{quote}

The Jikes RVM is a bleeding-edge research project.  You will find that
some of the code  does not live up to product  
quality standards. Don't hesitate to help rectify this by
contributing clean-ups, bug fixes, and missing documentation to 
the project.  

Many academic groups have adopted the Jikes RVM as their primary
research infrastructure, resulting in 
\xlink{publications}{\RVMUsersPubsURL}
in leading
conferences, such as PLDI, POPL, OOPSLA, and SIGMetrics.  In the first six
months of its open source release, the VM has been downloaded by over
a thousand unique IP addresses, including over seventy
universities around the world. A list
of some of the \xlink{users}{\RVMUsersURL}
is available.

\JikesTMFooter

\JavaTMFooter

\subsection {About this document}

This document provides Jikes\trademark RVM information that is not covered in
our published papers.  For high-level overviews, algorithms, and
structures, you will find the published papers to be the best starting
place. This document supplements
the Jikes RVM distribution, focusing on implementation
details of how to build, run, and add functionality to the system.

This document is available as both postscript and HTML.  You will find the
HTML version more useful, as it includes hyperlinks. The HTML version is
available at 
\xlink{{\tt \RVMUserGuideURL}}{\RVMUserGuideURL}.  Each released Jikes RVM
tarball holds a postscript version of this document as {\tt userguide.ps}
in the root tarball directory.

The RVM web page includes 
javadoc API 
\xlink{documentation}{\RVMJavadocURL}. 
This HTML should be the
primary reference for individual classes.  The javadoc for the optimizing
compiler classes contains at least some description for most classes,
while most other classes provide javadoc with only a minimal API.

You may find sections of this user's guide missing, incomplete or
otherwise confusing. We intend this document to live as a continual
work-in-progress, hopefully growing and maturing as community members
edit and add to the guide.  Please accept this invitation to
contribute.

Please send feedback, bug fixes, and text contributions to the Jikes RVM
researchers 
\xlink{mailing list}{\RVMResearcherMailingListURL}.  
Constructive criticism will be cheerfully 
accepted. 

\JikesTMFooter
