\subsection {Welcome to RVM}

The Jikes\trademark RVM is a Research Virtual Machine for
Java\trademark developed at the IBM T.J.\ Watson Research Center.  Key
features of RVM include
\begin{itemize}
\item the entire virtual machine (VM) is implemented in the
  Java\trademark  programming language,
\item the VM utilizes two compilers and no interpreter,
\item a family of parallel, type-exact garbage collectors,
\item a lightweight thread package with compiler-supported preemption,
\item an aggressive optimizing compiler, and 
\item a flexible online adaptive compilation infrastructure.
\end{itemize}

A significant body of information about RVM (formerly known as \jp) appears 
in our published
papers.  For overviews of RVM structure, including the runtime system,
optimizing compiler, and adaptive systems, see the published papers
available from the RVM web page:
\begin{quote}
\xlink{{\RVMHomeURL}}{\RVMHomeURL}
\end{quote}

The best paper for a general introduction to RVM is 
the 
\xlink{IBM Systems Journal, January 2000
paper
\T~\cite{jalapeno-ibmsj-00}
}{\SystemJournalPaperURL}.  
For introductions to the
optimizing compiler and adaptive system, see the 
\xlink{1999 ACM Java Grande\begin{iftex}~\cite{jalapeno-opt-grande-99}\end{iftex}}
{\JavaGrandePaperURL}
 and 
 \xlink{2000 OOPSLA\begin{iftex}~\cite{jalapeno-adaptive-00}\end{iftex}}
{\OOPSLAPaperURL}  
papers, respectively.

As you know, RVM is a bleeding-edge research project.  You will find that
some of the code and most of documentation does not live up to product  
quality standards. Don't hesitate to help rectify this by
contributing clean-ups, bug fixes, and missing documentation to 
the project.  
\subsection {About this document}

This document provides RVM information that is not covered in
our published papers.  For high-level overviews, algorithms, and
structures, you will find the published papers to be the best starting
place. This document supplements
the RVM distribution, focusing on implementation
details of how to build, run, and add functionality to RVM.

This document is available as both postscript and html.  You will find the
html version more useful, as it includes hyperlinks. The html version is
available at 
\xlink{{\tt \RVMUserGuideURL}}{\RVMUserGuideURL}.

The RVM web page includes 
javadoc API 
\xlink{documentation}{\RVMJavadocURL}. 
This html should be the
primary reference for individual classes.  The javadoc for the optimizing
compiler classes contains at least some description for most classes,
while most other classes provide javadoc with only a minimal API.

This version of the user's guide is for use with the initial open-source
release.  You may find sections missing or incomplete. We intend this
document to live as a continual work-in-progress, hopefully growing
and maturing as community members edit and add to the
guide.  Please accept this invitation to contribute.

Please send feedback, bug fixes, and text contributions to the RVM
researchers mailing list.  Constructive criticism will be cheerfully 
accepted. 
