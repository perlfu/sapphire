\subsection {Welcome to \jp}

\jp\ is a Research Virtual Machine for Java\trademark developed at the IBM T.J.\ Watson 
Research Center.  Key features of \jp\ include 
\begin{itemize}
\item the entire virtual machine (VM) is implemented in the
  Java\trademark  programming language,
\item the VM utilizes two compilers and no interpreter,
\item a family of parallel, type-exact garbage collectors,
\item a lightweight thread package with compiler-supported preemption,
\item an aggressive optimizing compiler, and 
\item a flexible online adaptive compilation infrastructure.
\end{itemize}

A significant body of information about \jp\ appears in our published
papers.  For overviews of \jp\ structure, including the runtime system,
optimizing compiler, and adaptive systems, see the published papers
available from the \jp\ web page:
\begin{quote}
\xlink{{\tt http://www.research.ibm.com/jalapeno}}{http://www.research.ibm.com/jalapeno}
\end{quote}

The best paper for a general introduction to \jp\ is 
the 
\xlink{IBM Systems Journal, January 2000
paper
\T~\cite{jalapeno-ibmsj-00}
}{http://www.research.ibm.com/jalapeno/publication.html\#ibmsj00}.  
For introductions to the
optimizing compiler and adaptive system, see the 
\xlink{1999 ACM Java Grande\begin{iftex}~\cite{jalapeno-opt-grande-99}\end{iftex}}
{http://www.research.ibm.com/jalapeno/publication.html\#grande99}
 and 
 \xlink{2000 OOPSLA\begin{iftex}~\cite{jalapeno-adaptive-00}\end{iftex}}
{http://www.research.ibm.com/jalapeno/publication.html\#oopsla00\_aos}  
papers, respectively.

As you know, \jp\ is a bleeding-edge research project.  You will find that
some of the code and most of documentation does not live up to product or 
even open-source quality standards. Don't hesitate to help rectify this by
contributing clean-ups, bug fixes, and missing documentation to your
contact at Watson.  

\subsection {About this document}

This document provides \jp\ information that is not covered in
our published papers.  For high-level overviews, algorithms, and
structures, you will find the published papers to be the best starting
place. This document supplements
the \jp\ distribution, focusing on implementation
details of how to build, run, and add functionality to \jp.

The \jp\ distribution also includes javadoc API documentation for the 
source code in the directory {\tt doc/api}.  This html should be the
primary reference for individual classes.  The javadoc for the optimizing
compiler classes contains at least some description for most classes,
while most other classes provide javadoc with only a minimal API.

This version of the user's guide is for use with the university
release.  You may find sections missing or incomplete. We intend this
document to live as a continual work-in-progress, hopefully growing
and maturing as good Samaritans volunteer to edit and add to the
guide.  Please accept this invitation (plea) to contribute.

Please send feedback, bug fixes, and text contributions to Stephen Fink,
{\tt sjfink@us.ibm.com}.  Constructive criticism will be cheerfully 
accepted. 

\subsection{What's New in this Release?}

Below is a list of major changes from the previous release
\begin{description}
\item [Linux/IA32 minimal functionality]
%% Tony/Maria
This base release includes a partial release of the Linux/IA32 support.
It includes the baseline compiler, threading, locking, copying
and noncopying garbage collectors, and the debugger (jdp).

This release does NOT include: Reflection, SMP support, Java\trademark native
Interface(JNI), 
the remaining garbage collectors,
and the optimizing compiler. 

\item [Adaptive optimization system]
This release includes the adaptive optimization system as described
in the \xlink{2000 OOPSLA paper}
{http://www.research.ibm.com/jalapeno/publication.html\#oopsla00\_aos}
\T~\cite{jalapeno-adaptive-00}

\item [Opt compiler counter instrumentation infrastructure:]
%% Matt
Basic infrastructure to simplify the process of inserting
instrumented counters in opt-compiled code. See
Section~\ref{counting_events}.

\item [Hybrid garbage collector]
%% Steve Smith
A generational collector that combines a Nursery region for object
allocation, managed using the logic of the copying generational collector,
and a Mature Object Heap for objects that survive one or more
collections, managed using the logic of the non-copying
mark-sweep collector.

\item [Elimination of specific Jikes\trademark dependency]
%% Igor
As of this release, \jp\ is not dependent on a particular version of
Jikes\trademark.  It will work with any Jikes\trademark binary.  \jp\ still requires that
its source-to-bytecode compiler be Jikes\trademark (as opposed to javac), because
it uses features that are only available in Jikes\trademark.

\item [VM callbacks]
%% Igor
A mechanism to allow arbitrary code to be executed on certain VM events.
See section~\ref{sssec:callbacks}.

\item [Updated libraries]
%% Maria  source?

\end{description}
