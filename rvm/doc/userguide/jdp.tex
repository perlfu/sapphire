This section provides information regarding {\tt jdp}, the RVM
debugger.
\index{debugging}
\index{jdp}

\subsection{About {\tt jdp}}

\index{Remote Reflection}
  {\tt jdp}, the RVM Debugger Primitive, is a low-level symbolic debugger 
developed to support the
RVM effort.  Because no existing debugger satisfactorily meets the
needs of our RVM, {\tt jdp} was written essentially from scratch, hence the
name primitive.  {\tt jdp} uses a technique called Remote
Reflection (see the related paper on the RVM 
\xlink{web page}{\RVMPubsURL}).  This
allows {\tt jdp} to run out-of-process, yet it uses reflection extensively to
access the internal data structures as if it resides inside the RVM.

{\tt jdp} is written in Java\trademark and JNI with native code in C.  {\tt jdp}
uses the
AIX {\tt ptrace} interface and the standard Unix process support (fork,
wait, ...).  When porting to another platform, this native portion
(called from Platform.java) will have to be rewritten.  Since {\tt jdp} and
the RVM being debugged are two separate processes, an interpreter is
used to connect {\tt jdp} with the RVM:  it maps the access bytecodes in the
reflection methods to obtain the internal values in the RVM.
  
  {\tt jdp} is not intended to be a general purpose debugger; it is 
customized for RVM but is easily extendable.  

   Because the optimizing compiler does not currently provide all of
   the mapping support required, {\tt jdp} functionality is limited in
   opt-compiled code. 

\subsection{Getting started}

   You can use {\tt jdp} in one of three modes: debugging the boot image only, 
general debugging, and debugging a RVM that is already running:

\begin{enumerate}
\item Use the boot image only mode if you are only debugging code
in the boot image since this is faster and the debugger is more stable
(even when the RVM becomes corrupted):
\begin{verbatim}
jdp -jdpbootonly myProgram myArguments
\end{verbatim}
   In this mode, methods of classes outside the boot image appear
as "unknown method" and you cannot set breakpoint for these methods.


\item Use the general mode if you are debugging codes in classes that are 
dynamically loaded:
\begin{verbatim}
jdp myProgram myArguments
\end{verbatim}
   This mode is slower because there are two levels of interpretation
and the debugger can get lost if the dictionaries in the RVM become 
corrupted.


\item If the RVM is already running and appears to be hung, you can attach 
{\tt jdp} to this RVM by:
\begin{verbatim}
jdp -jdpattach<processID>
\end{verbatim}
   The process ID is obtained from the "ps" command.  
   The option "{\tt -jdpbootonly}" can be used in conjunction with 
   "{\tt -jdpattach}".  The effect is that the debugger is faster and more
   stable, but it will only show methods in the boot image.

\item If there is an error in the RVM initialization, you can start debugging
 at a point where the boot image has been loaded but the RVM has not begun
 executing yet. Specify the option:
\begin{verbatim}
jdp -jdpviewboot
\end{verbatim}
  Note that if the option "{\tt -jdpbootonly}" is used, the option 
  "{\tt -jdpviewboot}" is also on by default because jdp would be unable
  to find the dynamically loaded user's main method.


\end{enumerate}


\subsection{{\tt jdp} environment}

   This section describes the files and settings that the debugger
needs.  The {\tt jdp} command is a {\tt ksh} script that does some preliminary
argument parsing and fills in default arguments before invoking the
debugger in the appropriate mode.

The debugger requires the following files:

\begin{enumerate}
\item The RVM boot image:
   This file is typically {\tt \$RVM\_BUILD/RVM.image} but it could
   have other names with the {\tt .image extension}.  You can specify a
   different boot image name.
\begin{itemize}
\item    {\tt jdp} argument:    {-i bootImageName}
\item   Default value:          {\tt \$RVM\_BUILD/RVM.image}
\end{itemize}

\index{JTOC}
\index{symbol map}
\index{jconfigure script}
\index{configurations}
\item The symbol map:
   This file is a symbol map of the boot image.  It contains a listing of the 
   JTOC and the compiled methods.  It is generated when the build configuration 
   in {\tt \$RVM\_ROOT/rvm/bin/jconfigure} contains the flag:
        {\tt export GENERATE\_MAP=1}

   {\tt jdp} uses offset 
   values from this file to find all symbols in the boot image.
   In the dynamic mode, the interpreter also uses this file to find the base 
   offset for the dictionaries for the remote reflection feature.
\begin{itemize}
\item   {\tt jdp} argument:     (none)
\item    Default value: {\tt \$RVM\_BUILD/RVM.map}
\end{itemize}

\item The list of classes in the boot image:
   {\tt jdp} looks for the file {\tt \$RVM\_BUILD/RVM.primordials}

\begin{itemize}
\item   {\tt jdp} argument:     {\tt -n classListName}
\item   Default value:  {\tt \$RVM\_BUILD/RVM.primordials}
\end{itemize}

\item The program to load and run the RVM boot image:
   This file is typically {\tt \$RVM\_BUILD/JikesRVM}.
   Normally you don't have to be concerned about this file since it rarely
   changes.
\begin{itemize}
\item   {\tt jdp} argument:     {\tt -jdpbootrunner booterName}
\item   Default value:  {\tt \$RVM\_BUILD/JikesRVM}
\end{itemize}
\end{enumerate}

In addition to these files, the following setting can be made:

\begin{enumerate}
\index{breakpoints}
\item Initial breakpoint:  
   This is where the RVM will stop first.  By default, {\tt jdp} stops
   after the prologue of the user's main method.  However, this point
   comes after the RVM has executed for a significant amount of time.
   If there is an error in the initialization of the RVM, {\tt jdp} can stop
   before the RVM begins execution by specifying the argument {\tt -jdpviewboot}.
   In this case, {\tt jdp} sets the initial breakpoint at a point after the
   first 4 instructions in the assembler procedure .bootThread in:
        {\tt \$RVM\_ROOT/rvm/src/tools/bootImageRunner/bootThread.c}.
   This is after the RVM boot image has been loaded into memory and
   the 4 registers jtoc, proc, thread index and FP have been initialized.
   At this point, only one thread is running and the stack has only 
   one frame pointing to the booting C code.
\begin{itemize}
\item   {\tt jdp} argument:     {\tt -jdpbreakpoint breakpointHexValue}
\item    Default value:         bootThread
\end{itemize}

\item Process ID:
   This is used for attaching {\tt jdp} to a currently running RVM.  After 
   initializing itself, {\tt jdp} will send an interrupt to the running process 
   to gain control.  The process ID is from the "{\tt ps}" command in AIX.
\begin{itemize}
\item   {\tt jdp} argument:     {\tt -jdpattachXXXX} where XXXX is the process ID
\item    Default value:         (none)
\end{itemize}
\end{enumerate}


Finally, jdp assumes the following conventions in navigating within the 
boot image (they arise from the base compiler convention):

\begin{enumerate}
\item The method ID is saved a the end of the instruction block for each
   method.

\item The end of the prolog of the instruction block is marked by a code pattern.
For AIX/PowerPC, it is:
\begin{verbatim}
          4ffffb82      
\end{verbatim}
   which is the instruction:

\begin{verbatim}
         cror   0x1f,0x1f,0x1f
\end{verbatim}
For Linux/Intel, it is:
\begin{verbatim}
          90
\end{verbatim}
   which is the NOP instruction:
\end{enumerate}


\subsection {{\tt jdp} commands}

\begin{tabular}{|l|l|l|} \hline
Command       & Shortcut & Description     \\ \hline
step          & s   & step current thread by instruction, into method           \\ 
stepbr        & sbr & step current thread by instruction, over method           \\ 
stepline      & sl  & step current thread by source line, into method   \\ 
steplineover  & slo & step current thread by source line, over method   \\ 
creturn       & cr  & continue to calling method (up one stack frame)           \\ 
cont          & c   & continue all threads                                      \\ 
kill          & k   & terminate program                                         \\ 
run           & run & start new program                                         \\ 
break         & b   & list/set breakpoint                                       \\ 
clearbreak    & cb  & clear breakpoints                                         \\ 

thread        & th  & select or turn off thread context                         \\ 
where         & w   & print short stack trace                                   \\ 
whereframe    & wf  & print full stack trace                                    \\ 
stack         & f   & display formatted stack                                   \\ 
mem           & m   & display memory                                            \\ 
memraw        & mraw & display actual memory (jdp breakpoints are visible)      \\ 
wmem          & wm  & write memory                                              \\ 
reg           & r   & display registers                                         \\ 
wreg          & wr  & write register                                            \\ 
printclass    & pc  & print the class statics or the type of an object address  \\ 
print         & p   & print local variables or cast an address as an object     \\ 
listi         & li  & list machine instruction                                  \\ 
listt         & lt  & list threads                                              \\ \hline 

quit          & q   & exit debugger                                             \\ 
preference    & pref & set user preference                                      \\ 
x2d, d2x      & (same) & convert number between hex and decimal                 \\
verbose       & v   & toggle verbose mode                                       \\ 
(macro name)  &   & load and execute this macro (a text file with suffix .jdp)  \\ 
(enter)       &   & repeat last command                                         \\ \hline 
\end{tabular}

To get more information on a specific command, type: 
\begin{verbatim}
        help thiscommand
\end{verbatim}


\subsection{ {\tt jdp} Macros}
A jdp macro is a text file that has the {\tt .jdp} suffix and contains a sequence of normal {\tt jdp} commands

\begin{itemize}
\item When you type {\tt mymacro} at the command line, {\tt jdp} will look for 
{\tt mymacro.jdp}
 in the normal class path and execute each line in the macro file as if 
 they are entered at the command line.
\item On entry to {\tt jdp}, if the file {\tt startup.jdp} exists in the current directory, 
 it will be loaded and executed automatically.
\end{itemize}

This should be convenient for:
\begin{itemize}
\item Short cut to repeat a series of commands to get a debugging point
\item Regression testing:  redirect the output and compare with previous 
 results.  This is specifically intended for regression testing of {\tt jdp} 
 itself but  may be useful for other tests as well, especially if the test 
 needs to inspect specific runtime values in registers, memory.
\end{itemize}

\subsection{Debugging tips}

\begin{itemize}
\item Compiling the system with local variables:
  The default boot image is built without the local variable tables.
  To get local variables, set this flag to true in your local shadow:
\begin{verbatim}
        VM_Control.LoadLocalVariableTables
\end{verbatim}
  Then rebuild the boot image. This will increase the boot image size 
  by about 10\%.

\index{breakpoints}
\item Setting breakpoint:
  Sometimes a breakpoint cannot be set.  In the boot image only mode, {\tt jdp}
  cannot find code outside the boot image.  In the general mode, {\tt jdp} cannot
  find codes for classes that have not been loaded and compiled, simply
  because they don't exist yet.  To get around these
  situations, the boot image contains a dummy method which {\tt jdp} can always
  find:
        {\tt VM.debugBreakpoint}.  
  You can insert a call to this method in the code where you want to stop.
  Then set a breakpoint on this method, proceed to this breakpoint, then
  continue to the caller to reach your method:
\begin{verbatim}
        b VM.debugBreakpoint
        c
        cr
\end{verbatim}
  In the general debugging mode, when {\tt jdp} stops at this breakpoint, the 
  class containing the calling method has been loaded and you can set then
  breakpoints in any method of this class.


\index{breakpoints}
\item Setting breakpoint in the prolog:
  Normally, {\tt jdp} skips past the prolog code when it sets a breakpoint in
  a method.  To force {\tt jdp} to set a breakpoint at the beginning of the
  method prolog, specify 0 as the line number:
\begin{verbatim}
        b myclass.mymethod:0
\end{verbatim}


\item Expanding expression:
  Symbolic expression for printing class static variable and stack local
  variable can be nested arbitrarily
\begin{verbatim}
        pc VM_Scheduler.threads[1].contextRegisters
        p 0:myLocalVar.field1.field2
\end{verbatim}


\item Casting an address as an object:
  If you have the address of an object and you know the class name,
  you can cast the address to print the object:
\begin{verbatim}
        p (className) xxxxxxxx
\end{verbatim}
  This can also be used to print a concrete object of an abstract class.
  If the casting is not correct, you will see messages such as:
\begin{verbatim}
        CAUTION, address not accessible: 0x80620420
\end{verbatim}


\item Getting the class name of an object address:
  If you have the address of an object and want to know the type:
\begin{verbatim}
        pc xxxxxxxx
\end{verbatim}


\item Getting to low level string:
  Strings are stored as {\tt VM\_Atom}s in the RVM.     
  To print the string as character, print the memory at the address of
  the field {\tt VM\_Atom.val}:

\begin{verbatim}
        jdp:0>p (VM_Atom) 30266638
        VM_Atom = 
          VM_Atom @0x30266638
            val = {100, 101, 98, 117, 103, 66, 114, 101, 97, 107, 112, 111, 105, 110, 116} @0x30269628
            hash = 1544209252  @0x30266624

        jdp:0>m 30269628
          0x30269628: 64 65 62 75    d e b u
          0x3026962c: 67 42 72 65    g B r e
          0x30269630: 61 6b 70 6f    a k p o
          0x30269634: 69 6e 74 00    i n t .
          0x30269638: 30 05 67 90    0 . g .
\end{verbatim}


\item Restarting your program:
  If your program runs to completion and you want to reexecute it,
  type "run" in {\tt jdp}.  A new RVM will be loaded and started in a new
  process.  The existing breakpoints will be saved and set again 
  when the boot image has been loaded.


\item Display number in hex or decimal:
  Integers are normally displayed in decimal for variables and
  in hex for stack.  Sometimes variables contain addresses as 
  integer and stack entries contain small integer values.    In
  these cases, it may be more convenient to view in the alternate
  format.
  To change the display format for variables to hex:
\begin{verbatim}
        pref int x
\end{verbatim}
  To change the display format for stack to decimal:
\begin{verbatim}
        pref stack d
\end{verbatim}
  Likewise, floating point registers are displayed as float, but
  to compare the actual values without any rounding, it may be more
  convenient to view the hex values.  To change the format to hex:
\begin{verbatim}
        pref fpr x
\end{verbatim}

  The format preference can be changed back by specifying x, d or f
  as appropriate.

\item Interrupting a hung RVM:  If the RVM is running under the debugger
  and appears to be hung, you can interrupt it by going to a different
  window and killing the RVM process.  This will send a signal to the
  RVM process and {\tt jdp} will catch the signal before it gets to
  the RVM process.  Hitting control-c in the jdp window will kill 
  {\tt jdp} itself.


\index{JNI}
\item Debugging with native code:  {\tt jdp} provides limited debugging
  for native codes that are interspersed via JNI.  {\tt jdp}
  displays the C procedure names in the stack traceback and stepping by 
  machine code is possible.  You can also view the raw C stack frame and 
  its registers, but native variables and line numbers are not available.
  For C program that creates the RVM via the JNI\_CreateJavaVM call, 
  {\tt jdp} can be used by specifying the initial breakpoint (see the 
  {\tt jdp} script).




\item Debugger error messages:
  What these messages mean:

\begin{itemize}
\item {\tt "CAUTION, address not accessible: 0x80620420"}
        This means that {\tt jdp} got an error when it reads an address
        in the RVM space.  This may have occurred because {\tt jdp} is 
        dereferencing an object from a bad address, or the RVM has
        become corrupted.  This message does not mean that {\tt jdp}
        has crashed; you should be able to continue debugging.

\item {\tt jid}:
        In the general debugging mode (running with the interpreter),
        this means that the interpreter has encountered an internal 
        error and has entered its own internal debugging mode.  At this 
        point, you can type "w" to see the interpreter stack and report
        the problem.
\end{itemize}
\end{itemize}

