 As of version 2.2.1, Jikes\JikesTMFootnote\ RVM has adopted the
\xlink{GNU Classpath}{http://www.classpath.org} libraries.  GNU
Classpath is an ongoing project working towards a complete set of
Java\JavaTMFootnote\ libraries.  The GNU Classpath web site includes
rough guides as to what portions of the libraries exist.  Jikes RVM
runs with unmodified versions of GNU Classpath straight from its ftp
site, and the Jikes RVM configuration process can check out
appropriate versions of the library as needed.  Thus, the integration
can be seamless from a user's perspective.

However, the fully automatic checkout and build of GNU Classpath by
the Jikes RVM build process requires that your machine be capable of
building the classpath libraries from sources (including running {\tt
autoconf} et al).  This requires a number of RPMs to be installed that
aren't included in the default Red Hat 7.3 or 8.0 builds (they are
part of the full Red Hat image).  So if you haven't built the GNU
classpath libraries from source before and you don't have a need to be
on the CVS head of Jikes RVM and/or GNU Classpath, we suggest you use
the ``Manual GNU Classpath'' option described below.

\begin{description}
\item[Manual GNU Classpath] If you set the CLASSPATH\_ROOT variable in
your Jikes RVM configuration file, the build process will use the
libraries that are located there.  The setup must mimic the one
generated automatically, which means that
{\tt \$CLASSPATH\_ROOT/`config.guess`} must contain a build for your platform
and {\tt \$CLASSPATH\_ROOT/classpath} must be a version of GNU Classpath from
CVS.  This configuration is useful if (1) you can't build GNU
Classpath from a CVS checkout and want to use a classpath release tar
ball or (2) you want to do classpath development using Jikes RVM. The
current release of Jikes RVM works with GNU classpath version \classpathversion.

\item[Automatic GNU Classpath] If you do not set CLASSPATH\_ROOT and 
{\tt RVM\_WITH\_GNU\_CLASSPATH=1} then the Jikes RVM build process will do a
CVS check out of appropriate version of GNU Classpath (using the date
in {\tt \$RVM\_ROOT/rvm/bin/classpath.stamp}) and build it for you.  This is
the easiest way to use GNU Classpath with Jikes RVM if you can do it.
This won't work on default (non-full) Red Hat 7.3 and 8.0 images
as the GNU classpath {\tt automake}/{\tt autoconf} step will fail.

\end{description}

Some library classes require special VM support.  This is provided by
the classes found in {\tt rvm/src/vm/libSupport}. 

\JavaTMFooter
\JikesTMFooter
