 As of version 2.2.1, Jikes\TMweb{} RVM has adopted the
\xlink{GNU Classpath}{http://www.classpath.org} libraries.  GNU
Classpath is an ongoing project working towards a complete set of
Java\TMweb{} libraries.  The GNU Classpath web site includes
rough guides as to what portions of the libraries exist.  Jikes RVM
runs with unmodified versions of GNU Classpath straight from its ftp
site, and the Jikes RVM configuration process can check out
appropriate versions of the library as needed.  Thus, the integration
can be seamless from a user's perspective.

\begin{description}
\item[Manual GNU Classpath] If you set the \varName{CLASS\-PATH\_ROOT} variable in
your Jikes RVM configuration file, the build process will use the
libraries that are located there.  The setup must mimic the one
generated automatically, which means that
{\tt \$CLASS\-PATH\_ROOT/\-\$(con\-fig.guess)} must contain a build for your platform
and {\tt \$CLASS\-PATH\_ROOT/\-class\-path} must be a version of GNU Classpath from
CVS.\@  This configuration is mainly useful if (1) you want to share one
copy of the Classpath libraries among multiple Jikes RVM users or (2)
you want to do Classpath development using Jikes RVM.\@ The
current release of Jikes RVM works with GNU Classpath version
\classpathversion.  \link{We discuss more about how to set
{\tt \$CLASS\-PATH\_\-ROOT} \texorhtml{in section~\Ref{} (page~\Pageref)}{elsewhere}}{sec:installDetails}.

\item[Automatic GNU Classpath] If you do not set \varName{CLASS\-PATH\_\-ROOT} and 
{\tt RVM\_\-WITH\_\-GNU\_\-CLASS\-PATH=1} then the Jikes RVM build process
will use {\tt wget} to download the appropriate version of GNU Classpath (using
the URL in {\tt \$RVM\_\-ROOT/\-rvm/\-bin/\-class\-path.stamp}) and build it for
you. 
\end{description}

Some library classes require special VM support.  This is provided by
the classes found in {\tt rvm/\-src/\-vm/\-lib\-Sup\-port}. 

