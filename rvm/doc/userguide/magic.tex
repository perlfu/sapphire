This section provides information on ``magic'' which is an escape
hatch that Jikes\TMweb{} RVM provides to implement
functionality that is not 
possible using the pure Java\TMweb{} programming language.  For example, the Jikes RVM garbage
collectors and 
runtime system must, on occasion, access memory or perform unsafe
casts.  Users are {\it strongly} discouraged from using magic in their code
except where absolutely necessary.  

There are currently three types of magical operations which are
described in the remainder of this section.  The first is a collection
of magical methods that are static methods of the class {\tt
VM\_Magic}.  The second is the magical classes {\tt VM\_Address}, {\tt
VM\_Word}, {\tt VM\_Offset}, and  {\tt VM\_Extent} used in
parts of the runtime and garbage collector. The third is various
mechanisms to declare code {\em uninterruptible}.

\subsection{VM\_Magic}
\index{magic methods}
\index{VM\_Magic}
\index{semantic inlining}
Certain methods in the class \xlink{{\tt VM\_Magic}}{\VMMagicURL} are
treated differently by the compiler. Because these methods access raw
memory or other machine state, perform unsafe casts, or are operating
system calls, they cannot be implemented in Java code.  A
Jikes\TMweb{} RVM implementor must be {\em extremely careful} when
writing code that uses {\tt VM\_Magic} to circumvent the Java type
system.  The use of {\tt VM\_Magic.objectAsAddress} to perform various
forms of pointer arithmetic is especially hazardous, since it can
result in pointers being ``lost'' during garbage collection.  All such
uses of magic must either occur in uninterruptible methods (see below)
or be guarded by calls to {\tt VM.disableGC} and {\tt VM.enableGC}.
The optimizing compiler performs aggressive inlining and code motion
and not explictly marking such dangerous regions in one of these two
manners will lead to disaster.

Since magic is inexpressible in the Java programming language , it is
unsurprising that the bodies of {\tt VM\_Magic} methods are undefined.
Instead, for each of these methods, the Java instructions to generate
the code is stored in
\xlink{{\tt OPT\_GenerateMagic}}{\OPTGenerateMagicURL} and 
\xlink{{\tt OPT\_GenerateMachineSpecificMagic}}{\OPTGenerateMachineSpecificMagicURL} (to generate HIR) and 
\xlink{{\tt VM\_Compiler}}{\VMCompilerURL} (to generate assembly code)\footnote{The optimizing
compiler always uses the set of instructions that generate HIR; the
instructions that generate assembly code are only invoked by the
baseline compiler.}.  Whenever the compiler encounters a call to one of these
magic methods, it inlines appropriate code for the magic method into the caller method.

\subsection{VM\_Address}
\index{VM\_Address}
The type {\tt VM\_Address} is used to represent a machine-dependent
address type.  In the past, the base type {\tt int} was used to
represent addresses but this approach had several shortcomings.
First, the lack of abstraction makes porting nightmarish.  Equally
important is that Java type {\tt int} is signed whereas address are
more appropriately considered unsigned.  The difference is problematic
since an unsigned comparison on {\tt int} is inexpressible in the Java
programming language.

To overcome these problems, instances of {\tt VM\_Address} are used to
represent addresses.  The class supports the expected well-typed
methods like adding an integer offset to an address to obtain another
address, computing the difference of two addresses, and comparing
addresses.  Other operations that make sense on {\tt int} but not on
addresses are excluded like multiplication of addresses.  Two methods
deserve special attention: converting an address into an integer and
the inverse.  These methods should be avoided where possible.

Without special intervention, using a Java object to represent an
address would be at best abysmally inefficient.  Instead, when the
Jikes compiler encounters creation of an address object, it will
return the primitive value that represents an address for that
platform.  Currently, the address type maps to a 32-bit unsigned
integer.  Since an address is not really an object, the following must
be kept in mind:

\begin{enumerate}
\item{} Do not pass a {\tt VM\_Address} instance where an {\tt Object}
is expected. This will type-check, but it is {\em not} what you want.  A
corollary is to avoid overloading a method where the two overloaded
versions of the method can only be distinguished by operating on an
{\tt Object}  versus a {\tt VM\_Address}. 
\item{} Do not synchronize on a {\tt VM\_Address} instance.
\item{} Due to a current shortcoming in the way {\tt VM\_Address} works, do not make an
array of {VM\_Address} values.  Instead make a
{VM\_AddressArray}. Creating {VM\_AddressArray[]} works as expected
and is allowed.
\end{enumerate}


\subsection{What are the Semantics of Uninterruptible Code?}
\index{uninterruptible}
\index{interruptible}
\index{VM\_Uninterruptible}
\index{VM\_PragmaInterruptible}
\index{VM\_PragmaUninterruptible}
\index{VM\_PragmaLogicallyUninterruptible}
\index{yield point}
Declaring a method uninterruptible enables a Jikes RVM developer to
prevent the Jikes RVM compilers from inserting ``hidden'' thread
switch points in the compiled code for the method.  As a result, the
code can be written assuming that it cannot involuntarily ``lose
control'' while executing due to a timer-driven thread switch. In
particular, neither yield points nor stack overflow
checks will be generated for uninterruptible methods. 

When writing uninterruptible code, the programmer is restricted to a
subset of the Java language.  The following are the restrictions on
uninterruptible code.
\begin{enumerate}
\item{} Because a stack overflow check represents a potential yield
point (if GC is triggered when the stack is grown), stack overflow
checks are omitted from the prologues of uninterruptible code.  As a
result, all uninterruptible code must be able to execute in the
stack space available to them when the first uninterruptible method on
the call stack is invoked.  This is typically about 8K for
uninterruptible regions called from mutator code.  The collector
threads must preallocate enough stack space, since all collector code
is uninterruptible. As a result, using recursive methods in the GC
subsystem is a bad idea.
\item{} Since no yield points are inserted in uninterruptible code,
there will be no timer-driven thread switches while executing it.  So,
if possible, one should avoid ``long running'' uninterruptible methods
outside of the GC subsystem.

\item{} Certain bytecodes are forbidden in uninterruptible code,
because Jikes RVM cannot implement them in a manner that ensures
uninterruptibility. The forbidden bytecodes are: {\instruction aastore};
{\instruction invokeinterface}; {\instruction new}; {\instruction
newarray}; {\instruction anewarray}; {\instruction athrow};
{\instruction checkcast} and
{\instruction instanceof} unless the LHS type is a final class;
{\instruction monitorenter},
{\instruction monitorexit}, {\instruction multianewarray}. 
\item{} Uninterruptible code cannot cause class loading and thus must
not contain unresolved getstatic, putstatic, getfield, putfield,
invokevirtual, or invokestatic bytecodes. 
\item{} Uninterruptible code cannot contain calls to interruptible
code. As a consequence, it is illegal to override an uninterruptible
virtual method with an interruptible method.
\item{} Uninterruptible methods cannot be synchronized. 
\end{enumerate}

We have augmented the baseline compiler to print a warning message
when one of these restrictions is violated.  Because there are still a
small number of violations in Jikes RVM, this checking is not enabled
by default, but can be enabled by setting {\tt
VM\_Configuration.VerifyUnint} to  true. 
If uninterruptible code were to raise a runtime exception such
as NullPointerException, ArrayIndexOutOfBoundsException, or
ClassCastException, then it could be interrupted.  We assume that such
conditions are a programming error and do not flag bytecodes that
might result in one of these exceptions being raised as a violation of
uninterruptibility. Checking for a particular method can be disabled
by having the method throw the exception {\tt
VM\_PragmaLogicallyUninterruptible}. This should be done with extreme
care, but in a few cases is necessary to avoid spurious warning
messages. 

The following rules determine whether or not a method is
uninterruptible.
\begin{enumerate}
\item{} All class initializers are interruptible, since they
can only be invoked during class loading.
\item{} All object constructors are interruptible, since they an
only be invoked as part of the implementation of the new bytecode.
\item{} If a method throws the exception {\tt
VM\_PragmaInterruptible} then it is interruptible.
\item{} If none of the above rules apply and a method throws the
exception {\tt VM\_\-Prag\-ma\-Un\-in\-ter\-rup\-ti\-ble}, then it is uninterruptible.
\item{} If none of the above rules apply and the declaring class of
the method directly implements the interface {\tt VM\_\-Un\-in\-ter\-rup\-ti\-ble}
then it is uninterruptible.
\end{enumerate}
Whether to use {\tt VM\_Uninterruptible} or the {\tt
VM\_PragmaUninterruptible} is a matter of taste and mainly depends on
the ratio of interruptible to uninterruptible methods in a class.  If
most methods of the class should be uninterruptible, then implementing
{\tt VM\_Uninterruptible} is preferred. 

