This Section provides additional information on the implementation
of the memory management component of \jp\ runtime system.
 
\subsection{Directory Structure} \label{sssec:directories}
The classes related to memory management are contained in the 
src/vm/allocator directory.  Sub-directories of allocator 
contain classes for a specific memory management strategy.
For example, the allocator/copyingGC directory implements a
simple semi-space copying allocator and collector.

A \jp\ configuration specifies a memory management strategy, 
and the jconfigure command will include in the build all the
classes in the allocator directory and all the classes in the
sub-directory for the specified memory management strategy.
The allocator directory contains classes common to multiple
configurations, such as the load balancing work queue (see
VM\_GCWorkQueue.java) which is used by all of the \jp\
garbage collectors.

\subsection{Choosing an Allocator/Collector} \label{sssec:choosinggc}
Depending on your purposes, you may choose to build \jp\
with a copying, noncopying (mark-sweep), or hybrid collector.
For programs which do not perform many collections, the
copying collector will most likely perform best, since object
allocation path lengths are shortest for this collector.  To
minimize garbage collection delays, the hybrid collector may
be used, since it allows a small nursery for which minor collections
can be quite fast, and major collections are noncopying.  

Generally, the noncopying (mark-sweep)
collector requires smaller heap storage for a given application	
than a copying collector, which in \jp\ uses a semispace structure.
Also, major collections are faster with a noncopying collector,
since objects are not moved.  The best performance for a particular	
application should be determined by experiment.

\subsection{Adding A New Allocator/Collector} \label{sssec:newalloc}
\index{garbage collection}
\index{stop-the-world garbage collection}
It is not difficult to add your own Allocator/Collector to \jp\,
especially if it uses the same ``stop the world'' parallel collection
strategy used by all the collectors in this release.  The basic steps
are:
\index{jconfigure script}
\index{VM\_Allocator class}
\begin{enumerate}
\item Create a new directory for the Allocator/Collector, such as ``newGC''.
\item Modify bin/jconfigure to add a stanza(s) for your new Allocator/Collector,
which includes your new directory in the build.  Name it appropriately, such as
``BaseBasenewGC''.
\item Copy some existing VM\_Allocator.java file into your new directory,
and modify it, choosing one that has similar properties, such as copying or
non-copying.  If you want to start from scratch, start with VM\_Allocator in noGC.
This file provides simple implementations of the required object allocation
methods, and stubs for the other fields and methods expected by the rest
of the \jp\ runtime (not all will be needed in any one implementation).
\end{enumerate}

The provided collectors contain a number of common methods which you
will want to use (``use'' presently means copy into your VM\_Allocator file).
Some of them are:
\begin{description}
\item[scanStatics] which scans all class static variables (in the JTOC) and
reports all references.
\item[scanObjectOrArray] which scans an object or array and reports any references
in that object or array.
\item[scanStack] which scans a threads stack and reports any references in
local variables or saved registers.  This will use make use of the compiler
specific StackMapIterators and the compiler generated stack maps for GC safe points.
All threads, except the executing collector threads, will be stopped a safe points.
\end{description}

\subsection{Load Balancing Work Queue} \label{sssec:workqueue}
\index{VM\_GCWorkQueue class}
\index{finalizable objects}
\index{finalizer method}
The class VM\_GCWorkQueue, implements a load balancing workqueue
which is used by all the \jp\ collectors to find all references reachable from
some initial set of references.  It is used to find all objects reachable
from roots, and also to find all objects reachable from finalizable objects
that are made live in order to execute their finalizer methods.

The work queue is implemented using thread local ``get'' and ``put'' buffers
and a shared pool of buffers needing to be processed.  The performance of the
work queue is affected by the size of these buffers.  The default size is 1024 
entries, but can be altered by the command line argument ``-X:wbsize=nnn'' where
nnn is the maximum number of entries in the buffer.  When running with multiple
processors, better load balancing has been observed with smaller buffers, for
example with 256 entries per buffer.  

\subsection{Generational Write Barrier} \label{sssec:writebarrier}
\index{VM\_WriteBuffer class}
\index{write buffer}
\index{write barrier}
\index{object header}
\jp\ provides a class VM\_WriteBarrier to support generational garbage 
collection. When the static final field VM\_Allocator.writeBarrier is true,
all \jp\ compilers generate write barrier code, which tests a specified
bit in the object header of an object into which a ref is being stored,
and, if the bit is set to one, makes an entry into a write buffer of the 
address of the object, and sets the bit to zero.  Thus generational 
collectors can use this class to 
identify objects which may contain pointers to objects 
younger than themselves.  During garbage collection, collectors should
turn the bit back to one, so that subsequent mutator execution that stores
refs into such objects will be recorded.

Currently, \jp\ copying generational and hybrid 
`collectors treat all objects which
survive one garbage collection as old.  The noncopying (mark-sweep) 
generational contains an as yet very experimental capability to require
objects to survive more than one collection before being treated as old.
(The constants GC\_OLD, GC\_STEPS, and GC\_THRESHHOLD) used in this 
collector is part of the implementation.)


\subsection{Starting Garbage Collection} \label{sssec:startgc}
\index{VM\_Handshake class}
\index{VM\_CollectorThread class}
\index{stop-the-world garbage collection}
\index{concurrent garbage collection}
The stopping of mutator threads and the scheduling of the collector
threads is handled by the classes VM\_Handshake and VM\_CollectorThread.
Mutator threads initiate collection by calling the collect method
of VM\_CollectorThread.  It causes collector threads to be scheduled
on all the VM\_Processors, and then yields, to allow the collector thread
on the executing processor to be scheduled. When all collector threads
arrive at the first Rendezvous, in the run method of VM\_CollectorThread,
the collection begins.
All collector threads 
call the gc\_collect method of VM\_Allocator.  They execute
in parallel in this method until collection is complete.  By default,
there is one collector thread for each VM\_Processor (system thread).
While collection is in progress, all non-collector threads are suspended
in queues of the VM\_Processors waiting to be re-dispatched.

If you want something other than stop-the-world parallel collection 
(such as a concurrent collector) then it will be necessary to modify
or extend the VM\_Handshake and VM\_CollectorThread classes.

\subsection{Measuring Collector Performance} \label{sssec:verbosegc}
\index{command-line arguments}
\index{VM\_Allocator class}
When the ``-verbose:gc'' command line argument is specified, the time
spent in each collection will be written out after each collection.
In addition, summary statistics are generated when \jp\ exits, which
specify the number of collections, and average and maximum collection times.
For generational collectors, this is given for Minor and Major collections.
Some collectors provide additional information, that can be measured
with minimal cost, such a number of bytes copied for the copying collectors.

Compile time flags can be set to cause additional information, at additional
cost, to be measured and reported.  Some of the more useful ones are
described here, others are described in the various VM\_Allocator
source files.

\paragraph{TIME\_GC\_PHASES}
This flag will cause the collector to measure the time spent in each
of the phases of garbage collection, such as stopping mutators, finding roots,
marking reachable objects, and finalization.  Turning on this flag will cause
the summary statistics, with average times in each phase, to be generated 
when \jp\ exits.  If -verbose:gc is specified the output is generated after
each collection. Turning on TIME\_GC\_PHASES in the common VM\_CollectorThread
class will cause it to be set in what ever VM\_Allocator is being used.

\paragraph{MEASURE\_WAIT\_TIMES}
This flag will cause the collector to measure the time each collector thread
spends waiting during a collection.  This is time waiting for buffers while
processing the Work Queue, and time waiting in Rendezvous between phases of the
collection process. Turning on this flag will cause
the summary statistics, with average wait times, to be generated 
when \jp\ exits.  If -verbose:gc is specified the output is generated after
each collection, for each collector thread.
Turning on MEASURE\_WAIT\_TIMES in the common VM\_CollectorThread
class will cause necessary flags in VM\_Allocator and VM\_GCWorkQueue to be set.

\begin{description}
\item[RENDEZVOUS\_TIMES] prints per thread entrance and exit times for the
rendezvous during each collection.
\item[GC\_COUNT\_BY\_TYPES] prints counts of the number of live objects grouped
by type.
\item[WORK\_QUEUE\_COUNTS] counts work queue buffers processed by each collector
thread (in VM\_GCWorkQueue). 
\item[COUNT\_GETS\_AND\_PUTS] counts object references processed by each collector
thread (in VM\_GCWorkQueue). 
\end{description}
Not all flags are available in all VM\_Allocator files.

\subsection{Expected Revisions} \label{sssec:revisions}
At the time of this release, some revisions to the
memory management component of \jp\
are being planned.  It is likely that they will appear in
the next release.

\paragraph{Revised Directory Structure}
With the current directory structure, all files in the allocator
directory get compiled and included in all builds, even though
only some subset will get used in any particular build.  The 
files in the allocator directory will probably be moved into strategy
specific subdirectories, such as for copying and non-copying 
collectors, so that builds of \jp\ need only include the files
needed in a specific build.

\paragraph{Factoring Out Common Code}
The strategy specific VM\_Allocator files have much code in common,
particularly for the different copying collectors.  This common code
will be factored out into separate classes, and moved into a
separate directory, where it can be shared by the different collectors.

\paragraph{Splitting Allocation and Collection}
Presently the VM\_Allocator files contain both the support for
object allocation and garbage collection.  It is likely that
the methods implementing collection will be split out into
a separate VM\_Collector class.  All the VM\_Allocator classes use
the same implementation for Large Object Space, and it is likely
that this will also be split out into a separate class.

\subsection{Optional Builds} \label{ssses:optionalbuilds}
Some of the memory management classes have internal compile
time flags that can be modified to alter the way allocation 
and collection is performed.
(Note that the optional builds have not been recently tested)

\paragraph{Thread Private Work Queues}
\jp\ originally used private work queues for each collector thread
participating in collection.  This has been replaced by the current
load balancing work queue strategy.  It is possible to build \jp\
with the old thread private queues by setting the flag
USE\_OLD\_PRIVATE\_WORKQUEUE in VM\_GCWorkQueue.java to true.
(This is not recommended, but can be used to observe the relative
performance of the two approaches)

\paragraph{Turn Off Local Allocation} 
By default, VM\_Processors allocate objects from local ``chunks'', without
needing to synchronize with other VM\_Processors, except when acquiring
an additional chunk from the shared global heap.  Flags in the
copying allocators allow one to turn this off, causing all allocations
to be synchronized. 
The PROCESSOR\_LOCAL\_ALLOCATES flag controls processor local object
allocations, and the PROCESSOR\_LOCAL\_MATURE\_ALLOCATES flag controls
collector thread local allocation of space for marked objects being
copied. (These should only be turned off if one is trying to observe
the impact of cache contention on a multi-processor computer)

\paragraph{Alternatives For Zeroing Memory}
By default, VM\_Processors zero chunks of memory, when the chunks are
acquired from the shared global heap, rather than zeroing freed
space at the end of collection.  In some of the copying Collectors,
compile time flags can be set to alter when and how the zeroing is
done.  The options are:
\begin{description}
\item[Zero Chunks When Allocated] (the default)
\item[Zero Nursery In Parallel] The collector threads divide up the
Nursery (free space) and zero their assigned portion in parallel.
This occurs at the end of collection.
\item[One Thread Zeros] One collector thread zeros the entire Nursery
at the end of collection. (A bad idea, but shows how long zeroing can take)
\end{description}



