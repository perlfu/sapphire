This section provides some tips on collecting performance numbers with
Jikes RVM.

\index{boot image}
\index{configurations}
\subsection{Which boot image should I use?}

To make a long story short the best performing configuration of Jikes
RVM will almost always be {\tt FastAdaptive}{\em<some GC>}.  Unless you really
know what you are doing, don't use any other configuration to do a 
performance evaluation of Jikes RVM. In most cases,
{\tt FastAdaptiveSemispace} is likely to be your best choice. You can
use the {\tt FastSemispace} configuration if you want to force all
dynamically loaded methods to be compiled by the optimizing compiler,
but this configuration will have extremely poor startup behavior and
will not perform any profile-directed optimizations. 

Any bootimage you use for performance evaluation must have the
following characteristics for the results to be meaningful:
\begin{itemize} 
\item {\tt RVM\_WITHOUT\_ASSERTIONS=1}. Unless this is set, the runtime
system and optimizing compiler will perform fairly extensive assertion
checking. This introduces significant runtime overhead. By convention,
a configuration with the {\tt Fast} prefix disables assertion
checking.
\item {\tt RVM\_WITH\_OPT\_BOOTIMAGE\_COMPILER=1}. Unless this is set, the
bootimage will be compiled with the baseline compiler and virtual
machine performance will be abysmal.  Jikes RVM has been designed
under the assumption that aggressive inlining and optimization will be
applied to the VM source code. 
\item Any configuration that performs opt compilation at runtime (
{\tt RVM\_WITH\_ADAPTIVE\_SYSTEM=1} or {\tt
RVM\_WITH\_OPT\_RUNTIME\_COMPILER=1}) should be built with {\tt
RVM\_WITH\_ALL\_CLASSES=1}.  This includes the optimizing compiler and
associated support classes in the bootimage where they can be
optimized by the bootimage compiler. By convention, configurations
that include the opt compiler in the bootimage have the {\tt Full} or
{\tt Fast} prefix.  Configurations where {\tt RVM\_WITH\_ALL\_CLASSES}
is not set to 1 that use the optimizing compiler will dynamically load
it (which will force it to be baseline compiled, {\em even if
RVM\_WITH\_OPT\_RUNTIME\_COMPILER is set to 1}).
\end{itemize}

\subsection{What command-line arguments should I use?}

For best performance we recommend the following:

\begin{itemize}
\item {\tt -processors all}: By default, Jikes\trademark RVM uses only
one processor.  Setting this option tells the runtime system to
utilize all available processors. 
\item {\tt -X:irc:O2}: For non-adaptive configurations, this command-line option tells the optimizing compiler to use our highest level of optimization.
\item Set the heap and large heap sizes generously.  We typically set the heap size to at least half the physical memory on a machine.
\item Use a dedicated machine with no other users.  The Jikes RVM thread and synchronization implementation do not play well with others.
\end{itemize}

\JikesTMFooter

\subsection{Jikes RVM is really slow! What am I doing wrong?}

Perhaps you are not seeing stellar Jikes\trademark RVM performance.
If Jikes RVM as 
described above is not competitive with the IBM AIX\AIXTMFootnote or
Linux/IA32 product DK, we recommend you test your installation with
the SPECjvm98 benchmarks.  We expect Jikes RVM performance to be competitive
with the IBM DK 1.3.0 on the SPECjvm98 benchmarks.

Of course, SPECjvm98 does not guarantee that Jikes RVM runs all codes
well.  We have also tested various flavors of pBOB and the Volano
benchmarks, and usually see superior or competitive performance.

The IA32 port is somewhat less mature than the PPC port, and does not
deliver competitive performance on some codes.  In particular, IA32
floating-point performance is mediocre.

Some classes of codes will not run fast on Jikes RVM.  Known issues include:
\begin{itemize}
\item Jikes RVM start-up is slow compared to the IBM product JVM.
\item Remember that the non-adaptive configurations (eg. Fast) opt-compile
{\em every} method the first time it executes.  With aggressive optimization
levels, opt-compiling will severely slow down the first execution of
each method.  For many benchmarks, it is possible to test the quality
of generated code by either running for several iterations and ignoring
the first, or by building a warm-up period into the code.  The SPEC benchmarks
already use these strategies.  The adaptive configuration does not
have this problem; however, we cannot stipulate that the adaptive
system will compete with the product on short-running codes of a few seconds.
\item We expect Jikes RVM to perform well on codes with many threads, such as
VolanoMark.  However, if you have a code with many threads, each using
JNI, Jikes RVM performance will suffer due to factors in the design of
the current thread system.
\index{on-stack replacement}
\item Jikes RVM does {\em not} yet support on-stack replacement for
optimizing methods.  The adaptive system will not optimize a single
invocation of a long-running 
method.
\index{quasi-preemption}
\item Performance on tight loops may suffer.  The Jikes RVM thread system
relies on quasi-preemption; the optimizing compiler inserts a thread-switch
test on every back edge.  This will hurt tight loops, including many
simple microbenchmarks.  We should someday alleviate this problem by
strip-mining and hoisting the yield point out of hot loops.
\item The thread system currently uses a spinning idle thread. If a
Jikes RVM
virtual processor (ie., pthread) has no work to do, it spins chewing up
cpu cycles.  Thus, Jikes RVM will only perform well if there is no other activity on the machine.
\item The load balancing in the system is naive and unfair.  This can hurt some styles of codes, including bulk-synchronous parallel programs.
\item The adaptive system may not perform well on SMPs; this may be due to bad
interaction with the thread load balancer.
\end{itemize}

The Jikes RVM developers wish to ensure that Jikes RVM delivers
competitive performance. 
If you can isolate reproducible performance problems, please let us
know. 

\AIXTMFooter