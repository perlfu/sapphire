Some of the code in Jikes\TMweb{} RVM is run through a preprocessor before the
real Java\TMweb-to-byte-code compiler ever sees it.  As of July, 2003, preprocessor
constructs are in 128 of Jikes RVM's 1045 Java source files.

\subsection{Usage}

Here is the help message the preprocessor now displays; this
is a placeholder until someone writes more attractive documentation
for it.

\begin{verbatim}

Usage: preprocessModifiedFiles [--help] [--trace]
        [ --[no-]undefined-constants-in-conditions ]
        [ --[no-]only-boolean-constants-in-conditions ]
        [ --disable-modification-exit-status ]
        [ -D<name>[ =1 | =0 | =<string-value> ] ]...
        [ -- ] <output directory> [ <input file> ]...
   Preprocess source files that are new or have changed.

   The timestamp of each input file is compared with that
   of the corresponding file in the <output directory>.  If the
   output file doesn't exist, or is older than the input file,
   then the input file is copied to the <output directory>, 
   with preprocessing.

   Invocation parameters:
      - zero or more definitions of preprocessor constants
        of the form "-D<name>=1",
        of the equivalent shorthand form "-D<name>", 
        of the form "-D<name>=0",
        and/or of the form "-D<name>=<string-value>".
      - name of directory to receive output files
      - names of zero or more input files
      - other flags

   Process exit status means:
           0 - no files changed
           1 - some files changed
       other - trouble

   With --disable-modification-exit-status, the process will
   exit with status 0 even when some files changed.  Under
   --disable-modification-exit-status, non-zero exit status
   always means trouble.  This is handy inside Makefiles.


   --trace  The preprocessor prints a 
         '.' for each file that did not need to be changed and a 
         '+' for each file that needed preprocessing.
   --verbose, -v  The preprocessor prints a message for each file
         examined, and prints a summary at the end 

   --help, -h  Show this long help message and exit with status 0.

   --keep-going, -k  Keep going in spite of errors; still exit with bad status.

   -D<name>=0 is historically a no-op; equivalent to never defining <name>.
   (However, the --no-undefined-constants-in-conditions flag changes that
    behavior, requiring that any <name> in an //-#if <name> directive.
    be defined with -D<name>=0 or -D<name>=1.)

   -D<name>=1 and -D<name> are equivalent.

   -D<name>=<any-string-value-but-0-or-1> will define a constant that is
       usable in a //-#value directive.

   The following preprocessor directives are recognized
   in source files.  They must be the first non-whitespace characters
   on a line of input.

      //-#if    <name>
           Historically, it is not an error for <name> to have never been
           defined; it's equivalent to -D<name>=0.  This can be 
           experimentally promoted to an error with the flag
           --no-undefined-constants-in-conditions.  The historical
           behavior (default) is explicitly requested with
	    --undefined-constants-in-conditions
           
           Historically, //-#if only checks whether <name> is defined.
           If you specify --only-boolean-constants-in-conditions, then
           you get stricter behavior where <name> must've been defined with
           -D<name>=1 or -D<name>=0.

           "//-#if" also supports the constructs '!' (invert the sense of 
           the next test), '&&', and '||'.  '!' binds more tightly 
           than '&&' and '||' do.   '&&' and '||' are at the same precedence,
           and are short-circuit evaluated in left-to-right order.

           The preprocessor does not support parentheses in //-#if constructs.
           If you don't mix '&&' and '||' in the same line, you'll be OK.

      //-#elif  <name>
            Takes the same arguments that //-#if does. 

      //-#else  <optional-comment>

      //-#endif <optional-comment>

      //-#value <preprocessor-symbol>
           <-preprocessor-symbol> is the name of a constant defined on the
           command line with -D; it will be replaced with the defined value.

           It is an error for <preprocessor-symbol> not to be defined.

           It is an error for <preprocessor-symbol> to have been defined
           with -D<name>=1 or with -D<name>

          (This is an odd restriction, but is the way the code was written
           when I found it.  You're free to rewrite it if you want it to act
           just like the C preprocessor does.)

     There is no equivalent to the C preprocessor's "#define" construct;
     all constants are defined on the command line with "-D".
\end{verbatim}

\subsection{Where It Is}

The source code for the preprocessor is in {\tt
  rvm/src/tools/preprocessor}.  The Jikes RVM build process installs a
preprocessor executable for the host as as {\tt
  RVM\_BUILD/jbuild.prep.host} and one for the target at {\tt
  RVM\_BUILD/jbuild.prep.target}.  The preprocessing is done
automatically during the build process.  The preprocessor is just an
executable program; you can use it yourself for other projects without
using the rest of Jikes RVM.\@


\subsection{Why use a preprocessor?}

The Java programming language has a conditional compilation idiom that tries to make up for its
lack of a preprocessor.  In the idiom, a programmer can
declare a {\tt final boolean} constant such as {\it on\_platform\_xyz}
and then write {\tt if} statements using that constant, such as:
\begin{example}
if ({\it on\_platform\_xyz}) \{
   doSomethingIOnlyNeedToDoForPlatformXYZ();
\}    
\end{example}

This is a good substitute in many cases.  However, it doesn't work for
Jikes RVM; we need to (for example) work with processor words in some of
the code.  On some platforms they're 64 bits long (a Java {\tt long})
and on some they're 32 bits long (a Java {\tt int}).  The Java
programming language has no way,
using the conditional compilation idiom, to write methods that return
a {\tt long} in some situations and a {\tt int} in others.

