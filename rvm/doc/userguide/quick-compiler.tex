Release 2.3.4 marks the first public appearance of the ``Quick''
compiler for the PowerPC architecture targets.  

\xname{quick_compiler_goals}
\subsection{Goals}

The Quick compiler is supposed to be rather smarter than the Base
compiler about register use without being nearly as complex as the Opt
compiler. 

Rather smarter means that stack and local variables are kept in
registers if possible, not in memory. 

Not nearly as complex means, for instance, that the Quick compiler
code generator works in a single pass, with no intermediate
representations. Right now the only additional optimizations are to
remember if stack or local variable values are already in a temporary
or work register and hence don't need to be loaded, and to postpone
local variable stores until it becomes obvious that the value really
needs to be written. 

One purpose of the Quick compiler is to explore how close to the
performance of the Opt compiler we can come by using just the most
beneficial optimizations. Also, we hope to use it as a starting point
for experimenting with other optimizations without having to deal with
the implementation details of the Opt compiler and the complex
interactions among the many optimizations performed by the Opt
compiler. 

\xname{quick_compiler_why_use_it}
\subsection{Why use it?}

Baseline compiled code has horrible performance on modern (Power4,
Power5) PowerPC implementations.  The very frequent store/load to the
same memory location (a.k.a.\ the expression stack) causes a large number
of mis-speculations and replays. 

As of this writing (December 22, 2004), we are discussing moving to
using the quick compiler (in its simplest, fastest form) as a more or
less complete replacement for the base compiler on PowerPC.  Our guess
is that this is a more attractive option than injecting the quick
compiler as another level available to AOS.  It would be ok to rely on
the baseline compiler for some fringe stuff (DynamicBridge,
JNIFunctions, complex magics), but we'd like to see all normal Java
going to the quick compiler for initial compilation.

\xname{quick_compiler_status}
\subsection{Status}

As of this writing, the quick compiler is not part of the default
configurations (\textbf{prototype}, \textbf{prototype-opt},
\textbf{development}, and \textbf{production}).  It's been fairly
stable, but part of the gctest regression continues to fail, for
instance. 

