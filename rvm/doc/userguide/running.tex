This section describes how to run an RVM  image built from the previous
section. 

%%%%%%%%%%%%%%%%%%%%%%%%
\subsection{Running RVM}

\index{jikes}
\index{rvm script}
RVM executes bytecodes from {\tt .class} files. It does {\em not} compile
Java\trademark source code. Therefore, all files 
required by your program must have already been compiled into bytecode
files by a Java\trademark compiler.  We recommend you use the IBM Jikes\trademark compiler.
The current distribution includes an executable version of Jikes\trademark in
{\tt contrib/jikes}.

For example, to run class {\tt foo} with source code in file {\tt foo.java}:
\begin{verbatim}
% jikes foo.java
% jalapeno foo 
\end{verbatim}

The general syntax is
\begin{verbatim}
jalapeno [rvm options] class [args]
\end{verbatim}

\index{command-line options}
You may choose from myriad options for the RVM command-line.  
Options fall into two categories: {\em standard} and {\em
non-standard}.  Non-standard options are preceded by {\bf ``{\tt -X:}''}.

%%%%%%%%%%%%%%%%%%%%%%%%%%%%%%%%%%%%
\subsubsection{Command-Line Options}

We currently support a subset of the JDK 1.3 standard options.  Below
is a list of all options and their descriptions.  Unless otherwise noted each
option is supported in RVM.
\begin{description}
\item[{\tt -cp -classpath <directories and zip/jar files separated by :>}]
set search path for application classes and resources

\item[{\tt -verbose[:gc|:jni|:class]}]
enable verbose output; currently ``jni'' is not supported

\item[{\tt -version}] print current VM version and terminate

\item[{\tt -showversion}] print current VM version and continue

\item[{\tt -?} or {\tt -help}] print help message

\item[{\tt -X}] print help on non-standard options

\item[{\tt -jar}] not supported

\end{description}

The non-standard options are

\begin{description}
\item[{\tt -X:h=<number>}]
allocate {\tt <number>} megabytes of small object heap (default=20)

\item[{\tt -X:mx<number>}]
allocate {\tt <number>} megabytes of small object heap (default=20)

\item[{\tt -X:lh=<number>}]
allocate {\tt <number>} megabytes of large object heap (default=10)

\item[{\tt -X:nh=<number>}]
allocate {\tt <number>} megabytes of nursery object heap (default=10,
ignored if no nursery)

\item[{\tt -X:ph=<number>}]
allocate {\tt <number>} megabytes of permanent object heap (default=0)

\item[{\tt -X:i=<filename>}]
read boot image from {\tt <filename>}

\item[{\tt -X:sysLogfile=<filename>}]
writes standard error message to {\tt <filename>}

\item[{\tt -X:rvmClasses=<filename>}]
load classes from {\tt <filename>}

\item[{\tt -X:cpuAffinity=<int>}]
physical cpu to which first virtual processor is bound

\item[{\tt -X:processors=<int|``all''>}]
number of processors to use on a multiprocessor

\item[{\tt -X:verbose}]
print out additional information for GC

\item[{\tt -X:irc[:help]}]
print options supported by runtime compiler when in a non-adaptive configuration 

\item[{\tt -X:irc<option>}]
pass {\tt <option>} to the runtime compiler when in a non-adaptive configuration 

\item[{\tt -X:aos[:help]}]
print options supported by adaptive optimization system when in an
adaptive configuration

\item[{\tt -X:aos<option>}]
pass {\tt <option>} to the adaptive optimization system when in an adaptive configuration

\end{description}

See section~\ref{appendix:nonadaptive:cmdline} for more details on 
command-line options, including the list of options supported by the 
optimizing compiler and adaptive optimization system.


