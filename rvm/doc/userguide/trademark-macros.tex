% $Id$
%
% trademark-macros.tex

% From the SPEC page (HTML ISO Latin 1):
% [ <b>&reg;</b> = Registered Trademark, <b>&#153;</b> = Trademark
% Unicode: Registered Trademark is 0x00AE,
% Unicode: Trademark sign is 0x2122 or <super> 0x0054 0x004D
% TS1 (the Cork encoding) uses 0227 (octal)

%% First, define a symbol for TM and for (R), both in HTML and in LaTeX
%% I wonder what \texttrademark does normally.
\T \newcommand{\trademark}{\ensuremath{%
\T               \protect\raisebox{.41\baselineskip}{\tiny \textsc{TM}}}}
\T \newcommand{\registeredTrademark}{\ensuremath{%
\T		 \protect\raisebox{.41\baselineskip}{\tiny \textregistered}}}

%% The spaces do not hurt the HTML and may help it.
\W \newcommand{\trademark}{\xmlsym{##8482}}
\W \newcommand{\registeredTrademark}{\xml{sup}\xml{small}\xmlsym{reg}\xml{/small}\xml{/sup}}


%% The \*heading*{} forms of the macros are for inside
%% LaTeX \section{} and \subsection{} commands.

%% The \*web*{} forms of the macros will only appear in web pages.
%% The scoop here is that each web page which refers to a trademark
%% should have some reference to the trademarked status of that
%% same trademark.  Hyperlatex breaks our document into web pages on
%% subsection boundaries, so somewhere in each subsection where we
%% refer to a trademark, there has to be a TM or (R) in that
%% subsection, but ONLY in the HTML version.  For the printed book,
%% one or two attributions at the beginning are enough.  Those get
%% the \*both*{} forms of the macros.

%% Use the defined symbols to generate \TM and \R macros.

\newcommand{\TMboth}{\link{\trademark}{trademarks}\xspace}
\newcommand{\TMweb}{\htmlonly{\TMboth}\xspace}
%\newcommand{\TMheading}{\trademark\xspace}
\newcommand{\TMheadingweb}{\htmlonly{\trademark}\xspace}

\newcommand{\Rboth}{\link{\registeredTrademark}{trademarks}\xspace}
\newcommand{\Rweb}{\htmlonly{\Rboth}\xspace}
%\newcommand{\Rheading}{\registeredTrademark\xspace}
\newcommand{\Rheadingweb}{\htmlonly{\registeredTrademark}\xspace}
